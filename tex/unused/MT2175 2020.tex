\documentclass[../MathsNotesBase.tex]{subfiles}

\date{\vspace{-6ex}}

\begin{document}
	\subheading{MT2175 Further Linear Algebra - Candidate No. A14313}
	
	\paragraph{1.} The characteristic equation for $A$ is
	\[ 	\begin{vmatrix}
			7 - \lambda & -1\\
			a & 5 - \lambda
		\end{vmatrix} = (7 - \lambda)(5 - \lambda) - (a \times -1) = \lambda^2 - 12 \lambda + 35 + a. 
	\]
	
	\nl[8]
	(a) A real matrix cannot be diagonalised when the characteristic equation has repeated roots. So, if ${ a = 1 }$ then the characteristic equation of $A$ becomes,
	\[ \lambda^2 - 12 \lambda + 36 = (\lambda - 6)^2 \]
	which has a repeated root: ${ \lambda = 6 }$ and, in this case, $A$ is not diagonalisable.\\
	
	\[ A - \lambda I = 	\begin{bmatrix}
							1 & -1\\
							1 & -1
						\end{bmatrix}
	\]
	which has nullspace
	\[ t \begin{bmatrix}1\\1\end{bmatrix} \eqword{for} t \in \R{}. \]
	We can find a generalised eigenvector such that 
	\[ (A - \lambda I) \v = \begin{bmatrix}1\\1\end{bmatrix} \]
	which gives us
	\[ \v = \begin{bmatrix}1\\0\end{bmatrix}. \]
	So, the change of basis matrix to this jordan basis is
	\[ P = 	\begin{bmatrix}
				1 & 1\\
				1 & 0
			\end{bmatrix} 
	\]
	and so the jordan matrix is
	\[ J = \inv{P}AP = 	\begin{bmatrix}
							0 & 1\\
							1 & -1
						\end{bmatrix}
						\begin{bmatrix}
							7 & -1\\
							1 & 5
						\end{bmatrix}
						\begin{bmatrix}
							1 & 1\\
							1 & 0
						\end{bmatrix} =
						\begin{bmatrix}
							6 & 1\\
							0 & 6
						\end{bmatrix}.				
	\]
	
	\pagebreak
	(b) Roots of ${ \lambda^2 - 12\lambda + (35 + a) }$ take the form:
	\[\begin{aligned}
		\frac{12}{2} &\pm \frac{\sqrt{144 - 140 - 4a}}{2} \\
		&= 6 \pm \frac{\sqrt{4(1 - a)}}{2} = 6 \pm \sqrt{1 - a}.
	\end{aligned}\]
	\begin{enumerate}[label=(\roman*)]
		\item Real values of $a$ that make $A$ diagonalisable: ${ a < 1}$
		\item Complex values of $a$ that make $A$ diagonalisable: ${ a \neq 1 }$
	\end{enumerate}

	\nl[6]
	(c) $A$ is not diagonlisable for $a = 3$ but is for ${ a = -3 }$ so I will assume that the question meant to say ${ a = -3 }$.\\
	
	For ${ a = -3 }$, the characteristic equation becomes,
	\[ \lambda^2 - 12 \lambda + 32 = (\lambda - 8)(\lambda - 4) \]
	yielding 8 and 4 as the eigenvalues. Then the eigenvectors are the nullspaces of the matrices,
	\[
		 \begin{bmatrix}
		 	-1 & -1\\
		 	-3 & -3
		 \end{bmatrix} \eqand
	 	 \begin{bmatrix}
	 	 	3 & -1\\
	 	 	-3 & 1
	 	 \end{bmatrix}
	\]
	which are
	\[ s \begin{bmatrix}-1\\1\end{bmatrix} \eqand t \begin{bmatrix}1\\3\end{bmatrix} \eqword{for} s,t \in \R{}. \]
	So, to diagonalise $A$ we use the change of basis matrix,
	\[ P = 	\begin{bmatrix}
				-1 & 1\\
				1 & 3
			\end{bmatrix}
	\]
	to obtain
	\[ D = \inv{P}AP = 	\frac{1}{4}
						\begin{bmatrix}
							-3 & 1\\
							1 & 1
						\end{bmatrix}
						\begin{bmatrix}
							7 & -1\\
							-3 & 5
						\end{bmatrix}
						\begin{bmatrix}
							-1 & 1\\
							1 & 3
						\end{bmatrix} =
						\begin{bmatrix}
							8 & 0\\
							0 & 4
						\end{bmatrix}.
	\]
	 
%   A real matrix cannot be diagonalised if it is singular. So, if the determinant is 0 then it is not diagonalisable. Therefore, ${ a = -35 }$
%	\[ 	\begin{vmatrix}
%		7 & -1\\
%		a & 5
%	\end{vmatrix} = (7 \times 5) - (a \times -1) = 35 - (-a) = 35 + a. 
%	\]
\end{document}