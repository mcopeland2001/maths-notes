\documentclass[../MathsNotesBase.tex]{subfiles}




\date{\vspace{-6ex}}


\begin{document}	
	\searchableSection{Limits}{calculus, limits}
	\biggerskip
	\searchableSubsection{Calculating Limits}{calculus, limits}{\label{ssection:calculating-limits}
		\bigskip
		\begin{enumerate}[label=(\arabic*)]
			\item{If the expression is a composition of functions that tend to finite limits (with the exception of a denominator tending to 0) then we can use the algebra of limits (\autoref{prop:func-finite-limit-algebra}).}
			\item{Otherwise we can try to use the algebra of infinite limits (\autoref{prop:func-infinite-limit-algebra}).}
			\item{Else look for algebraic manipulations to reduce the expression to one that can be solved using (1) or (2).}
			\item{Else use L'H\^{o}pital's Rule.}
		\end{enumerate}
	
		\nl[4]
		For example, the forms:
		\begin{enumerate}
			\item{${ (\infty) - (\infty), \, (\infty) \times (-\infty) }$: These are case (3) in which we look for algebraic manipulations that reduce the problem to case (1) or (2).}
			\item{${ \frac{0}{0}, \, \frac{\infty}{\infty} }$: Case (4), L'H\^{o}pital's Rule}
			\item{${ (0) \times (\infty) }$: This case can be reduced to (4) by converting the product to a ratio.}
			\item{${ \frac{0}{\infty} }$: This case can be reduced to (1) by expressing it as ${ (0) \times (\frac{1}{\infty}) }$ and then treating the second operand (the ratio) as case (2) to obtain the form ${ (0) \times (0) = 0 }$.}
		\end{enumerate}
	
		\sep
		\begin{exe}
			\ex{$\lim_{x\to\infty} \left[\left(x^3 + x^2\right)^\frac{1}{3} - x\right]$:\\
			
				First, note that 
				\[ \lim_{x \to \infty} \left(x^3 + x^2\right)^\frac{1}{3} = \infty \eqand \lim_{x \to \infty} x = \infty. \]
				So the expression we are trying to find the limit of takes the form ${ (\infty) - (\infty) }$ and so this is case (3) and we need to look for algebraic manipulations that can transform it into one of the other cases.\\
				
				We can use the difference of cubes (\autoref{theo:difference-of-cubes}) as follows.
				\[\begin{aligned}
					&\left(x^3 + x^2\right)^\frac{1}{3} - x \\
					&= \left(x^3 + x^2\right)^\frac{1}{3} - x \cdot \frac{\left(x^3 + x^2\right)^\frac{2}{3} + x^2 + x\left(x^3 + x^2\right)^\frac{1}{3}}{\left(x^3 + x^2\right)^\frac{2}{3} + x^2 + x\left(x^3 + x^2\right)^\frac{1}{3}} \nnn
					&= \frac{\left(x^3 + x^2\right) - x^3}{\left(x^3 + x^2\right)^\frac{2}{3} + x^2 + x\left(x^3 + x^2\right)^\frac{1}{3}} &\sidecomment{by \autoref{theo:difference-of-cubes}}\nnn
					&= \frac{x^2}{\left(x^3 + x^2\right)^\frac{2}{3} + x^2 + x\left(x^3 + x^2\right)^\frac{1}{3}} \nnn
					&= \frac{\frac{1}{x^2}}{\frac{1}{x^2}}\cdot\frac{x^2}{\left(x^3 + x^2\right)^\frac{2}{3} + x^2 + x\left(x^3 + x^2\right)^\frac{1}{3}} \nnn
					&= \frac{1}{\left(1 + \frac{1}{x}\right)^\frac{2}{3} + 1 + \frac{1}{x}\left(x^3 + x^2\right)^\frac{1}{3}} \nnn
					&= \frac{1}{\left(1 + \frac{1}{x}\right)^\frac{2}{3} + 1 + \left(1 + \frac{1}{x}\right)^\frac{1}{3}}. \nnn
				\end{aligned}\]
				Now we have reduced it to an expression that is a composition of functions that have finite limits so it can now be solved as an instance of case (1).
				\[\begin{aligned}
					&\lim_{x\to\infty} \left[\left(x^3 + x^2\right)^\frac{1}{3} - x\right] \nnn
					&= \lim_{x\to\infty} \frac{1}{\left(1 + \frac{1}{x}\right)^\frac{2}{3} + 1 + \left(1 + \frac{1}{x}\right)^\frac{1}{3}} \nnn
					&= \frac{\lim_{x\to\infty} 1}{\lim_{x\to\infty} \left(1 + \frac{1}{x}\right)^\frac{2}{3} + \lim_{x\to\infty} 1 + \lim_{x\to\infty} \left(1 + \frac{1}{x}\right)^\frac{1}{3}} &\sidecomment{by \autoref{prop:func-finite-limit-algebra}} \nnn
					&= \frac{1}{1 + 1 + 1} = \frac{1}{3}.
				\end{aligned}\]
			}
		\end{exe}
	
		\bigskip
		\subsubsection{\texorpdfstring{L'H\^{o}pital's Rule}{L'Hôpital's Rule}}\label{sssection:calc_l_hopitals_rule}
		see: \href{https://web.ma.utexas.edu/users/m408n/m408c/CurrentWeb/LM4-4-11.php}{UTexas}
		\begin{itemize}
			\item{L'H\^{o}pital's Rule does not apply if the limit of derivatives does not exist (though some sources say that the limit being infinite is an 'existing' limit in this case - which would leave only oscillatory divergent functions as not having limits - needs confirmation)}
			\item{\href{https://math.stackexchange.com/questions/1342202/why-doesnt-lhopitals-rule-work-in-this-case/1342222}{math.stackexchange discussion}}
			\item{Example of L'H\^{o}pital's Rule failing to find a simple limit: 
				\[ \lim_{x \to \infty} \frac{ \sqrt{4x^2 + 3} }{ x + 3 } \]
			}
		\end{itemize}
		
		 
		\biggerskip
		\subsection{Classes of Asymptotic Behaviour}
		We can identify three classes of asymptotic behaviour of a function $f(x)$ as ${ x \to a }$ (where $a$ may be infinity), distinguished by the value of
		\[ L = \lim_{x \to a} f(x). \]
		These are functions $f(x)$ such that:
		\begin{enumerate}[label=(\roman*)]
			\item{${ L = 0 }$;}
			\item{${ 0 < L < \infty }$;}
			\item{${ L = \infty }$.}
		\end{enumerate}
	
		Functions of classes (i) and (ii) can always be combined using \autoref{prop:func-finite-limit-algebra} apart from the specific exemption if a function of class (i) is a denominator.\\
		Linear combinations including one function of class (iii) are dominated by it. For example, if $A$ is some number, we have,
		\[  A + \infty = \infty = \infty - A \]
		and
		\[ A - \infty = -\infty = -A - \infty. \]
		Products and ratios involving functions of classes (i) and (iii) are resolved by looking at the derivatives of the functions -- i.e. using L'H\^{o}pital's Rule.
		
		\medskip
		\subsubsection{Orders of Growth}
		see: \href{https://en.wikipedia.org/wiki/Asymptotic_analysis}{wikipedia - Asymptotic Analysis}\\
		
		Let a test value $T$ be defined as,
		\[ T = \lim_{t \to \infty} \frac{f(t)}{g(t)}. \]
		Since,
		\[ \frac{f(t)}{g(t)} - 1 = \frac{f(t) - g(t)}{g(t)} \]
		we therefore have
		\[ \lim_{t \to \infty} \frac{f(t)}{g(t)} = 1 \implies \lim_{t \to \infty} \frac{f(t)}{g(t)} - 1 = \lim_{t \to \infty} \frac{f(t) - g(t)}{g(t)} = 0. \]
		This will be the case if the function,
		\[ h(x) = f(x) - g(x) \]
		has a lower order of growth then $g(x)$. This can be seen most easily with polynomials. Say, for example,
		\[ f(x) = \alpha_3 x^3 + \alpha_2 x^2 + \alpha_1 x + \alpha_0 \eqand g(x) = \beta_3 x^3 + \beta_2 x^2 + \beta_1 x + \beta_0 \]
		then we have
		\[ h(x) = (\alpha_3 - \beta_3) x^3 + (\alpha_2 - \beta_2) x^2 + (\alpha_1 - \beta_1) x + (\alpha_0 - \beta_0). \]
		If ${ \alpha_3 = \beta_3 }$ then the order of $h$ is lower than that of $g$ and so ${ \frac{h(x)}{g(x)} \to 0 }$ which means that ${ T = 1 }$ and $f$ and $g$ have the same order.\\
		
		In general, if $f$ and $g$ have the same degree, then the degree of $h$ is less than or equal to the degree of $g$ and so $T$ will be 0 --- in the case that the degree of $h$ is lower than that of $g$ --- and some finite non-zero number in the case that the degree of $h$ is the same as that of $g$.\\
		
		On the other hand, if the degree of $f$ is greater than that of $g$ then the degree of $h$ will also be greater than that of $g$ and so ${ \frac{h(x)}{g(x)} \to \infty }$ and the value of $T$ will be infinite.
	}
	
\end{document}