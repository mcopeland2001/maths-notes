\documentclass[../MathsNotesBase.tex]{subfiles}




\date{\vspace{-6ex}}


\begin{document}	
	\searchableSection{Integration}{analysis, calculus, integration}
	\bigskip\bigskip
	
	\searchableSubsection{Definite and Indefinite Integration}{analysis, calculus, integration}{\label{ssection:definite-and-indefinite-integration}
		Indefinite integration determines an antiderivative up to a constant so that, if $F(x)$ is some antiderivative of $f(x)$ and $C$ is a constant, then
		\[ \int f(x) \dif x = F(x) + C. \]
		This expresses the fact that any value of $C$ would produce a valid antiderivative of $f(x)$. In fact, these are the fibres -- the cosets of the kernel -- of the differentiation linear transformation. The kernel of differentiation is the set of constant-valued functions,
		\[ f(x) = C \]
		for any ${ C \in \R{} }$.\\
		
		In the case of a definite integral, however, the constant cancels out:
		\[ \int_a^b f(x) \dif x = [F(x) + C]_a^b = (F(b) + C) - (F(a) + C) = F(b) - F(a) \]
		so that we are able to resolve the value of the definite integral to a specific constant value. In this case, the result is the sum of two elements of the kernel of the differentiation transform and so the result is also in the kernel.\\
		
		However, if we consider a function of a variable, $t$, such that
		\[ h(t) = \int_a^t f(x) \dif x \]
		then $h(t)$ is also an antiderivative of $f(t)$ but also,
		\[ h(t) = \int_a^t f(x) \dif x = [F(x) + C]_a^t = F(t) - F(a) = F(t) + C_2 \]
		where ${ C_2 = -F(a) }$ is a constant. So, the definite integral -- specifically, the initial value, $a$ -- has allowed us to resolve a particular antiderivative from the set of functions produced by the possible values of $C$. That's to say, the initial value $a$ specified for us a particular element in the kernel of the differentiation operator -- namely, ${ C_2 = -F(a) }$.\\
		
		Furthermore, since the definite integral is a particular antiderivative of $f(x)$ we can also relate the indefinite and definite integrals as follows,
		\[ \int f(x) \dif x = \int_a^t f(x) \dif x + C. \]
		This amounts to -- expressed in terms of group theory -- the statement that, if $G$ is a group with kernel $K$ and ${ x,k \in G,\; k \in K }$, then
		\[ xK = xkK \] 
		where 
		\begin{align*}
			xK &= \int f(x) \dif x = F(x) + C \\
			xk &= \int_a^t f(x) \dif x = F(t) - F(a) \\
			xkK &= \int_a^t f(x) \dif x + C = F(t) - F(a) + C.
		\end{align*}
	}

\end{document}