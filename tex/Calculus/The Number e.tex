\documentclass[MathsNotesBase.tex]{subfiles}

\date{\vspace{-6ex}}


\begin{document}

\searchableSection{The Number e}{analysis}{
	\bigskip\bigskip
	\boxeddefinition{The \textbf{natural logarithm} is defined as,
		\[ \ln{x} = \int_1^x \frac{1}{t} \dif t. \]
	}
	\boxeddefinition{The number \textbf{e} can be defined in various ways. Some of the most common of these are:
		\begin{enumerate}[label=(\roman*)]
			\item{\[ e = \lim_{n \to \infty} \left(1 + \frac{1}{n}\right)^n \]}
			\item{\[ e = \sum_{n=0}^\infty \frac{1}{n!} = \frac{1}{0!} + \frac{1}{1!} + \frac{1}{2!} + \cdots \]}
			\item{e is the unique number such that, \[ \ln{e} = 1 \]}
		\end{enumerate}
	}

	\bigskip\bigskip
	\labeledProposition{The definition of the natural log implies:
		\begin{enumerate}[label=(\roman*)]
			\item{${ \ln{1} = 0 }$;}
			\item{${ \frac{\dif}{\dif x}\ln{x} = 1/x }$;}
			\item{${ \ln{x} = \ln{y} \implies x = y }$.}
		\end{enumerate}}{properties-of-natural-log}
	\begin{proof} The proofs of each of the properties are the following.
		\begin{enumerate}[label=(\roman*)]
			\item{${\bm{ \ln{1} = 0 : }}$\\\\
				By the properties of integrals,
				\[ \ln{1} = \int_1^1 \frac{1}{t} \dif t = 0. \]
			}
			\item{${\bm{ \frac{\dif}{\dif x}\ln{x} = 1/x : }}$\\\\
				This is a consequence of the Fundamental Theorem of Calculus and,
				\[ \ln{x} = \int_1^x \frac{1}{t} \dif t. \]
			}
			\item{${\bm{ \ln{x} = \ln{y} \implies x = y : }}$\\\\
				This is a consequence of the previous property that,
				\[ \frac{\dif}{\dif x}\ln{x} = \frac{1}{x}. \]
				For ${ x > 0 }$, the function ${ 1/x }$ is strictly positive. This, in turn, means that the function,
				\[ f(x) = \int_1^x \frac{1}{t} \dif t \]
				is strictly monotonically increasing for all ${ x > 0 }$. Therefore,
				\[ x > y \implies f(x) > f(y) \]
				and so,
				\[ x \neq y \implies f(x) \neq f(y). \qedhere\]
			}
		\end{enumerate}
	\end{proof}

	\bigskip\bigskip
	\labeledProposition{Let ${ f: \R{} \longmapsto \R{} }$ be defined as
	\[ f(x) = e^x. \] Then the function $f$ is the inverse of the natural log function $\ln$.}{exp-x-is-inverse-of-ln}
	\begin{proof}
		Using the definition of the natural log and properties of integrals we have,
		\begin{align*}
		\ln{e^x} &= \int_1^{e^x} \frac{1}{t} \dif t \\[8pt]
		  		 &= \int_1^{e} \frac{1}{t} \dif t + \int_e^{e^2} \frac{1}{t} \dif t + \cdots + \int_{e^{x-1}}^{e^x} \frac{1}{t} \dif t &\sidecomment{} \\[8pt]
		  		 &= \sum_{i=1}^x \int_{e^{i-1}}^{e^i} \frac{1}{t} \dif t \\[8pt]
		  		 &= \sum_{i=1}^x \int_{1}^e \frac{e^{i-1}}{t} \dif u = \sum_{i=1}^x \int_{1}^e \frac{1}{u} \dif u &\sidecomment{${ u = t/(e^{i-1}),\, e^{i-1}\dif u = \dif t }$}\\[8pt]
		  		 &= \sum_{i=1}^x \ln{e} = \sum_{i=1}^x 1 = x. \\[8pt]
		\end{align*}
		Conversely, using similar logic,
		\begin{align*}
		&& \ln{e^{\ln{x}}} &= \int_1^{e^{\ln{x}}} \frac{1}{t} \dif t \\[8pt]
		&&  &= \sum_{i=1}^{\ln{x}} \ln{e} = \sum_{i=1}^{\ln{x}} 1 = \ln{x}. &\sidecomment{} \\
		\end{align*}
		Then, by the injectivity of the natural log (property (iii) of \autoref{prop:properties-of-natural-log}),
		\[ \ln{e^{\ln{x}}} = \ln{x} \implies e^{\ln{x}} = x. \]
		So, we have shown that,
		\[ \ln{e^x} = x = e^{\ln{x}} \]
		which implies that the functions are inverses.
	\end{proof}

	
	\bigskip\bigskip
	\labeledProposition{For any ${ x,r \in \R{} }$, \[ \ln{x^r} = r\ln{x}. \]}{log-x-to-power-r-is-r-times-log-x}
	\begin{proof}
		By \autoref{prop:exp-x-is-inverse-of-ln}, the functions ${ e^x }$ and ${ \ln{x} }$ are inverses. Therefore,
		\[ \ln{x^r} = \ln{(e^{\ln{x}})^r} = \ln{e^{r\ln{x}}} = r\ln{x}. \qedhere \]
	\end{proof}
	

	\bigskip\bigskip
	\labeledProposition{Definitions (i) and (ii) of the number e are equivalent. That's to say,
	\[ e = \lim_{n \to \infty} \left(1 + \frac{1}{n}\right)^n \iff e = \sum_{n=0}^\infty \frac{1}{n!} = \frac{1}{0!} + \frac{1}{1!} + \frac{1}{2!} + \cdots \]
	}{limit-def-of-e-equiv-to-inf-series-def-of-e}
	\begin{proof}
		Consider, for finite $n$, the binomial expansion,
		\begin{align*}
		\left(1 + \frac{1}{n}\right)^n &= \sum_{k=0}^n {{n}\choose{k}} \frac{1}{n^k} \\[8pt]
		&= \frac{1}{0!} + \frac{n}{1!}\left(\frac{1}{n}\right) + \frac{(n)(n-1)}{2!}\left(\frac{1}{n^2}\right) + \frac{(n)(n-1)(n-2)}{3!}\left(\frac{1}{n^3}\right) + \\[8pt]
		&\hspace{28pt}\cdots + \frac{(n)(n-1)\cdots(1)}{n!}\left(\frac{1}{n^n}\right) \\[8pt]
		&= \frac{1}{0!} + \frac{1}{1!} + \frac{1}{2!}\left(\frac{n-1}{n}\right) + \frac{1}{3!}\left(\frac{(n-1)(n-2)}{n^2}\right) + \\[8pt]
		&\hspace{28pt}\cdots + \frac{1}{n!}\left(\frac{(n-1)(n-2)\cdots(1)}{n^{n-1}}\right) \\[8pt]
		&= \frac{1}{0!} + \frac{1}{1!} + \frac{1}{2!}\left(1 - \frac{1}{n}\right) + \frac{1}{3!}\left(1 - \frac{1}{n}\right)\left(1 - \frac{2}{n}\right) + \\[8pt]
		&\hspace{28pt}\cdots + \frac{1}{n!}\left(1 - \frac{1}{n}\right)\left(1 - \frac{2}{n}\right)\cdots\left(1 - \frac{n-1}{n}\right)
		\end{align*}
		which tends, as ${ n \to \infty }$, to
		\[ \frac{1}{0!} + \frac{1}{1!} + \frac{1}{2!} + \frac{1}{3!} + \cdots \qedhere\]
	\end{proof}
	\note{The actual proof of this requires limit inferior and superior (see \href{https://en.m.wikipedia.org/wiki/Limit_superior_and_limit_inferior}{wikipedia}) because we haven't shown that the expression derived from the binomial expression converges. The full proof can be found in Artin[73] and \href{https://en.m.wikipedia.org/wiki/Characterizations_of_the_exponential_function\#Equivalence_of_the_characterizations}{wikipedia}.}

	\bigskip\bigskip
	\labeledProposition{Definitions (i) and (iii) of the number e are equivalent. That's to say,
		\[ e = \lim_{n \to \infty} \left(1 + \frac{1}{n}\right)^n \iff \text{$e$ is the unique number such that } \ln{e} = 1. \]
	}{limit-def-of-e-equiv-to-log-def-of-e}
	\begin{proof} Assume that,
		\[ e = \lim_{n \to \infty} \left(1 + \frac{1}{n}\right)^n. \]
		Then, taking logs of both sides,
		\begin{align*}
		&& \ln{e} &= \ln{\lim_{n \to \infty} \left(1 + \frac{1}{n}\right)^n} \\[8pt]
		&&  &= \lim_{n \to \infty} \ln{\left(1 + \frac{1}{n}\right)^n} &\sidecomment{by \autoref{theo:comps_continuous_functions_are_continuous}} \\[8pt]
		&&  &= \lim_{n \to \infty} \frac{\ln{\left(1 + \frac{1}{n}\right)}}{1/n} &\sidecomment{using \autoref{prop:log-x-to-power-r-is-r-times-log-x}} \\[8pt]
		&&  &= \lim_{n \to \infty} \frac{\ln{\left(1 + \frac{1}{n}\right)} - \ln{1}}{1/n} &\sidecomment{} \\[8pt]
		&&  &= \lim_{h \to 0} \frac{\ln{\left(1 + h\right)} - \ln{1}}{h} &\sidecomment{} \\[8pt]
		&&  &= \eval{\frac{\dif(\ln{x})}{\dif x}}_{x=1} = \eval{\frac{1}{x}}_{x=1} = 1. &\sidecomment{}
		\end{align*}
		This shows that 
		\[ e = \lim_{n \to \infty} \left(1 + \frac{1}{n}\right)^n \implies \ln{e} = 1. \]
		This number is unique by property (iii) of the natural log in \autoref{prop:properties-of-natural-log}.\\\\
		
		Conversely, if we assume that e is the unique number such that ${ \ln{e} = 1 }$ then the fact already shown, that 
		\[ \ln{\lim_{n \to \infty} \left(1 + \frac{1}{n}\right)^n} = 1 \]
		implies that 
		\[ \lim_{n \to \infty} \left(1 + \frac{1}{n}\right)^n = e. \qedhere\]
	\end{proof}

	\bigskip\bigskip\bigskip
	\subsubsection{\texorpdfstring{$e^x$}{e to the power x}}\label{sssection:e-to-the-power-x}
	\bigskip
	Taking the limit definition of $e$ (definition (i)) and raising it to the power of $x$, by (vi) of \autoref{theo:func-limit-properties}, we have
	\[ e^x = \left[\lim_{n \to \infty} \left(1 + \frac{1}{n}\right)^n\right]^x = \lim_{n \to \infty} \left(1 + \frac{1}{n}\right)^{xn}. \]
	If we expand this out using binomial theorem we get,
	\begin{align*}
	\left(1 + \frac{1}{n}\right)^{xn} &= \sum_{k=0}^{xn} {{xn}\choose{k}} \frac{1}{n^k} \\[8pt]
	    &= 1 + (xn)\left(\frac{1}{n}\right) + \frac{(xn)(xn-1)}{2!}\left(\frac{1}{n^2}\right) + \frac{(xn)(xn-1)(xn-2)}{3!}\left(\frac{1}{n^3}\right) + \cdots &\sidecomment{} \\[8pt]
	    &= 1 + x + \frac{1}{2!}\left(\frac{x(xn-1)}{n}\right) + \frac{1}{3!}\left(\frac{x(xn-1)(xn-2)}{n^2}\right) + \cdots &\sidecomment{} \\[8pt]
	    &= 1 + x + \frac{x}{2!}\left(x - \frac{1}{n}\right) + \frac{x}{3!}\left(x - \frac{1}{n}\right)\left(x - \frac{2}{n}\right) + \cdots &\sidecomment{}
	\end{align*}
	If we take the limit of this expression as ${ n \to \infty }$ then we get,
	\[ e^x = \frac{1}{0!} + \frac{x}{1!} + \frac{x^2}{2!} + \frac{x^3}{3!} + \cdots \]
	
	But also, if we take the expression (which arises if we infinitely compound interest at an interest rate of $x$),
	\[ \left(1 + \frac{x}{n}\right)^n \]
	and again expand it using binomial theorem,
	\begin{align*}
	\left(1 + \frac{x}{n}\right)^n &= \sum_{k=0}^n {{n}\choose{k}} \frac{x^k}{n^k} \\[8pt]
	&= 1 + \frac{nx}{n} + \frac{n(n-1)}{2!}\frac{x^2}{n^2} + \frac{n(n-1)(n-2)}{3!}\frac{x^3}{n^3} + \cdots \\[8pt]
	&= 1 + x + \frac{x^2}{2!}\frac{n-1}{n} + \frac{x^3}{3!}\frac{(n-1)(n-2)}{n^2} + \cdots \\[8pt]
	&= 1 + x + \frac{x^2}{2!}\left(1 - \frac{1}{n}\right) + \frac{x^3}{3!}\left(1 - \frac{1}{n}\right)\left(1 - \frac{2}{n}\right) + \cdots
	\end{align*}
	again if we take the limit of this expression as ${ n \to \infty }$ then we get,
	\[ \frac{1}{0!} + \frac{x}{1!} + \frac{x^2}{2!} + \frac{x^3}{3!} + \cdots = e^x. \]
}

\end{document}