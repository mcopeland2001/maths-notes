\documentclass[../MathsNotesBase.tex]{subfiles}

\date{\vspace{-6ex}}

\begin{document}
	
	\pagebreak
	\searchableSection{Fields}{abstract algebra}
	\label{def:field}
	
	\searchableSubsection{Infinite Fields}{abstract algebra, fields}{
		
		\boxeddefinition{A \textbf{field} F is a set together with two laws of composition, addition and multiplication, satisfying the following axioms:
			\begin{enumerate}[label=(\roman*)]
				\item{Addition makes $F$ into an abelian group $F^+$ (or ${ (F,+) }$). Its identity element is denoted 0.}
				\item{Multiplication is associative and commutative and makes ${ (F \setminus \{0\}, \times) }$ into a group. Its identity element is denoted 1.}
				\item{Distributive law: For all ${ a,b,c \in F,\, (a + b)c = ac + bc }$.}
			\end{enumerate}
		}
		
		\smallskip
		\note{The distributive law establishes a relationship between the two laws of composition such that, for ${ a \in \F{} }$,
			\[ a + a = 1a + 1a = (1 + 1)a. \]
			In so doing, it establishes a relationship between the two groups: the additive group and the multiplicative group.
		}
	
		\bigskip
		\labeledTheorem{\textbf{(Difference of Cubes)} For ${ a,b \in \F{} }$,
			\[ a^3 - b^3 = (a - b)(a^2 + b^2 + ab). \]
		}{difference-of-cubes}
		\begin{proof}
			\[\begin{aligned}
				&(a - b)(a^2 + b^2 + ab) \\
				&= (a^3 + ab^2 + a^2b) - (ba^2 + b^3 + bab)  &\sidecomment{by distributive law}\\
				&= (a^3 + ab^2 + a^2b) - (a^2b + b^3 + ab^2)  &\sidecomment{by commutativity of multiplication}\\
				&= a^3 + ab^2 + a^2b + (-a^2b) + (-b^3) + (-ab^2)  &\sidecomment{by distributive law}\\
				&= (a^3 + (-b^3)) + (ab^2 + (-ab^2)) + (a^2b + (-a^2b))  &\sidecomment{by associativity of +}\\
				&= (a^3 + (-b^3)) + 0 + 0 = a^3 - b^3.  &\sidecomment{by additive inverse}\\
			\end{aligned}\]
		\end{proof}
		\note{Note that the above proof does not require the multiplicative inverse property of fields and so a field is sufficient but not necessary. For this reason, the above theorem holds also in the integers $\Z{}$ although it wouldn't be defined in the naturals $\N{}$ as, e.g. $(-b^3)$ would be undefined.}
	
		
		\biggerskip
		\subsubsection{Real-Number Exponentiation}\label{sssection:real-number-exponentiation}
		
		\subheading{Positive exponents}
		\bigskip
		Let ${ a \in \F{} }$. Then
		\[\begin{aligned}
			2a &= a + a, \nn
			3a &= a + a + a, \nn
			a^2 = aa &= \underbrace{a + a + \cdots + a}_{a \text{ times}}, \nn
			a^3 = aaa &= \underbrace{a + a + \cdots + a}_{a \text{ times}} \, \underbrace{a + a + \cdots + a}_{a \text{ times}}.
		\end{aligned}\]
		So, exponentiation by a positive integer can be defined in any field using the properties of the additive group.\\
		
		Let ${ a \in \Q{} }$. Then
		\[\begin{aligned}
			\frac{7}{2} a &= \underbrace{a + a + a}_{\floor{\frac{7}{2}} = 3 \text{ times}} + \frac{1}{2} a, \nn
			q &= a^{\frac{1}{2}} \in \Q{}, \nn
			q^2 = qq &= \underbrace{q + q + \cdots + q}_{q \text{ times}}, \nn
			\exists m,n \in \N{} \suchthat q &= \frac{m}{n} \implies \nn
			qq = \frac{m}{n}\frac{m}{n} &= \underbrace{\frac{m}{n} + \frac{m}{n} + \cdots + \frac{m}{n}}_{\floor{\frac{m}{n}} \text{ times}} + \left(\frac{m \bmod n}{n}\right) \frac{m}{n}.
		\end{aligned}\]
		So, exponentiation by a rational number can be defined in $\Q{}$.\\
		
		Therefore, since the reals can be expressed as summations of series of rationals, exponentiation by any real number can be defined in $\R{}$.
		
		\biggerskip
		\subsubsection{Subfields}
		\boxeddefinition{
			A field $F$ is a \textbf{subfield of ${\bm{ \C{} }}$} if the following properties hold:
			\begin{itemize}
				\item If $a, b \in F$, then $a + b \in F$.
				\item If $a \in F$, then $-a \in F$.
				\item If $a, b \in F$, then $ab \in F$.
				\item If $a \in F \text{ and } a \ne 0$, then $a^{-1} \in F$.
				\item $1 \in F$.
			\end{itemize}
		}
		
		\smallskip
		\note{Note that using the first, second and last of these axioms we can deduce that $1 - 1 = 0$ is an element of $F$.\\\\
			Also notice that addition on the field makes ${ (F, +) }$ into an abelian group and multiplication makes ${ (F \setminus \{0\}, \times) }$ into an abelian group also. Conversely, any subset of $F$ for which this is also true is a \textbf{subfield}.
		}
	
		\biggerskip
		\boxeddefinition{\textbf{(Ordered Field)} A field ${ (\F{}, +, \times) }$ together with a total order $\leq$ on $\F{}$ is an \textbf{ordered field} if the order satisfies the following properties.\\
			${ \forall a,b,c \in \F{} }$,
			\begin{enumerate}[label=(\roman*)]
				\item{${ a \leq b \implies a + c \leq b + c }$,}
				\item{${ (0 \leq a) \land (0 \leq b) \implies 0 \leq a \times b  }$.}
			\end{enumerate}
		}\label{defn:ordered-field}
	}

	\bigskip
	\searchableSubsection{The Complex Field}{abstract algebra, complex numbers}{\bigskip
		\labeledProposition{For every $\alpha \in \C{}$, there exists a unique $\beta \in \C{}$ such that $\alpha + \beta = 0$.}{unique_complex_additive_inverse}
		\begin{proof}
			By contradiction: Say there are two such elements, $\beta, \gamma$ such that,
			\begin{align*}
				\alpha + \beta &= 0 = \alpha + \gamma \\
				(\alpha + \beta) + \beta &= (\alpha + \beta) + \gamma \\
				0 + \beta &= \beta = 0 + \gamma = \gamma \qedhere\\
			\end{align*}
		\end{proof}
		
		\labeledProposition{For every $\alpha \in \C{}$ with $\alpha \neq 0$, there exists a unique $\beta \in \C{}$ such that $\alpha\beta = 1$.}{unique_complex_multiplicative_inverse}
		\begin{proof}
			By contradiction: Say there are two such elements, $\beta, \gamma$ then,
			\begin{align*}
				\alpha\beta &= 1 = \alpha\gamma \\
				\beta &= \frac{1}{\alpha} = \gamma \qedhere\\
			\end{align*}
		\end{proof}
	
		\bigskip
		\labeledProposition{The complex field $\C{}$ is not an ordered field.}{complex-field-is-not-ordered}
		\begin{proof}
			By definition (\ref{defn:ordered-field}) an ordered field must satisfy the properties:
			${ \forall a,b,c \in \F{} }$,
			\begin{enumerate}[label=(\roman*)]
				\item{${ a \leq b \implies a + c \leq b + c }$,}
				\item{${ (0 \leq a) \land (0 \leq b) \implies 0 \leq a \times b  }$.}
			\end{enumerate}
			The second of these implies that if ${ 0 \leq i }$ then,
			\[\begin{aligned}
				&& 0 &\leq i \times i = i^2  \\
				&\iff & 0 &\leq -1.
			\end{aligned}\]
			Conversely, if ${ 0 \leq -i }$ then,
			\[\begin{aligned}
				&& 0 &\leq -i \times -i = i^2  \\
				&\iff & 0 &\leq -1.
			\end{aligned}\]
			
			Since these are the only two possibilities for an ordering of 0 and i, any ordering of the two elements results in ${ 0 \leq -1 }$. This results in a contradiction because,
			\[\begin{aligned}
				&& 0 &\leq -1 \times -1 = 1  \\
				&\iff & 0 &\leq 1
			\end{aligned}\]
			and we cannot have both ${ 0 \leq 1 }$ and ${ 0 \leq -1 }$.
		\end{proof}
	
		\biggerskip
		\subsubsection{The Fundamental Theorem of Algebra}
		\begin{theorem}[Fundamental Theorem of Algebra]\label{theo:fundamental-theorem-of-algebra}
			Every non-constant single-variable polynomial with complex coefficients has at least one complex root.
		\end{theorem}
		\begin{proof}
			\href{https://en.wikipedia.org/wiki/Fundamental_theorem_of_algebra\#Proofs}{wikipedia}
		\end{proof}
		\begin{corollary}
			Every non-constant degree-$n$ single-variable polynomial with complex coefficients has exactly $n$ complex roots when counted with the multiplicity.
		\end{corollary}
		\begin{proof}
			This is equivalent to the Fundamental Theorem of Algebra because if we have a degree-$n$ polynomial with complex coefficients,
			\[ p_n(z) = a_n z^n + a_{n-1} z^{n-1} + \cdots + a_1 z + a_0 \]
			then the theorem tells us that this polynomial has at least one complex root. Let $r_n$ be a root of $p_n(z)$. Then,
			\[ p_n(z) = (z - r_n)p_{n-1}(z). \]
			But since $p_{n-1}(z)$ is a degree-$n-1$ polynomial with complex coefficients, the theorem also assures us that it too has at least one complex root, say $r_{n-1}$ so that
			\[ p_n(z) = (z - r_n)(z - r_{n-1})p_{n-2}(z). \]
			Clearly, we can proceed in this way until we obtain
			\[ p_n(z) = a(z - r_n)(z - r_{n-1})\cdots(z - r_1) \]
			for some constant $a$. Since the roots $r_i$ may not be distinct, this shows that $p_n(z)$ has $n$ roots when counted with the multiplicity.
		\end{proof}
	
		\biggerskip
		\subsubsection{Examples}
		\bigskip
		\begin{exe}
			\ex{Find all the roots of $x^3 = 1$ for $x \in \C{}$.\\
			Since $x^3 - 1 = (x - 1)(x^2 + x + 1)$, we have (via zero-factor theorem) possible roots from,
			\[ x - 1 = 0 \iff x = 1 \]
			\begin{align*}
				x^2 + x + 1 &= 0
				\implies x = \frac{-1 \pm \sqrt{-3}}{2} = \frac{-1 \pm \sqrt{3}\,i}{2}
			\end{align*}
			More generally,
			\[ (a+bi)+(a-bi)=2a \] 
			and since also,
			\[ \left[\frac{-1 + \sqrt{3}\,i}{2}\right]^2 = \frac{-1 - \sqrt{3}\,i}{2} \]
			as well as the reverse,
			\[ \left[\frac{-1 - \sqrt{3}\,i}{2}\right]^2 = \frac{-1 + \sqrt{3}\,i}{2} \]
			this means that if $x = \frac{-1 \pm \sqrt{3}\,i}{2}$ then $x^2 + x$ is of the form $(a+bi)+(a-bi) = 2a$ and so we have that $x^2 + x = -1 \iff x^2 + x + 1 = 0$.\\\\
			In addition,
			\[ (a+bi)(a-bi)=a^2+b^2 \]
			which means that if $x = \frac{-1 \pm \sqrt{3}\,i}{2}$ then $x^3 = x^2x$ is of the form $(a+bi)(a-bi)=a^2+b^2$ so we have
			that $x^3 = {\frac{-1}{2}}^2 + {\frac{\sqrt{3}}{2}}^2 = \frac{1}{4} + \frac{3}{4} = 1$.\\\\
			So we see that - allowing for complex $x$ - the cubic polynomial $x^3 - 1$ has 3 roots as we should expect from the \href{https://en.wikipedia.org/wiki/Fundamental_theorem_of_algebra}{Fundamental Theorem of Algebra}.
			}
			\bigskip
			\ex{Consider the roots of the complex polynomial ${ x^3 = -1 }$.\\
				
				If we use the exponential form of the complex number $x$ then we have,
				\[ (re^{i\theta})^3 = r^3 e^{3i\theta} = -1. \]
				So, if we let ${ r = 1 }$, we are looking for an angle $\theta$ (in radians) such that,
				\[ \cos(3\theta) + i \sin(3\theta) = -1. \]
				Since, for ${ z = -1 = -1 + 0i }$, the imaginary part is 0, we know that ${ i \sin(3\theta) = 0 }$ so,
				\[ \sin(3\theta) = 0 \implies 3\theta = n\pi \eqword{for} n \in \Z{}. \]
				Any ${ n \in \Z{} }$ will generate an angle that satisfies the equation but, from the definition of the complex exponential (ref: \ref{def:complex-exponential}), we know that the complex numbers represented by exponentials in this way, cycle every $2\pi$-length interval of $\theta$. Since we are only interested in finding all the unique solutions (i.e. the unique complex numbers that are roots of the polynomial), we can restrict the search to a single $2\pi$-length interval of $\theta$. Common practices are to use either ${ [0, 2\pi) }$ or ${ (-\pi, \pi] }$.\\
				
				Using the interval ${ \theta \in [0, 2\pi) }$, the valid values of the parameter $\theta$ are 
				\[ \theta \in \left\{ 0, \frac{\pi}{3}, \frac{2\pi}{3}, \pi, \frac{4\pi}{3}, \frac{5\pi}{3} \right\}. \]
				
				But we also need that ${ \cos(3\theta) = -1 }$ so,
				\[ \cos(3\theta) = -1 \implies 3\theta = \pi + 2n\pi \eqword{for} n \in \Z{} \]
				which gives the following valid values for $\theta$,
				\[ \theta \in \left\{ \frac{\pi}{3}, \pi, \frac{5\pi}{3} \right\}. \]
				
				The solutions are generated from the intersection of the two sets of valid values for $\theta$. So, the solutions are
				\[ x \in \{ e^{\frac{i\pi}{3}}, e^{i\pi}, e^{\frac{5i\pi}{3}} \}. \]
				
				A quicker way to have generated the solutions in this case would be to use Euler's Identity (\href{https://en.wikipedia.org/wiki/Euler\%27s_identity}{wikipedia}) to say,
				\[ x^3 = -1 \implies x^3 = e^{i\pi}  \cdot e^{2ni\pi} \eqword{for} n \in \Z{} \]
				which implies that
				\[ x = e^{\frac{i\pi}{3}} \cdot e^{\frac{2ni\pi}{3}} = e^{\frac{i\pi(1 + 2n)}{3}} \eqword{for} n \in \Z{}  \]
				where, if we define a homomorphism between multiplicative groups from angles $\theta$ to the unit circle of complex numbers of modulus 1,
				\[ \phi(\theta) = \cos\theta + i \sin\theta \]
				then the kernel of $\phi$ is
				\[ 2n\pi \eqword{for} n \in \Z{}. \]
			}
			\biggerskip
			\ex{Find the roots of the complex polynomial ${ z^6 = -1 }$.\\
				\[\begin{aligned}
					&& x^6 &= -1 \\
					&\iff & x^6 &= e^{i\pi} &\sidecomment{Euler's equation} \\
					&\iff & x &= e^{\frac{i\pi}{6}} \cdot e^{\frac{2n\pi}{6}} \eqword{for} n \in \Z{} &\sidecomment{} \\
					&\iff & x &\in \{ e^{\frac{i\pi}{6}}, e^{\frac{3i\pi}{6}}, e^{\frac{5i\pi}{6}}, e^{\frac{7i\pi}{6}}, e^{\frac{9i\pi}{6}}, e^{\frac{11i\pi}{6}} \} \\
					&\iff & x &\in \{ e^{\frac{i\pi}{6}}, e^{\frac{i\pi}{2}}, e^{\frac{5i\pi}{6}}, e^{\frac{7i\pi}{6}}, e^{\frac{3i\pi}{2}}, e^{\frac{11i\pi}{6}} \}.
				\end{aligned}\]
				Now, observing that
				\[ e^{\frac{11i\pi}{6}} = e^{\frac{-i\pi}{6}} \eqand e^{\frac{3i\pi}{2}} = e^{\frac{-i\pi}{2}} \eqand e^{\frac{7i\pi}{6}} =  e^{\frac{-5i\pi}{6}}, \]
				we see that, by the properties of the complex conjugate (\autoref{prop:complex-conjugate-properties}), these pairs are conjugates. So, if we define
				\[ z_1 = e^{\frac{i\pi}{6}}, \; z_2 = e^{\frac{i\pi}{2}}, \; z_3 = e^{\frac{5i\pi}{6}}, \]
				then we have factorized the polynomial into linear factors,
				\[\begin{aligned}
					&& x^6 + 1 &= 0 \\
					&\iff & (x - z_1)(x - \conj{z_1})(x - z_2)(x - \conj{z_2})(x - z_3)(x - \conj{z_3}) &= 0.
				\end{aligned}\]
			}
		\end{exe}
	}
	
	\biggerskip
	\searchableSubsection{Finite Fields}{abstract algebra, fields}{
		\TODO{Finite Fields (Artin[98])}
	}

\end{document}