\documentclass[12pt]{report}

\usepackage{subfiles}
\usepackage[utf8]{inputenc}
\usepackage[pdftex]{hyperref}
\usepackage[depth=3]{bookmark}


\usepackage{amsmath}
\usepackage{amssymb}
\usepackage{centernot}
\usepackage{listings}
\usepackage[pdftex]{graphicx}
\usepackage{color}
\usepackage{bm}
\usepackage{amsthm}
\usepackage{mathtools}
\usepackage{enumitem}
%\usepackage{esvect} provides longer over arrows (not currently in use)

\usepackage[T1]{fontenc}
\usepackage{lmodern}
\usepackage{amsfonts}
\usepackage{epstopdf}
\usepackage[table]{xcolor}
\usepackage{siunitx}
\usepackage{booktabs}
\usepackage{multirow}

%----- fix the naming clash of \exp between gb4e and math as described in http://https://tex.stackexchange.com/questions/528818/crash-between-math-exp-and-that-of-gb4e
\let\mathexp=\exp % save the current (math) definition of \exp
\usepackage{gb4e}
\let\gbexp=\exp % save the current (gb4e) definition of \exp

\DeclareRobustCommand{\exp}{\ifmmode\mathexp\else\expandafter\gbexp\fi}
%----- end fix


\usepackage{titlesec}
% ---- titlesec fix 
\usepackage{etoolbox}

\makeatletter
\patchcmd{\ttlh@hang}{\parindent\z@}{\parindent\z@\leavevmode}{}{}
\patchcmd{\ttlh@hang}{\noindent}{}{}{}
\makeatother
% ---- 
\noautomath
\usepackage[most]{tcolorbox}
\usepackage{csquotes}
\usepackage{wrapfig}
\usepackage[none]{hyphenat}
\usepackage{commath}
\usepackage{optidef} % for optimisation problems

\usepackage{subfig}
\usepackage{microtype} % Load this package to obtain a fine composition. 

\allowdisplaybreaks


\def\buildDir{../build}
\def\resourceDir{../resources}
\def\texBaseDir{.}
\def\foundationsDir{\texBaseDir/Foundations}
\def\algebraDir{\texBaseDir/Algebra}
\def\analysisDir{\texBaseDir/Analysis}
\def\calculusDir{\texBaseDir/Calculus}
\def\ProblemsDir{\texBaseDir/Problems}


\definecolor{note_bg}{RGB}{247,247,247}
\AtBeginEnvironment{displayquote}{\small}
\setcounter{secnumdepth}{3}
\setcounter{tocdepth}{4}


\titleformat*{\subparagraph}{\small\bfseries}{}{}{}
%\titleformat{\subsubsection}[hang]{\sffamily\bfseries}{\thesection}{1em}{}  % bug -> breaks numbering
\urlstyle{same}


% default formatting for code listings
\lstset{ % docs: https://en.wikibooks.org/wiki/LaTeX/Source_Code_Listings
	backgroundcolor=\color{white},   	% choose the background color; you must add \usepackage{color} or \usepackage{xcolor}; should come as last argument
	basicstyle={\small\ttfamily},		% the size of the fonts that are used for the code
	frame=tb,	                  		% adds a frame around the code
	tabsize=2	                   		% sets default tabsize to 2 spaces
}



%\documentclass{report}
% ----- search code ------
\usepackage{ifthen}
\usepackage{xifthen}
\usepackage{tikz}
\usepackage{xstring}


% copied from: https://ipfs-sec.stackexchange.cloudflare-ipfs.com/tex/A/question/161590.html
\newcommand*{\isinxp}[2]{\expandafter\isinxpp\expandafter{#1}{#2}}
\newcommand*{\isinxpp}[2]{\isin {#1}{#2}}


% copied from: https://tex.stackexchange.com/questions/122634/variable-persist-outside-environment
\makeatletter
\def\newglobalboolean#1{%
  \expandafter\@ifdefinable\csname if#1\endcsname{%
    \expandafter\let\csname if#1\endcsname\iffalse
    \expandafter\def\csname #1true\endcsname{%
      \global\expandafter\let\csname if#1\endcsname\iftrue
    }%
    \expandafter\def\csname #1false\endcsname{%
      \global\expandafter\let\csname if#1\endcsname\iffalse
    }%
}}
\makeatother



\newglobalboolean{isSearchResult}

\newcommand\processKeywords[3]{		
	\setboolean{isSearchResult}{false}
	\foreach \word in {#1} {
		\ifthenelse{ \isinxp{\word}{#3} }
			{ \setboolean{isSearchResult}{true} }
	}
	\ifthenelse{ \boolean{isSearchResult} }
		{#2}
}
\newcommand\search[3]{
	\processKeywords{#1}{#2}{#3}
}

\newcommand\nosearch[2]{#2}
\providecommand\searchResult\nosearch

\newcommand{\tags}[1]{
	%{ \tiny{(tags: #1)} }
}
\newcommand{\searchableSubsection}[3]{
	\searchResult{#2}{
		\subsection{\subsectionTitle{#1}}\tags{#2}
			{#3}
	}
}
\newcommand{\searchableSection}[2]{
	\searchResult{#2}{
		\section{\sectionTitle{#1}}\tags{#2}\bigskip\bigskip
	}
}

\newcommand{\searchableChapter}[2]{
	\searchResult{#2}{
		\chapter{#1}\tags{#2}
	}
}

\newcommand\relay[2]{%
  \expandafter\csname #1\endcsname*{#2}%
}

\newcommand{\searchable}[4]{
	\searchResult{#3}{
		\expandafter\csname #1\endcsname*{#2 \tiny{(tags: #3)}}%\tags{#3}
			{#4}
	}
}

%\renewcommand\searchResult[2]{\search{#1}{#2}{subsection 2 tag}}

% -------- test code --------------
%\iffalse 
%\begin{document}
%\searchableSection{Section 1}{subsection 1 tag, subsection 2 tag}{

%	\searchableSubsection{\huge{Section 1}}{subsection 1 tag, subsection 2 tag}{}
%	
%	\searchableSubsection{Subsection 1}{subsection 1 tag}{
%		blah blah blah, blah blah blah
%	}
%	
%	\searchableSubsection{Subsection 2}{subsection 2 tag}{
%		blah blah blah, blah blah blah
%	}
	
%}
%\end{document}
%\fi

%\iffalse % blows up with 'no line to end here' error because of // at the beginning of subsection content
%\begin{document}
%\searchable{section}{Section Title}{some tags}{
%	\searchable{subsection}{Subsection Title}{some subsection tags}{//
%		blah blah blah, blah blah blah
%	}
%} 
%\end{document}
%\fi


\setlength{\parindent}{0em}
\newcommand\chapterTitle\Huge
\newcommand\sectionTitle\Huge
\newcommand\subsectionTitle\Large
\newcommand{\biggerskip}{\bigskip\bigskip}
\newcommand\subsubsubsection[1]{\textbf{\textit{#1}}\nopagebreak\\[8pt]\nopagebreak}% this is the level below the last toc/bookmark level
\newcommand\problem[1]{\paragraph{#1}}
\newcommand\sidecomment[1]{ \text{\scriptsize{#1}} }
\newcommand{\divides}{\mid}
\newcommand{\notdivides}{\nmid}
\newcommand{\eqand}{\hspace{10pt}\text{and}\hspace{10pt}}
\newcommand{\eqword}[1]{\hspace{10pt}\text{#1}\hspace{10pt}}
\newcommand{\modulo}[1]{\;(\bmod\; #1)}
\newcommand{\uminus}{\scalebox{0.5}[1.0]{$-$}}  % unary minus scaled by argument
\newcommand{\nn}{\\[6pt]}
\newcommand{\nnn}{\\[8pt]}
\newcommand{\sep}{\biggerskip\hrule\biggerskip}
\newcommand{\subheading}[1]{\bigskip\par\textbf{#1}\nopagebreak\\[4pt]\nopagebreak}


\newcommand{\wrong}{\begingroup\color{red}\text{Wrong!}\endgroup}
\newcommand{\TODO}[1]{\textcolor{blue}{\small{\underline{TODO:} #1}}}
\newcommand{\addref}{\TODO{add reference}}
\newcommand{\references}[1]{\small \textit{references:} #1}

\DeclarePairedDelimiterX\setc[2]{\{}{\}}{\,#1 \,\;\delimsize\vert\;\, #2\,}
\DeclareRobustCommand{\N}[1]{\mathbb{N}^{#1}}
\DeclareRobustCommand{\R}[1]{\mathbb{R}^{#1}}
\DeclareRobustCommand{\Z}[1]{\mathbb{Z}^{#1}}
\DeclareRobustCommand{\C}[1]{\mathbb{C}^{#1}}
\DeclareRobustCommand{\Q}[1]{\mathbb{Q}^{#1}}
\DeclareRobustCommand{\F}[1]{\mathbb{F}^{#1}}
\DeclareRobustCommand{\inv}[1]{{#1}^{-1}}
\DeclareRobustCommand{\logicsep}{\;.\;}


\newenvironment{amatrix}[1]{% from https://gitlab.com/jim.hefferon/linear-algebra/-/blob/master/src/sty/linalgjh.sty
	\left[\begin{array}{@{}*{#1}{c}|c@{}}
	}{%
	\end{array}\right]
}


\def\u{\vec{\bm{u}}}
\def\v{\vec{\bm{v}}}
\def\w{\vec{\bm{w}}}
\def\0{\vec{\bm{0}}}
\def\x{\vec{\bm{x}}}
\def\b{\vec{\bm{b}}}
\newcommand{\e}[1]{\vec{\bm{e_#1}}}

\newcommand{\bigsetc}[2]{\left\{#1\; \middle\vert \;#2\right\}}
\renewcommand{\abs}[1]{\left\lvert#1\right\rvert}
\newcommand{\cardinality}{\abs}
%\renewcommand{\norm}{\abs}
\newcommand{\dotprod}{\boldsymbol{\cdot}}
\newcommand{\congruent}[2]{\equiv #1 \modulo{#2}}
\newcommand{\conj}[1]{\overline{#1}}

\newcommand*{\V}[1]{\vec{\bm{#1}}}
\newcommand{\dotsuchthat}{\;\cdot\;}
\newcommand{\suchthat}{\text{ s.t. }}
\newcommand{\wrt}{w.r.t.\;}
\newcommand{\WLOG}{w.l.o.g.\;}
\newcommand{\notation}[1]{\noindent\fbox{%
		\parbox{\textwidth}{%
		        \textbf{Notation.} #1
		    }%
		}\bigskip
}
\newcommand{\nomenclature}[1]{\noindent\fbox{%
		\parbox{\textwidth}{%
			\textbf{Terminology.} #1
		}%
	}\bigskip
}

\newtheorem{theorem}{Theorem}[section]
\newtheorem{proposition}{Proposition}[section]
\newtheorem*{definition}{Definition}
\newtheorem{axiom}{Axiom}[section]
\newtheorem{corollary}{Corollary}[section]
\newtheorem{lemma}{Lemma}[section]
\newtheorem{claim}{Claim}[section]

\newcommand{\labeledProposition}[2]{
	\begin{proposition}
	\expandafter\label{prop:#2}
	{#1}
	\end{proposition}
}
\def\propositionautorefname{Proposition}

\newcommand{\labeledTheorem}[2]{
	\begin{theorem}
		\expandafter\label{theo:#2}
		{#1}
	\end{theorem}
}
\def\theoremautorefname{Theorem}
\def\corollaryautorefname{Corollary}
\def\lemmaautorefname{Lemma}

% \def\exautorefname{Example} doesn't work - don't know why

\newcommand{\boxeddefinition}[1]{\noindent\fbox{%
		\parbox{\textwidth}{%
			\begin{definition}
				#1
			\end{definition}	
		}%
	}\bigskip
}

\newcommand{\framed}[1]{\noindent\fbox{%
		\parbox{\textwidth}{%
			#1	
		}%
	}\bigskip
}

\newcommand{\boxedaxiom}[1]{\framed{
			\begin{axiom}
				#1
			\end{axiom}	
		}
}




\newcommand{\note}[1]{
	\begin{tcolorbox}[breakable,enhanced jigsaw,colback=note_bg,boxrule=0pt,arc=0pt]
	\noindent{%
			{\small \textit{#1}}	
		}
	\end{tcolorbox}
	\smallskip
}


\title{Mathematics Notes}
\date{\vspace{-6ex}}



\begin{document}

\end{document}