\documentclass[../MathsNotesBase.tex]{subfiles}

\date{\vspace{-6ex}}

\begin{document}

	\searchableSubsection{Functions}{Discrete Maths}{
		\biggerskip
		\boxeddefinition{\textbf{(Function)} A \textit{function} or \textit{mapping} associates every element in its \textit{domain} set with a single element in its \textit{codomain} set.\\
			
		Informally, a function sets up a relation between sets that is either many-to-1 or 1-to-1 but cannot be 1-to-many or many-to-many.
		\label{def:function}}
		
		\notation{If $f$ is a function with domain $X$ and codomain $Y$, then we write:
			\[ f: X \longmapsto Y. \]
		}
	
		\boxeddefinition{\textbf{(Function Image)} Let ${ f: X \longmapsto Y }$. Then the set ${ I \subseteq Y }$ such that,
			\[ I = \setc{y \in Y}{\exists x \in X \logicsep f(x) = y} \]
			is known as the \textit{image} of $Y$.
		}
	
		\notation{There are various notations for the image of a function, depending on context, but a common notation when dealing with general functions is
			\[ f(X) = \setc{y \in Y}{\exists x \in X \logicsep f(x) = y}. \]
			This notation allows for a natural notation for the image of subsets of the domain: If ${ Z \subset X }$ then the image of $Z$ under $f$ can be written $f(Z)$.
		}
	
		\boxeddefinition{\textbf{(Identity Function)} The \textit{identity function} associates every element in its \textit{domain} set with the same element. As a result its codomain is necessarily equal to its domain.\\
		The identity function from the set $X$ to itself is often denoted $id_X$.
		}
	
		\boxeddefinition{\textbf{(Function Inverse)} If $f$ is a function ${ f: X \longmapsto Y }$ then $g$ is an inverse function of $f$ iff $g$ is a function ${ Y \longmapsto X }$ such that
			\[ f g = id_Y \eqand g f = id_X \]
			which is to say,
			\[ (f g)(y) = y \eqand (g f)(x) = x. \]
		\label{def:function-inverse}}
	
		\boxeddefinition{\textbf{(Function Left/Right Inverse)} \TODO{needs review!} If $f$ is a function ${ f: X \longmapsto Y }$ and ${ I \subseteq Y }$ is the image of $f$, then $g$ is a \textit{left inverse} function of $f$ iff $g$ is a function ${ I \longmapsto X }$ such that
			\[ f g = id_I \eqand g f = id_X \]
			which is to say, for ${ i \in I }$ and for any ${ x \in X }$, 
			\[ (f g)(i) = i \eqand (g f)(x) = x. \]
			
			Similarly, $h$ is a \textit{right inverse} function of $f$ iff $h$ is a function ${ I \longmapsto X }$ such that
			\[ f g = id_Y \eqand g f = id_I \]
			which is to say, for ${ i \in I }$ and for any ${ y \in Y }$, 
			\[ (f g)(y) = y \eqand (g f)(i) = i. \]
		}
	
		\boxeddefinition{\textbf{(Function Pre-Image)} Let ${ f: X \longmapsto Y }$ and let ${ Z \subset Y }$. Then the set ${ P \subseteq X }$ such that,
			\[ P = \setc{x \in X}{\exists z \in Z \logicsep f(x) = z} \]
			is known as the \textit{pre-image} or \textit{inverse image} of $Z$ under $f$.
		}
		
		\notation{The most common notation for the pre-image of a set $Z$ is $\inv{f}(Z)$ but care must be taken not to confuse this notation with an inverse function; the pre-image is a set and, in fact, the function $f$ may be uninvertible.
		}
	
		\boxeddefinition{\textbf{(Surjection, Injection, Bijection)} A function ${ f: X \longmapsto Y }$ is \textit{surjective} iff
			\[ \forall y \in Y \logicsep \exists x \in X \logicsep f(x) = y; \]
			\textit{injective} iff
			\[ \forall x_1, x_2 \in X \logicsep f(x_1) = f(x_2) \implies x_1 = x_2; \]
			and \textit{bijective} iff both injective and surjective.
		}
		
		\notation{The set ${ S \subset \N{} }$ defined as
			\[ S = \setc{n \in \N{}}{n < m+1} \]
			has a standard notation $\N{}_m$.
		}
		\note{Contrast this with the notation $\F{m}$ (\ref{ssection:notation-S-dimensional-F-space}) which refers to the set of all functions ${ \N{}_m \longmapsto \F{} }$.}
		
		\boxeddefinition{\textbf{(Function Composition)} If $f$ and $g$ are two functions,
			\[ f: X \longmapsto Y \eqand g: Z \longmapsto X \]
			then the composition of $f$ and $g$, denoted ${ f\circ g }$ is the function defined as,
			\[ f \circ g : Z \longmapsto Y  \suchthat (f\circ g)(z) = f(g(z)). \]
			Function composition is associative:
			\[ f \circ g \circ h = (f \circ g) \circ h = f \circ (g \circ h). \]
		}
		\notation{The composition of functions $f$ and $g$ is also often denoted $fg$ although this can cause confusion with the pointwise product.}
		
		\biggerskip
		\labeledProposition{A bijection is a 1-to-1 mapping.}{bijection-is-1-to-1-mapping}
		\begin{proof}
			If a function ${ f: X \longmapsto Y }$ is bijective then it is injective and surjective. Surjectivity of the function means that every ${ y \in Y }$ is mapped to by at least one ${ x \in X }$ and injectivity means that every ${ y \in Y }$ is mapped to be at most one ${ x \in X }$.
		\end{proof}
		
		\bigskip
		\labeledProposition{The restriction of an injection to a subset of the domain is an injection.}{restriction-of-injection-is-injection}
		\begin{proof}
			Let ${ f: X \longmapsto Y }$ be an injection so that, for any ${ x_1, x_2 \in X }$, if ${ f(x_1) = f(x_2) }$ then ${ x_1 = x_2 }$. If we define the restriction
			\[ g: Z \subset X \longmapsto Y \suchthat g(z) = f(z), \]
			then we clearly also have, for any ${ z_1, z_2 \in Z }$,
			\[ g(z_1) = g(z_2) \iff f(z_1) = f(z_2) \iff z_1 = z_2. \] 
		\end{proof}
		\note{The same is not, in general, true of surjections.}
		
		\bigskip
		\labeledProposition{Composition preserves injectivity and surjectivity.}{composition-preserves-injectivity-and-surjectivity}
		\begin{proof}
			Let 
			\[ f: X \longmapsto Y \eqand g: Y \longmapsto Z. \]
			Then the composition ${ h = gf }$ is a function ${ h: X \longmapsto Z }$.\\
			
			Assume $f$ and $g$ are both injections. Then, for any ${ x_1, x_2 \in X }$,
			\[\begin{aligned}
				&& h(x_1) &= h(x_2) \\
				&\iff & g(f(x_1)) &= g(f(x_2)) &\sidecomment{by defn. of composition} \\
				&\iff & f(x_1) &= f(x_2) &\sidecomment{by injectivity of $g$} \\
				&\iff & x_1 &= x_2. &\sidecomment{by injectivity of $f$}
			\end{aligned}\]
			Therefore, the composition $h = gf$ is an injection.\\
			
			Assume $f$ and $g$ are both surjections. Then, by surjectivity of $g$, for any ${ z \in Z }$, there is some ${ y \in Y }$ such that ${ g(y) = z }$.
			Furthermore, by surjectivity of $f$, for any ${ y \in Y }$, there is some ${ x \in X }$ such that ${ f(x) = y }$. Therefore,
			\[ \forall z \in Z \logicsep \exists x \in X \logicsep g(f(x)) = z. \]
			Therefore, the composition $h = gf$ is a surjection.
		\end{proof}
		\begin{corollary}\label{coro:composition-of-bijections-is-bijection}
			A composition of bijections is a bijection.
		\end{corollary}
		\medskip
		\begin{corollary}
			The set of bijective functions forms a group (see: \ref{def:group}) with function composition.
		\end{corollary}
		\begin{proof}\nl
			\begin{itemize}
				\item Function composition is associative by definition.
				\item The composition of bijections is a bijection (by \autoref{coro:composition-of-bijections-is-bijection}) so function composition closes over the set of bijective functions.
				\item Bijective functions have inverses under function composition (by \autoref{prop:function-has-inverse-iff-bijective}).
			\end{itemize}
		\end{proof}
		
		\bigskip
		\labeledProposition{A function has a left inverse iff the function is injective.}{function-has-left-inverse-iff-injective}
		\TODO{needs review!}
		\begin{proof}
			Let ${ f: X \longmapsto Y }$ and let ${ I \subseteq Y }$ be the image of $f$.\\
			
			Assume $f$ is an injection. Then for any ${ x_1, x_2 \in X }$, if ${ x_1 \neq x_2 }$ then ${ f(x_1) \neq f(x_2) }$ so for each ${ i \in I }$ there exists a unique ${ x \in X }$ such that ${ f(x) = i }$. Therefore we can define a function,
			\[ g: I \longmapsto X \suchthat g(i) = x \]
			and we have, for all ${ i \in I }$,
			\[ f(g(i)) = i \implies fg = id_I \]
			and also, for all ${ x \in X }$,
			\[ g(f(x)) = x \implies gf = id_X. \]
			
			Conversely, assume $f$ has a left inverse ${ g: I \longmapsto X }$. Then we have, for all ${ x \in X }$,
			\[ g(f(x)) = x \implies gf = id_X. \]
			Suppose, for contradiction, that $f$ is not injective. Then there exist some ${ x_1, x_2 \in X }$ with ${ x_1 \neq x_2 }$ but ${ f(x_1) = f(x_2) }$. But this, together with ${ gf = id_X }$ means that ${ x_1 = g(f(x_1)) = g(f(x_2)) = x_2 }$ which is a contradiction.
		\end{proof}
	
		\bigskip
		\labeledProposition{A function has a right inverse iff the function is surjective.}{function-has-right-inverse-iff-surjective}
		\begin{proof}
			\TODO{needs review!}
		\end{proof}
		
		\bigskip
		\labeledProposition{A function has an inverse iff the function is bijective.}{function-has-inverse-iff-bijective}
		\begin{proof}
			Let ${ f: X \longmapsto Y }$.\\
			
			Suppose $f$ is a bijection. Then, by \autoref{prop:bijection-is-1-to-1-mapping}, $f$ maps one and only one $x$ to one and only one $y$. Therefore, we can define
			\[ g: Y \longmapsto X \suchthat g(y) = x \eqword{where} f(x) = y \]
			and then we have
			\[ \forall x \in X \logicsep g(f(x)) = x \eqand \forall y \in Y \logicsep f(g(y)) = y \]
			which is to say that $g$ is an inverse of $f$.
			
			\nl[4]
			Suppose $f$ has an inverse $g$. Then, by definition of the inverse (\ref{def:function-inverse}), we have
			\[ gf = id_X \iff \forall x \in X \logicsep g(f(x)) = x  \tag{1} \]
			and also
			\[ \forall y \in Y \logicsep fg = id_Y \iff f(g(y)) = y. \tag{2} \]
			Using (1) we can reason, for ${ x_1, x_2 \in X }$,
			\[\begin{aligned}
				&& f(x_1) &= f(x_2) \\
				&\iff & g(f(x_1)) &= g(f(x_2)) &\sidecomment{by defn. of function \ref{def:function}} \\
				&\iff & x_1 &= x_2 &\sidecomment{${ \because gf = id_X }$} \\
			\end{aligned}\]
			which is to say that $f$ is injective.\\
			Using (2) we can reason, for all ${ y \in Y }$,
			\[ \exists g(y) \in X \logicsep f(g(y)) = y \implies \exists x \in X \logicsep f(x) = y \]
			which is to say that $f$ is surjective.\\
			So $f$ is a bijection.
		\end{proof}
	
		\medskip
		\labeledProposition{The inverse of a function (if it exists) is unique.}{function-inverses-are-unique}
		\begin{proof}
			Let ${ f: X \longmapsto Y }$ and assume that $g$ and $h$ are inverses of $f$.\\
			
			The proposition can be proven in a number of equivalent ways. One way is to use only the associativity of function composition to reason,
			\[ g(y) = (hf)(g(y)) = ((hf)g))(y) = (h(fg))(y) = h((fg)(y)) = h(y). \]
			We also can use just the definition of a function to reason,
			\[ \forall x \in X \logicsep g(f(x)) = x = h(f(x)) \implies \forall y \in Y \logicsep g(y) = h(y). \]
			Alternatively, we can use the group properties of the group formed by function composition over the set of invertible functions to reason,
			\[ gf = id_X = hf \iff gfg = hfg \iff g \circ id_Y = h \circ id_Y \iff g = h. \]
		\end{proof}
		\note{Compare the proof of uniqueness of inverses in Group Theory \ref{sssection:corollaries-of-group-axioms}.}
		
		\medskip
		\labeledProposition{The inverse of a function is a bijection.}{function-inverses-are-bijective}
		\begin{proof}
			Using the definition of function inverses (\ref{def:function-inverse}) it's clear to see that if $g$ is an inverse of $f$ then $f$ is also an inverse of $g$. Therefore, by \autoref{prop:function-has-inverse-iff-bijective}, $g$ is bijective.
		\end{proof}
		
		
		
		\biggerskip
		\subsubsection{Cardinality}
		\bigskip
		\boxeddefinition{\textbf{(Set Cardinality)} If there exists a bijection between $\N{}_m$ and a set $X$ then we say that $X$ has \textit{cardinality} equal to $m$, denoted ${ \cardinality{X} = m }$. \label{def:set-cardinality}}
		
	
		\biggerskip
		\labeledTheorem{\textbf{(Pigeonhole Principle)} Let $m$ be a natural number. Then, for all ${ n \in \N{} }$, if there exists an injection ${ \N{}_n \longmapsto \N{}_m }$ then ${ n \leq m }$.}{pigeonhole-principle}
		\begin{proof}
			We will prove this by induction on the cardinality of the domain of the injection. The base case of ${ n = 1 }$ holds trivially as ${ m \in \N{} }$ and so, by \autoref{prop:mult_identity_least_element}, ${ 1 = n \leq m }$.\\
			
			For the induction step we assume that the proposition holds for some ${ n = k > 1 }$ so that, if there is an injection ${ \N{}_k \longmapsto \N{}_m }$, then ${ k \leq m }$.\\
			
			Now we consider the case ${ n = k+1 }$: Assume we have an injection ${ f: \N{}_{k+1} \longmapsto \N{}_m }$. Since ${ k+1 > k > 1 }$, there are at least two distinct elements in $\N{}_{k+1}$ --- which is to say ${ \N{}_2 \subseteq \N{}_{k+1} }$. Since $f$ is an injection, we have ${ f(1) \neq f(2) }$ and so there are at least two distinct elements in $\N{}_m$ and we deduce that ${ m > 1 }$.\\
			
			Since ${ m > 1 }$, let ${ s + 1 = m }$. Either there is some ${ a \in \N{}_k }$ such that ${ f(a) = m = s+1 }$ or there is not.
			\begin{itemize}
				\item If there is no ${ a \in \N{}_k }$ such that ${ f(a) = m = s+1 }$ then the restriction of $f$ to $\N{}_k$, by \autoref{prop:restriction-of-injection-is-injection}, is an injection and, by the induction hypothesis we can deduce that
				\[ k \leq s \iff k + 1 \leq s + 1 = m. \]
				
				\bigskip
				\item Conversely, if there is some ${ a \in \N{}_k }$ such that ${ f(a) = m = s+1 }$ then, because $f$ is an injection and so we must have ${ f(k+1) \neq f(a) }$, we can deduce that
				\[ f(k+1) < m = s+1 \iff f(k+1) \leq s. \]
				So we can define an injection from $\N{}_k$ to $\N{}_s$ using $f(k+1)$ as follows,
				\[	g: \N{}_k \longmapsto \N{}_s \suchthat g(x) = 	\begin{cases}
					f(k+1) & x = a \\
					f(x) & x \neq a.	
				\end{cases}
				\]
				The function $g$ is guaranteed to be injective because $f$ is injective and all we've done is change the mapping of ${ a \in \N{}_k }$ to return the element mapped to by ${ k+1 \centernot\in \N{}_k }$. Since $g$ is an injection, we can now employ the induction hypothesis to deduce that
				\[ k \leq s \iff k + 1 \leq s + 1 = m. \qedhere \]
			\end{itemize}
		\end{proof}
		
		
		\bigskip
		\labeledProposition{If a function ${ f: X \longmapsto Y }$ is injective then ${ \cardinality{X} \leq \cardinality{Y} }$.}{injection-from-X-to-Y-implies-cardinality-of-X-leq-to-that-of-Y}
		\begin{proof}
			Let ${ \cardinality{X} = m }$ and the ${ \cardinality{Y} = n }$. Then, by the definition of cardinality (\ref{def:set-cardinality}), there exist bijections
			\[ g_1: \N{}_m \longmapsto X \eqand g_2: \N{}_n \longmapsto Y. \]
			Since bijections are also injections, and inverses are also bijections (\autoref{prop:function-inverses-are-bijective}), $g_1$ and $\inv{g_2}$ are injections. Therefore, by \autoref{prop:composition-preserves-injectivity-and-surjectivity}, the composition ${ h = \inv{g_2} f g_1 }$ is an injection. Since $h$ is an injection ${ \N{}_m \longmapsto \N{}_n }$, by \autoref{theo:pigeonhole-principle}, we have ${ m \leq n }$ and so, by the definition of cardinality, ${ \cardinality{X} \leq \cardinality{Y} }$.
		\end{proof}
		
		\bigskip
		\labeledProposition{If a function ${ f: X \longmapsto Y }$ is surjective then ${ \cardinality{Y} \leq \cardinality{X} }$.}{surjection-from-X-to-Y-implies-cardinality-of-X-geq-to-that-of-Y}
		\begin{proof}
			Let ${ f: X \longmapsto Y }$ be a surjection so that, by definition, for every ${ y \in Y }$, there exists at least one ${ x \in X }$ such that ${ f(x) = y }$. Then ${ \cardinality{Y} \leq \cardinality{X} }$.\\
			
			More formally: The surjectivity of $f$ means that every ${ y \in Y }$ has a non-empty pre-image
			\[ I(y) = \setc{x \in X}{f(x) = y}. \]
			So we can define a function
			\[ g: Y \longmapsto X \suchthat g(y) = x \]
			for ${ x \in I(y) }$ (we assume that some selection rule is available). Since $g$ maps each $y$ to an element in its pre-image and these pre-images, by definition, must be disjoint, $g$ is an injection. Therefore, by \autoref{prop:injection-from-X-to-Y-implies-cardinality-of-X-leq-to-that-of-Y}, we have ${ \cardinality{Y} \leq \cardinality{X} }$.
		\end{proof}
	
		\begin{corollary}\label{coro:bijection-from-X-to-Y-implies-cardinality-of-X-eq-to-that-of-Y}
			If a function ${ f: X \longmapsto Y }$ is bijective then ${ \cardinality{Y} = \cardinality{X} }$.
		\end{corollary}
		\begin{proof}
			By definition of bijection, $f$ is an injection. Also, by \autoref{prop:function-inverses-are-bijective}, $\inv{f}$ is a bijection and therefore also an injection. So we have an injection in both directions
			\[ f: X \longmapsto Y \eqand \inv{f}: Y \longmapsto X. \]
			Therefore, by \autoref{prop:injection-from-X-to-Y-implies-cardinality-of-X-leq-to-that-of-Y},
			\[ \cardinality{X} \leq \cardinality{Y} \; \land \; \cardinality{Y} \leq \cardinality{X} \implies \cardinality{X} = \cardinality{Y}. \]
		\end{proof}
	
		\begin{corollary}
			The cardinality of the image of a function is less than or equal to that of the domain of the function. That's to say, if $I$ is the image of the function,
			\[ f: X \longmapsto Y \]
			then 
			\[ \cardinality{I} \leq \cardinality{X}. \]
		\end{corollary}
		\begin{proof}
			By the definition of the image of a function, the function is a surjection onto the image. Therefore, by \autoref{prop:surjection-from-X-to-Y-implies-cardinality-of-X-geq-to-that-of-Y}, we have ${ \cardinality{I} \leq \cardinality{X} }$.
		\end{proof}
		
		
		
		\sep
		\begin{exe}
			\ex \textit{In any room full of people, there will always be at least two people with the same number of friends in the room.}\\
			
			Say there are $n$ people in the room. Then, if there is a person in the room that is friends with all the other people in the room, then that person has ${ n - 1 }$ friends in the room. This is the maximum number of friends anyone can have in the room, and it also means that there is no-one with 0 friends in the room. Conversely, if there is someone with 0 friends in the room then there can't be anyone with ${ n - 1 }$ friends in the room.\\
			So, if we set up a function ${ f: P \longmapsto \N{} \cup \{0\} }$ from the set of people $P$ to the number of friends they have in the room, then the image of $f$ can contain 0 or $n-1$ but not both. So, the image of $f$,
			\[ f(P) \subseteq \N{}_{n-2} \cup \{0\} \eqword{or} f(P) \subseteq \N{}_{n-1}. \]
			Either way, the maximum cardinality of the image of $f$ is ${ n - 1 }$ which is obviously less than $n$, the number of people. Therefore, there cannot be any injection from the set of people to the set of numbers of friends and there must be two people with the same number of friends.
			
			\ex \textit{If there are 5 points in the plane $\R{2}$ with integer co-ordinates then there must be at least one pair, say $(x_1,y_1)$ and $(x_2,y_2)$, whose midpoint, given by
				\[ \left( \frac{x_1 + x_2}{2}, \, \frac{y_1 + y_2}{2} \right), \]
				has integer co-ordinates.}\\
			
			A midpoint co-ordinate --- say the $x$ co-ordinate with formula ${ \frac{x_1 + x_2}{2} }$ --- will be an integer when the sum ${ x_1 + x_2 }$ is even. By \autoref{prop:laws-of-addition-of-odd-and-even-numbers}, this will happen when both $x_1$ and $x_2$ are even or both are odd.\\
			For each of the 5 points the $x$ and $y$ co-ordinates may be odd or even in 4 different combinations: (odd, odd), (odd, even), (even, even), (even, odd). Since there are 5 points and only 4 possible combinations of odd and even, there cannot be an injection from points to odd/even combinations and, therefore, there must be, at least, one pair of points with the same combination. The midpoint of such a pair of points has integer co-ordinates.
			
			\ex \textit{Let ${ (a_1,a_2,\dots, a_n) }$ be a list of integers for ${ n > 1 }$. Then there exists a non-empty sublist whose sum is divisible by $n$.}\\
			
			Let $s_j$ be the partial sum ${ \sum_{i=1}^j a_i }$ for ${ 1 \leq j \leq n }$ and let ${ m_j = s_j \bmod n }$. Since there are only $n$ distinct values in modulo-$n$, the values $m_j$ must either:
			\begin{itemize}
				\item include all the modulo-$n$ values including 0, in which case the sublist ${ (a_1,a_2,\dots, a_j) }$ for the $j$ such that ${ m_j = 0 }$ has a sum that divides by $n$;
				\item or else there are ${ 1 \leq i < j \leq n }$ such that ${ m_j = m_i }$ and then the sublist ${ (a_{i+1},a_{i+2}, \dots, a_j) }$ has a sum that divides by $n$.
			\end{itemize}
			
			\TODO{complete this}
		\end{exe}
		\sep
	
	

	
		
		\biggerskip
		\subsubsection{Infinite Sets}
		
		\boxeddefinition{\textbf{(Infinite Set)} A set $X$ is described as \textit{infinite} if there is no ${ n \in \N{} }$ such that ${ \cardinality{X} = n }$.}
		
		\bigskip
		\labeledProposition{The set of natural numbers $\N{}$ is infinite.}{natural-numbers-is-infinite-set}
		\begin{proof}
			Suppose for contradiction that there is some ${ n \in \N{} }$ such that ${ \cardinality{\N{}} = n }$. Then there exists a bijection,
			\[ f: \N{}_n \longmapsto \N{}. \]
			Consider the value,
			\[ m = \sum_{i \in \N{}_n} f(i) = f(1) + f(2) + \cdots + f(n). \]
			By closure of addition of naturals we have ${ m \in \N{} }$ but for the naturals we also have,
			\[ \forall n \in \N{} \logicsep 1 \leq n \implies \forall i \in \N{}_n \logicsep f(i) < m. \]
			Therefore, there is no ${ i \in \N{}_n }$ such that ${ f(i) = m }$ which contradicts the surjectivity of $f$ and therefore also the bijectivity of $f$.
		\end{proof}
		\note{Note a certain similarity here between this proof and Euler's proof of the infinitude of the primes.}
		
	}
\end{document}