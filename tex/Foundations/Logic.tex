\documentclass[../MathsNotesBase.tex]{subfiles}

\date{\vspace{-6ex}}

\begin{document}

	\searchableSubsection{Logic}{logic}{
		\biggerskip
		\subsubsection{Basic Identities}
		
		\subheading{Negation of Universals and Existentials}
		\[ \lnot (\forall n \logicsep P(n)) \equiv \exists n \logicsep \lnot P(n) \]
		\[ \lnot (\exists n \logicsep P(n)) \equiv \forall n \logicsep \lnot P(n) \]
		
		\subheading{De Morgan's Laws}
		\[ \lnot (P \land Q) \equiv \lnot P \lor \lnot Q \]
		\[ \lnot (P \lor Q) \equiv \lnot P \land \lnot Q \]
		
		\subheading{Implication}
		\[ P \implies Q \equiv \lnot Q \lor P \]
		\[ P \implies Q \equiv \lnot Q \implies \lnot P \]
		\note{Note the (informally agreed) operator precedence rules for logical operators: \href{http://intrologic.stanford.edu/intrologic/glossary/operator_precedence.html}{Stanford}.}
	}

	\nl[12]
	\searchableSubsection{Sets}{set theory}{
		\bigskip
		\boxeddefinition{\textbf{(Intersection and Union)} Let $S$ be a set and ${ A,B \subseteq S }$. Then the intersection of $A$ and $B$ is defined as
			\[ A \cap B = \setc{x}{x \in A \; \land \; x \in B} \]
			and the union of $A$ and $B$ is defined as
			\[ A \cup B = \setc{x}{x \in A \; \lor \; x \in B}. \]
		}
	
		\boxeddefinition{\textbf{(Collection/Family of Sets)} Let $S$ be a set and $I$ be a non-empty indexing set (see: \href{https://en.wikipedia.org/wiki/Index_set}{wikipedia}) such that for each ${ i \in I }$ there exists ${ A_i \subseteq S }$. Then the set of sets,
			\[ \setc{A_i}{i \in I} \]
			is known as a \textit{collection} or \textit{family} of sets.
		}
		\notation{We denote the intersection of an indexed family of sets,
			\[ \bigcap_{i \in I} A_i = \setc{x}{ \forall i \in I \logicsep x \in A_i } \]
			and the union,
			\[ \bigcup_{i \in I} A_i = \setc{x}{ \exists i \in I \logicsep x \in A_i }. \]
		}
	
		\biggerskip
		\labeledTheorem{\textbf{(De Morgan Laws for Sets)} Let $S$ be a set and $I$ be a non-empty indexing set such that for each ${ i \in I }$ there exists ${ A_i \subseteq S }$. Then
			\begin{enumerate}[label=(\roman*)]
				\item  ${  S \setminus (\bigcap_I A_i) = \bigcup_I (S \setminus A_i)  }$
				\item  ${ S \setminus (\bigcup_I A_i) = \bigcap_I (S \setminus A_i).  }$
			\end{enumerate}
		}{set-de-morgan-laws}
		\begin{proof}\nl
			\begin{enumerate}[label=(\roman*)]
				\item  
				\[\begin{aligned}
					&S \setminus \bigcap_I A_i \\
					&= \setc{x}{ x \in S \; \land \lnot (x \in A_1 \land x \in A_2 \land \cdots \land x \in A_n) } \\
					&= \setc{x}{ x \in S \; \land [ \lnot (x \in A_1) \lor \lnot (x \in A_2) \lor \cdots \lor \lnot (x \in A_n) ] }  &\sidecomment{} \\
					&= \setc{x}{ [x \in S \; \land \lnot (x \in A_1)] \lor \cdots \lor [x \in S \; \land \lnot (x \in A_n)] }  &\sidecomment{} \\
					&= \bigcup_I (S \setminus A_i).
				\end{aligned}\]
				
				\bigskip
				\item
				\[\begin{aligned}
					&S \setminus (\bigcup_I A_i) \\
					&= \setc{x}{ x \in S \; \land \lnot (x \in A_1 \lor x \in A_2 \lor \cdots \lor x \in A_n) } &\sidecomment{} \\
					&= \setc{x}{ x \in S \; \land [\lnot (x \in A_1) \land \cdots \land \lnot (x \in A_n)] } &\sidecomment{} \\
					&= \setc{x}{ [x \in S \; \land \lnot (x \in A_1)] \land \cdots \land [x \in S \; \land \lnot (x \in A_n)] } &\sidecomment{} \\
					&= \bigcap_I (S \setminus A_i).
				\end{aligned}\]
			\end{enumerate}
		\end{proof}
	}
\end{document}