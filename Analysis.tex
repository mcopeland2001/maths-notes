\documentclass[MathsNotesBase.tex]{subfiles}




\date{\vspace{-6ex}}


\begin{document}
\searchableSubsection{\chapterTitle{Analysis}}{analysis}{\bigskip\bigskip}

	\searchableSubsection{\sectionTitle{Limits}}{analysis}{\bigskip}
	
	\bigskip\bigskip
	\searchableSubsection{Definitions of the Types of Limit}{analysis}{\bigskip
		\subsubsection{Problems with the informal description of a limit}
		If we say that a sequence tends to some value $L$ when the terms of the sequence \textit{gets closer and closer to} $L$ we have the following problems:
		\begin{itemize}
			\item{that the sequence gets closer and closer to many numbers so that this does not specify a single specific limit.}
			\item{that the sequence can have a limit but it's not the case that every term is closer than the previous term to the limit. For example,
				\[ a_{2k} = 1/k, \; a_{2k - 1} = \frac{1}{k+1} \]
				tends to 0 but $ a_{2k} > a_{2k - 1} $.
			}
		\end{itemize}
		\bigskip
		\boxeddefinition{A sequence $a_n$ is said to \textbf{tend} to $L$ or have the \textbf{limit} $L$ iff,
			\[ \forall \epsilon > 0 \in \R{} ,\, \exists N \in \N{} \suchthat \forall n > N ,\, \abs{a_{n} - L} < \epsilon. \]
		}
		\nomenclature{The interval $ (L - \epsilon, L + \epsilon) $ is called the \textbf{$\epsilon$-neighbourhood of $L$}.}
		
		\boxeddefinition{A sequence $a_n$ is said to \textbf{tend to infinity} iff,
			\[ \forall M > 0 \in \R{} ,\, \exists N \in \N{} \suchthat \forall n > N ,\, a_{n} > M \]
			and \textbf{tend to minus-infinity} iff,
			\[ \forall M < 0 \in \R{} ,\, \exists N \in \N{} \suchthat \forall n > N ,\, a_{n} < M. \]
		}
	
		\nomenclature{A sequence that has a limit is called \textbf{convergent} and otherwise is called \textbf{divergent}. Note that \textbf{divergent} sequences include both sequences that remain bounded but oscillate without converging and those that tend to infinity (or minus-infinity).}
	}
	
	
	\bigskip
	\searchableSubsection{Examples of Convergence and Divergence}{analysis}{\bigskip
		\subsubsection{Non-convergent Oscillation}
		The sequence $ a_n = (-1)^n $ is divergent despite always remaining bounded within the interval $ [-1, 1] $ as it neither converges to 1 or to -1.
			
		\subsubsection{Limit of an Infinite Recurrence}
		Take the sequence given by,
		\[ a_1 = 1, \; a_{n+1} = \frac{a_n}{2} + \frac{3}{2a_n} \;\; (n \ge 1). \]
		Assume there is an equilibrium value, $a^*$, then
		\begin{align*}
		&&a^* &= \frac{a^*}{2} + \frac{3}{2a^*}  \\
		&\iff &2(a^*)^2  &= (a^*)^2 + 3  &\sidecomment{}\\
		&\iff &(a^*)^2  &= 3  &\sidecomment{}\\
		&\iff &a^*  &= \sqrt{3}  &\sidecomment{}\\
		\end{align*}
		So $\sqrt{3}$ is the steady-state value that this recurrence converges to as $n \to \infty$. If the recurrence didn't converge then the assumption of an equilibrium value would result in a contradiction.
	}

	\bigskip\bigskip
	\searchableSubsection{Properties of Limits}{analysis}{\bigskip
		\labeledProposition{A sequence has at most one limit. In other words, a sequence can only converge, if at all, to a single unique value.}{uniqueness_of_limits}
		\begin{proof}
			Let $L$ and $L'$ both be limits of the sequence $a_n$, and the constant $\alpha = L - L'$. Then,
			\[ \forall \epsilon > 0 \in \R{} ,\, \exists N \in \N{} \suchthat \forall n > N ,\, \abs{a_{n} - L} < \epsilon \] and
			\[ \forall \epsilon' > 0 \in \R{} ,\, \exists N' \in \N{} \suchthat \forall n' > N' ,\, \abs{a_{n'} - L'} < \epsilon'. \]
			But also we have,
			\[ \abs{a_n - L'} = \abs{(a_n - L) + (L - L')} = \abs{(L - L') + (a_n - L)} \]
			and using the triangle inequality,
			\begin{align*}
			&& \abs{L - L'} = \abs{(L - L') + (a_n - L) - (a_n - L)} &\leq \abs{(L - L') + (a_n - L)} + \abs{-(a_n - L)} \\
			&\iff & \abs{L - L'}  &\leq \abs{(L - L') + (a_n - L)} + \abs{a_n - L}\\
			&\iff & \abs{L - L'} - \abs{a_n - L}  &\leq \abs{(L - L') + (a_n - L)}\\
			&\iff & \abs{\alpha} - \abs{a_n - L}  &\leq \abs{\alpha + (a_n - L)}\\
			\end{align*}
			Since $\alpha = L - L'$ is constant we can consider the situation when $\epsilon = \frac{\abs{\alpha}}{2}$ then we have that,
			\begin{align*}
			&& \exists N \in \N{} \suchthat \forall n > N ,\, \abs{a_{n} - L} &< \epsilon = \frac{\abs{\alpha}}{2} \\
			&\iff &  -\abs{a_{n} - L} &> -\frac{\abs{\alpha}}{2}  &\sidecomment{}\\
			&\iff & \abs{\alpha} - \abs{a_{n} - L} &> \abs{\alpha} - \frac{\abs{\alpha}}{2}  &\sidecomment{}\\
			&\iff & \abs{\alpha} - \abs{a_{n} - L} &> \frac{\abs{\alpha}}{2}.  &\sidecomment{}\\
			\end{align*}
			Combining this with the previous result gives, for $\forall n > N$,
			\[ \frac{\abs{\alpha}}{2} < \abs{\alpha} - \abs{(a_n - L)}  \leq \abs{\alpha + (a_n - L)} = \abs{a_n - L'} \]
			which, by rearranging a little, is,
			\[ \forall n > N,\, \abs{a_n - L'} > \frac{\abs{\alpha}}{2}. \]
			But this means that if we also choose $\epsilon' = \frac{\abs{\alpha}}{2}$ then there is no $N'$ such that $\forall n' > N' ,\, \abs{a_{n'} - L'} < \epsilon'$ which contradicts the hypothesis that $L'$ is also a limit of $a_n$.
		\end{proof}
		\begin{proof}
			Another quicker way of proving the proposition is by letting $ \epsilon = \epsilon' = \frac{\abs{\alpha}}{2} $ so that,
			\[ 2\epsilon = \abs{\alpha} = \abs{L - L'} = \abs{L - a_n + a_n - L'} \leq \abs{L - a_n} + \abs{a_n - L'} = \abs{a_n - L} + \abs{a_n - L'} \]
			which gives us,
			\[ 2\epsilon \leq \abs{a_n - L} + \abs{a_n - L'}. \]
			But by the limit definition,
			\[ \abs{a_n - L} + \abs{a_n - L'} < \epsilon + \epsilon' \]
			and since we have set $ \epsilon = \epsilon' $ then,
			\[ \abs{a_n - L} + \abs{a_n - L'} < 2\epsilon \]
			which contradicts $ 2\epsilon \leq \abs{a_n - L} + \abs{a_n - L'} $.
		\end{proof}
	}

\end{document}