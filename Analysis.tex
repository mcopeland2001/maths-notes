\documentclass[MathsNotesBase.tex]{subfiles}




\date{\vspace{-6ex}}


\begin{document}
\searchableSubsection{\chapterTitle{Analysis}}{analysis}{\bigskip\bigskip}

	\searchableSubsection{\sectionTitle{Supremum and Infimum}}{analysis}{\bigskip}
	
	\bigskip\bigskip
	\searchableSubsection{Definitions}{analysis}{\bigskip
		\boxeddefinition{An upper bound on a set A is a value $x$ such that,
			\[ \forall a \in A, \, a \leq x \]
			and a lower bound is similarly defined as a value $y$ such that,
			\[ \forall a \in A, \, a \geq y. \]
			A set is said to be \textbf{upper-bounded} if there exists some upper-bound on the set and is said to be \textbf{lower-bounded} if there exists some lower bound on the set. If there exists both upper and lower bounds then the set is said to be \textbf{bounded}.
		}
		\boxeddefinition{The \textbf{supremum} of a upper-bounded set $A$ is a value $\sigma_A$ such that $\sigma_A$ is an upper bound on $A$ and, 
			\[ \sigma_A' < \sigma_A \iff \exists a \in A \suchthat a > \sigma_A' \]
			which is to say that if $\sigma_A' < \sigma_A$ then $\sigma_A'$ is not an upper bound on $A$ and, if $\sigma_A'$ is not an upper bound on $A$ then it must be less than $\sigma_A$ since $\sigma_A$ is an upper bound on A.\\
			An alternative, equivalent definition is,
			\[ \forall \epsilon > 0, \, \exists a \in A \suchthat a > \sigma_A - \epsilon. \]
		}
		\paragraph{Issue} Note that there is an apparent paradox here: This second definition implies that
		\begin{align*}
			&&  \forall \epsilon > 0 \logicsep \exists a \in A &\suchthat a + \epsilon > \sigma_A  \\
			&\iff &\exists a \in A &\suchthat a \geq \sigma_A  &\sidecomment{}
		\end{align*}
		which result, when combined with the upper-bound property, gives
		\begin{align*}
			&& \exists a \in A &\suchthat (a \geq \sigma_A) \wedge (a \leq \sigma_A) \\
			&\iff &\exists a \in A &\suchthat a = \sigma_A  &\sidecomment{}
		\end{align*}
		which says that there is always an element in the bounded set that is equal to the supremum. This is not correct - the supremum may be in the set or external to it.\\
		The initial implication is not true, however. We cannot infer that ${ \exists a \in A \suchthat a \geq \sigma_A }$. This can be seen with another rearrangement,
		\begin{align*}
			&&  \forall \epsilon > 0 \logicsep \exists a \in A &\suchthat a + \epsilon > \sigma_A  \\
			&\iff &\forall \epsilon > 0 \logicsep \exists a \in A &\suchthat \epsilon > \sigma_A - a  &\sidecomment{}
		\end{align*}
		which shows us that for any positive epsilon there needs to be an $a$ close enough to the value of $\sigma_A$ that the difference in their values is less than epsilon. Since $a$ can approach arbitrarily close to $\sigma_A$ this is achievable for any positive epsilon. This property seems to be equivalent to the fact that $\sigma_A$ is a \textit{limit point} of $A$ but that will be covered properly in Topology.
		\bigskip\bigskip
		
		\boxeddefinition{The \textbf{infimum} of a lower-bounded set $A$ is defined similarly to the supremum: as a value $\tau_A$ such that $\tau_A$ is a lower bound on $A$ and, 
			\[ \tau_A' > \tau_A \iff \exists a \in A \suchthat a < \tau_A' \]
			or alternatively,
			\[ \forall \epsilon > 0, \, \exists a \in A \suchthat a < \tau_A + \epsilon. \]
		}
		\notation{The supremum of $A$ is denoted $sup \, A$ and the infimum is denoted $inf \, A$.}
	}
	\searchableSubsection{Deductions using the supremum and infimum}{analysis}{\bigskip
		\labeledProposition{If a bounded set $A \subset \R{}$ has the property that,
			\[ \forall x,y \in A \logicsep \abs{x - y} < 1 \]
			then it follows that,
			\[ (sup \, A - inf \, A) \leq 1. \]
		}{sup_minus_inf_max_difference}
		\begin{proof}
			Let $ \sigma_A = sup \, A $ and $ \tau_A = inf \, A $ and \WLOG assume that $x > y$. By the definitions of the supremum and infimum we have,
			\begin{align*}
			&& \forall \epsilon > 0 \logicsep \exists x,y \in A &\logicsep (x > \sigma_A - \epsilon) \wedge (y < \tau_A + \epsilon)  \\
			&\iff & \forall \epsilon > 0 \logicsep \exists x,y \in A &\logicsep (x > \sigma_A - \epsilon) \wedge (-y > -\tau_A - \epsilon)  &\sidecomment{}\\
			&\iff & \forall \epsilon > 0 \logicsep \exists x,y \in A &\logicsep (x - y) > (\sigma_A - \tau_A) - 2\epsilon  &\sidecomment{}\\
			\end{align*}
			Now suppose, for contradiction, that $ (\sigma_A - \tau_A) > 1 $ then we can say that,
			\[ \exists r > 0 \logicsep (\sigma_A - \tau_A) = 1 + r. \] 
			If we then constrict $\epsilon$ such that,
			\[ \epsilon < \frac{r}{2} \iff 2\epsilon < r \iff r - 2\epsilon > 0 \]
			then the previous result tells us that, for $ 0 < \epsilon < \frac{r}{2} $,
			\begin{align*}
			&& \exists x,y \in A &\logicsep (x - y) > (\sigma_A - \tau_A) - 2\epsilon  \\
			&\iff & \exists x,y \in A &\logicsep (x - y) > 1 + r - 2\epsilon > 1  &\sidecomment{}
			\end{align*}
			which contradicts the set property that $ \forall x,y \in A, \, \abs{x - y} < 1 $. So this shows that $ (\sigma_A - \tau_A) \leq 1 $.
		\end{proof}
		\bigskip
		\labeledProposition{Let $A \subset \R{}$ be a bounded set and let $B$ be the set defined by
			\[ B = \setc{b}{b = f(a), \; a \in A} \]
			where the function $f$ is some strictly monotonic function.
			Then it follows that,
			\[ sup \, B = f(sup \, A). \]
		}{monotonic_functions_preserve_supremum}
		\begin{proof}
			A is bounded and so $\sigma_A = sup \, A$ exists. So, using the supremum properties we have,
			\begin{align*}
			&& \forall a \in A \logicsep a &\leq \sigma_A  \\
			&\iff & \forall a \in A \logicsep f(a)  &\leq f(\sigma_A)  &\sidecomment{by monotonicity of f}\\
			&\iff & \forall b \in B \logicsep b  &\leq f(\sigma_A) 
			\end{align*}
			which is to say that $\sigma_B = f(\sigma_A)$ is an upper bound on $B$.\\
			Furthermore, using the other supremum property, we have that,
			\begin{align*}
			&& \sigma_A' < \sigma_A &\implies \exists a \in A \suchthat a > \sigma_A' \\
			&\iff & f(\sigma_A') < f(\sigma_A)  &\implies \exists a \in A \suchthat f(a) > f(\sigma_A')  &\sidecomment{by strict monotonicity of f}\\
			&\iff & \sigma_B' < \sigma_B  &\implies \exists b \in B \suchthat b > \sigma_B'.	
			\end{align*}
			Therefore $\sigma_B$ satisfies both requirements of the supremum and we have shown that,
			\[ sup \, B = f(sup \, A). \]
		\end{proof}
	}
	
	
	\pagebreak
	\searchableSubsection{\sectionTitle{Limits}}{analysis}{\bigskip}
	
	\bigskip\bigskip
	\searchableSubsection{Limits of sequences}{analysis}{\bigskip
		\subsubsection{Problems with the informal description of a limit}
		If we say that a sequence tends to some value $L$ when the terms of the sequence \textit{gets closer and closer to} $L$ we have the following problems:
		\begin{itemize}
			\item{that the sequence gets closer and closer to many numbers so that this does not specify a single specific limit.}
			\item{that the sequence can have a limit but it's not the case that every term is closer than the previous term to the limit. For example,
				\[ a_{2k} = 1/k, \; a_{2k - 1} = \frac{1}{k+1} \]
				tends to 0 but $ a_{2k} > a_{2k - 1} $.
			}
		\end{itemize}
		\bigskip
		\boxeddefinition{A sequence $a_n$ is said to \textbf{tend} to $L$ or have the \textbf{limit} $L$ iff,
			\[ \forall \epsilon > 0 \in \R{} ,\, \exists N \in \N{} \suchthat \forall n > N ,\, \abs{a_{n} - L} < \epsilon. \]
		}
		\boxeddefinition{The interval $ (L - \epsilon, L + \epsilon) $ is called the \textbf{\mbox{$\epsilon$-neighbourhood of $L$}}.}
		
		\boxeddefinition{A sequence $a_n$ is said to \textbf{tend to infinity} iff,
			\[ \forall M > 0 \in \R{} ,\, \exists N \in \N{} \suchthat \forall n > N ,\, a_{n} > M \]
			and \textbf{tend to minus-infinity} iff,
			\[ \forall M < 0 \in \R{} ,\, \exists N \in \N{} \suchthat \forall n > N ,\, a_{n} < M. \]
		}
	
		\boxeddefinition{A sequence that has a limit is called \textbf{convergent} and otherwise is called \textbf{divergent}. Note that \textbf{divergent} sequences include both sequences that remain bounded but oscillate without converging and those that tend to infinity (or minus-infinity).}
		
		\subsubsection{Examples of Convergence and Divergence}\bigskip
			\begin{exe}
				\ex {\textbf{Non-convergent Oscillation}\\\\
				The sequence $ a_n = (-1)^n $ is divergent despite always remaining bounded within the interval $ [-1, 1] $ as it neither converges to 1 or to -1.}\label{ex:flipping_sign}
				\ex {\textbf{Limit of an Infinite Recurrence}\\\\
					Take the sequence given by,
					\[ a_1 = 1, \; a_{n+1} = \frac{a_n}{2} + \frac{3}{2a_n} \;\; (n \ge 1). \]
					Assume there is an equilibrium value, $a^*$, then
					\begin{align*}
						&&a^* &= \frac{a^*}{2} + \frac{3}{2a^*}  \\
						&\iff &2(a^*)^2  &= (a^*)^2 + 3  &\sidecomment{}\\
						&\iff &(a^*)^2  &= 3  &\sidecomment{}\\
						&\iff &a^*  &= \sqrt{3}  &\sidecomment{}\\
					\end{align*}
					So $\sqrt{3}$ is the steady-state value that this recurrence converges to as $n \to \infty$. If the recurrence didn't converge then the assumption of an equilibrium value would result in a contradiction.
				}\label{ex:infinite_recurrence}
			\end{exe}
			
		\bigskip	
		\labeledProposition{A sequence has at most one limit. In other words, a sequence can only converge, if at all, to a single unique value.}{uniqueness_of_limits}
		\begin{proof}
			Let $L$ and $L'$ both be limits of the sequence $a_n$, and the constant $\alpha = L - L'$. Then,
			\[ \forall \epsilon > 0 \in \R{} ,\, \exists N \in \N{} \suchthat \forall n > N ,\, \abs{a_{n} - L} < \epsilon \] and
			\[ \forall \epsilon' > 0 \in \R{} ,\, \exists N' \in \N{} \suchthat \forall n' > N' ,\, \abs{a_{n'} - L'} < \epsilon'. \]
			But also we have,
			\[ \abs{a_n - L'} = \abs{(a_n - L) + (L - L')} = \abs{(L - L') + (a_n - L)} \]
			and using the triangle inequality,
			\begin{align*}
			&& \abs{L - L'} = \abs{(L - L') + (a_n - L) - (a_n - L)} &\leq \abs{(L - L') + (a_n - L)} + \abs{-(a_n - L)} \\
			&\iff & \abs{L - L'}  &\leq \abs{(L - L') + (a_n - L)} + \abs{a_n - L}\\
			&\iff & \abs{L - L'} - \abs{a_n - L}  &\leq \abs{(L - L') + (a_n - L)}\\
			&\iff & \abs{\alpha} - \abs{a_n - L}  &\leq \abs{\alpha + (a_n - L)}\\
			\end{align*}
			Since $\alpha = L - L'$ is constant we can consider the situation when $\epsilon = \frac{\abs{\alpha}}{2}$ then we have that,
			\begin{align*}
			&& \exists N \in \N{} \suchthat \forall n > N ,\, \abs{a_{n} - L} &< \epsilon = \frac{\abs{\alpha}}{2} \\
			&\iff &  -\abs{a_{n} - L} &> -\frac{\abs{\alpha}}{2}  &\sidecomment{}\\
			&\iff & \abs{\alpha} - \abs{a_{n} - L} &> \abs{\alpha} - \frac{\abs{\alpha}}{2}  &\sidecomment{}\\
			&\iff & \abs{\alpha} - \abs{a_{n} - L} &> \frac{\abs{\alpha}}{2}.  &\sidecomment{}\\
			\end{align*}
			Combining this with the previous result gives, for $\forall n > N$,
			\[ \frac{\abs{\alpha}}{2} < \abs{\alpha} - \abs{(a_n - L)}  \leq \abs{\alpha + (a_n - L)} = \abs{a_n - L'} \]
			which, by rearranging a little, is,
			\[ \forall n > N,\, \abs{a_n - L'} > \frac{\abs{\alpha}}{2}. \]
			But this means that if we also choose $\epsilon' = \frac{\abs{\alpha}}{2}$ then there is no $N'$ such that $\forall n' > N' ,\, \abs{a_{n'} - L'} < \epsilon'$ which contradicts the hypothesis that $L'$ is also a limit of $a_n$.
		\end{proof}
		\begin{proof}
			Another quicker way of proving the proposition is by letting $ \epsilon = \epsilon' = \frac{\abs{\alpha}}{2} $ so that,
			\[ 2\epsilon = \abs{\alpha} = \abs{L - L'} = \abs{L - a_n + a_n - L'} \leq \abs{L - a_n} + \abs{a_n - L'} = \abs{a_n - L} + \abs{a_n - L'} \]
			which gives us,
			\[ 2\epsilon \leq \abs{a_n - L} + \abs{a_n - L'}. \]
			But by the limit definition,
			\[ \abs{a_n - L} + \abs{a_n - L'} < \epsilon + \epsilon' \]
			and since we have set $ \epsilon = \epsilon' $ then,
			\[ \abs{a_n - L} + \abs{a_n - L'} < 2\epsilon \]
			which contradicts $ 2\epsilon \leq \abs{a_n - L} + \abs{a_n - L'} $.
		\end{proof}
	
		\bigskip\bigskip
		\boxeddefinition{If $a_n$ is a sequence and $ S = \setc{a_n}{n \in \N{}} $ then $a_n$ is said to be \textbf{bounded below} if $S$ has a lower bound and \textbf{bounded above} if $S$ has an upper bound, and \textbf{bounded} if it is bounded above and below.}
		
		\bigskip
		\labeledProposition{Any convergent sequence is bounded.}{convergent_seqs_bounded}
		\begin{proof}
			Firstly, we need to prove that any finite sequence is bounded. We can do this simply by observing that any finite set of numbers,
			\[ S = \{a_1, a_2, \dots , a_n\} \]
			is a well-founded set and so has a minimum and a maximum.\\
			Now, let $a_n$ be an arbitrary convergent sequence so that,
			\[ \forall \epsilon > 0 \logicsep \exists N \in \N{} \logicsep \forall n > N \logicsep \abs{a_n - L} < \epsilon \]
			for some $L \in \R{}$.\\\\
			Then, let $S_{max}$ and $S_{min}$ be the maximum and minimum respectively of the first $N$ terms of $a_n$, $S = \{a_1, a_2, \dots , a_N\}$, and,
			\[ \exists \epsilon > 0 \logicsep \forall n > N \logicsep \abs{a_n - L} < \epsilon \]
			so that, for $n > N$, the sequence $a_n$ is bounded in the $\epsilon$-neighbourhood of $L$.\\
			So, if we define ${ m = min\{S_{min}, L - \epsilon\} }$ and ${ M = max\{S_{max}, L + \epsilon\} }$, then the whole sequence $a_n$ for all $n \in \N{}$ is bounded below by $m$ and bounded above by $M$.\\
			Therefore $a_n$ is bounded. 
		\end{proof}
	
		\bigskip
		\boxeddefinition{An \textbf{increasing} sequence is a sequence $a_n$ such that,
			\[ \forall n \in \N{} \logicsep a_{n+1} \geq a_n \]
			and \textbf{decreasing} if,
			\[ \forall n \in \N{} \logicsep a_{n+1} \leq a_n \]
			and \textbf{monotonic} if either increasing or decreasing.
		}
		
		\bigskip
		\labeledProposition{Any increasing sequence that is bounded above has a limit.}{increasing_bounded_seqs_have_limit}
		\begin{proof}
			Let $a_n$ be an increasing sequence that is bounded above. Then,
			\[ \forall n \in \N{} \logicsep a_{n+1} \geq a_n \]
			and let ${ S = \setc{a_n}{n \in \N{}} }$. Since $a_n$ is bounded above it has a supremum. Let $ \sigma = sup \; S $ so that,
			\[ \forall a_n \in S \logicsep a_n \leq \sigma \eqand \forall \epsilon > 0 \logicsep \exists a_n \in S \logicsep a_n > \sigma - \epsilon. \]
			Therefore, for some arbitrary fixed $\epsilon > 0$,
			\[ \exists a_n \in S \logicsep a_n > \sigma - \epsilon \] 
			and setting $ N = n $ so that $ a_N > \sigma - \epsilon $, we have,
			\[ \forall n > N \in \N{} \logicsep a_n > a_N > \sigma - \epsilon \]
			and so, recalling that $\sigma$ is an upper bound on $S$,
			\begin{align*}
			&& \exists N \in \N{} \logicsep \forall n > N &\logicsep (a_n > \sigma - \epsilon) \wedge (a_n \leq \sigma) \\
			&\iff &\exists N \in \N{} \logicsep \forall n > N &\logicsep a_n \leq \sigma < a_n + \epsilon  &\sidecomment{}\\
			&\iff &\exists N \in \N{} \logicsep \forall n > N &\logicsep 0 \leq \sigma - a_n < \epsilon  &\sidecomment{}\\
			&\implies &\exists N \in \N{} \logicsep \forall n > N &\logicsep \abs{\sigma - a_n} < \epsilon.
			\end{align*}
			But $\epsilon$ was an arbitrary positive value so,
			\[ \forall \epsilon > 0 \logicsep \exists N \in \N{} \logicsep \forall n > N \logicsep \abs{\sigma - a_n} < \epsilon \]
			and $\sigma$ is, therefore, the limit of $a_n.$
		\end{proof}
	}
	

\end{document}