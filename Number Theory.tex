\documentclass[MathsNotesBase.tex]{subfiles}

\date{\vspace{-6ex}}

\begin{document}
\searchableSubsection{\chapterTitle{Number Theory}}{number theory}{\bigskip\bigskip}

	\searchableSubsection{\sectionTitle{Natural Numbers}}{number theory}{\bigskip}
			
			\searchableSubsection{Peano Axioms}{number theory}{\bigskip
				\begin{axiom}{Closure under addition:}\\ For all $a,b \in \N{}$ we have $a + b \in \N{}$.	
				\end{axiom}
				\begin{axiom}{Closure under multiplication:}\\ For all $a,b \in \N{}$ we have $a \times b \in \N{}$.
				\end{axiom}
				\begin{axiom}{Commutative Law for addition:}\\ For all $a,b \in \N{}$ we have $a + b = b + a$.
				\end{axiom}
				\begin{axiom}{Associative Law for addition:}\\ For all $a,b,c \in \N{}$ we have $(a + b) + c = a + (b + c)$.
				\end{axiom}
				\begin{axiom}{Commutative Law for multiplication:}\\ For all $a,b \in \N{}$ we have $a \times b = b \times a$.
				\end{axiom}
				\begin{axiom}{Associative Law for multiplication:}\\ For all $a,b,c \in \N{}$ we have $(a \times b) \times c = a \times (b \times c)$.
				\end{axiom}
				\begin{axiom}{Multiplicative Identity:}\\ There is a special element of $\N{}$, denoted by 1, which has the property that for all $n \in \N{},\; n \times 1 = n$.
				\end{axiom}
				\begin{axiom}{Additive cancellation:}\\ For all $a,b,c \in \N{}$ if $a + c = b + c \text{ then } a = b$.
				\end{axiom}
				\begin{axiom}{Multiplicative cancellation:}\\ For all $a,b,c \in \N{}$ if $a \times c = b \times c \text{ then } a = b$.
				\end{axiom}
				\begin{axiom}{Distributive Law:}\\ For all $a,b,c \in \N{},\; a \times (b + c) = (a \times b) + (b \times c)$.
				\end{axiom}
				\begin{axiom}{Definition of "less than":}\\ For all $a,b \in \N{},\; a < b$ if and only if there is some $c \in \N{} \suchthat a + c = b$.
				\end{axiom}
				\begin{axiom}{Trichotomous property:}\\ For all $a,b \in \N{}$ exactly one of the following is true: $a = b,\; a < b,\; b < a$.
				\end{axiom}
			}
		
		\paragraph{Notation:} We also write $ab$ for $a \times b$.
		
		\searchableSubsection{Properties following from these axioms}{number theory}{\bigskip
			
			\labeledProposition{If $a,b \in \N{}$ satisfy $a \times b = a$, then $b = 1$.}{mult_identity_unique}
				\begin{proof}
				\begin{align*}
				&& a \times b &= a = a \times 1 &\sidecomment{by Multiplicative Identity axiom}\\
				&\iff & b \times a &= 1 \times a  &\sidecomment{by Commutative Law for multiplication}\\
				&\iff & b &= 1  &\sidecomment{by Multiplicative cancellation}\\
				\end{align*}
				\end{proof}
			
			\labeledProposition{If $a,b,c \in \N{}$ and $a < b$ then $a \times c < b \times c$.}{less_than_mult_augment}
				\begin{proof}
				\begin{align*}
				&& a < b &\implies a + d = b \text{ for some } d \in \N{} &\sidecomment{by Definition of "less than"}\\
				&\therefore & b \times c &= (a + d) \times c = (a \times c) + (d \times c)  &\sidecomment{by Distributive Law}\\
				&\therefore & a \times c &< (a \times c) + (d \times c) = b \times c  &\sidecomment{by defn. "less than" and closure}
				\end{align*}
				\end{proof}
			
			\labeledProposition{$1$ is the least element of $\N{}$.}{mult_identity_least_element}
			\begin{proof}
				Assume $m$ is the least element of $\N{}$. Then, also $m < 1$. So, by \autoref{prop:less_than_mult_augment},
				\begin{align*}
				m < 1 \implies m \times m < 1 \times m = m
				\end{align*}
				But, closure of multiplication and $m \times m < m$ together contradict the assumption that $m$ is the least element of $\N{}$.\\
				Therefore $m$ cannot be less than $1$. Since we know that $1 \in \N{}$ and that the minimum element of $\N{}$, $m$, cannot be less than $1$, it follows that $1$ must be the minimum element of $\N{}$ and $m = 1$.
			\end{proof}
		}
	
	
	\searchableSubsection{\sectionTitle{Integers}}{number theory}{\bigskip}

		\searchableSubsection{Odd and Even Numbers}{number theory}{\bigskip
			\begin{definition}
			An \textbf{even} number, $n \in \Z{}$, is one that satisfies,
			\[ \exists\,m \in \Z\; \cdot \; n = 2m \]
			\end{definition}
			
			\begin{definition}
			An \textbf{odd} number, $n \in \Z{}$, is one that satisfies,
			\[ \exists\,m \in \Z\; \cdot \; n = 2m + 1 \]
			\end{definition}
			
			\subsubsection{Consequences}
			Sum of even numbers, $m + n$:
			\begin{align*}
			m + n &= 2k + 2l \hspace{15pt} \text{where $k,l \in \Z{}$} &\sidecomment{by defn. of even no.s $m,n$} \\
				  &= 2(k + l) \\
				  &= 2q      \hspace{15pt} \text{where $q \in \Z{}$}
			\end{align*}
			So $m + n$ is also even.
			However, if $m + n$ is even:
			\begin{align*}
			m + n &= 2k \hspace{15pt} \text{where $k \in \Z{}$} &\sidecomment{by defn. of even $m + n$} \\
			k &= \frac{m}{2} + \frac{n}{2}\\
			\end{align*}
			So $m$ and $n$ are not necessarily even. A counterexample is 
			\[3 + 5 = 8 \iff \frac{3}{2} + \frac{5}{2} = 4 \]
			To summarize:
			\begin{itemize}
			\item{$m,n$ even $ \implies m + n$ even}
			\item{$m + n$ even $ \implies m,n$ even \wrong} 
			\end{itemize}
		}
		

		\searchableSubsection{Divisibility and Primes}{number theory}{\bigskip
			\subsubsection{Greatest Common Divisor \tiny{(also called Highest Common Factor)}}
			The greatest common divisor of 16 and 6 can be visualized as follows:
			\begin{align*}			
			\vert \textcolor{blue}{\cdot \cdot \cdot\, \cdot} \vert \textcolor{red}{\cdot \cdot \cdot \cdot \cdot\, \cdot} \vert \textcolor{red}{\cdot \cdot \cdot \cdot \cdot\, \cdot} \vert &&\sidecomment{$16 = 6 \times 2 + 4$}\\			
			\cdot \cdot \cdot \cdot \cdot \cdot \cdot \cdot \cdot \cdot \vert \textcolor{green}{\cdot\, \cdot} \vert \textcolor{blue}{\cdot \cdot \cdot\, \cdot} \vert &&\sidecomment{$6 = 4 \times 1 + 2$}\\		
			\cdot \cdot \cdot \cdot \cdot \cdot \cdot \cdot \cdot \cdot \cdot \cdot \vert \textcolor{green}{\cdot\, \cdot} \vert \textcolor{green}{\cdot\, \cdot} \vert &&\sidecomment{$4 = 2 \times 2 + 0$}\\		
			\end{align*}
			This implies the algorithm:		
			\begin{align*}
			\textbf{gcd}(&a, b):\\
				&\text{if }\; b == 0\; \text{ then}\\
				&\;\;\;\;\text{return } \; a\\
				&\text{else}\\
				&\;\;\;\;\text{return } \;\textbf{gcd}(b, b \bmod a)\\
				&\text{end if}					
			\end{align*}
			
			\labeledProposition{For non-zero integers $a$ and $b$, if $a = bq + r$ where $q,r \in \Z{}$, then $gcd(a, b) = gcd(b, r) = gcd(b, a \bmod b)$.}{gcd_remainder}
				\begin{proof}
				$(a \bmod b) = r = a - bq$. For any $m \suchthat m \divides a \text{ and } m \divides b$ we must also have $m \divides (a - bq)$ so the set of divisors of $a$ and $b$ is a subset of the set of divisors of $b$ and $r = (a \bmod b)$. Conversely, for any $m \suchthat m \divides b \text{ and } m \divides r = (a \bmod b)$ we have that $m \divides (bq + r) = a$ so the set of divisors of $b$ and $r = (a \bmod b)$ is a subset of the set of divisors of $a$ and $b$. So the sets are equal proving that they must have the same maximum element - the greatest common divisor.
				\end{proof}
		}
	
		\searchableSubsection{The Fundamental Theorem of Arithmetic}{number theory}{\bigskip
			\paragraph{Definition of prime number:} An integer that is only divided cleanly by itself and one.
			More formally, an integer, $p$, is prime if it is greater than 1 and,
			\[ \exists!\,m,n \in \Z\; \cdot \;\frac{p}{m} = n \wedge (m \neq p \wedge m \neq 1) \]
			
			\paragraph{Primality $ \implies $ Unique Prime Factorization:}
			\begin{quote}
			``Any number either is prime or is measured by some prime number.''\\
			\textit{Euclid, Elements Book VII, Proposition 32}
			\end{quote}
			So, if an integer $n$ is not prime then,
			\begin{align*} 
			\exists\,a,b \in \Z\; \cdot \; \frac{n}{a} = b \\
			\iff n = ab
			\end{align*}
			Then, for $a$ (the same applies to $b$),
			\begin{align*}
			\exists\,c,d \in \Z\,,\,c,d \not\in \{1,a\}\; \cdot \; \frac{n}{a} = b \\
			\iff n = cd
			\end{align*}
			We can continue to descend like this until we must eventually encounter one or more primes.
			Furthermore, if a number, $n$, has a prime factorization, $p_1p_2$ then,
			\[ n = p_1p_2 = p_3p_4 \iff \frac{p_1}{p_3} = \frac{p_4}{p_2} = n \]
			But $\frac{p_1}{p_3} = n$ contradicts the definition of primeness of $p_1$. Therefore prime factorizations are unique.
			
			\paragraph{Proof of existence}
			\begin{proof}
			It must be shown that every integer greater than $ 1 $ is either prime or a product of primes. First, $ 2 $ is prime. Then, by strong induction, assume this is true for all numbers greater than $ 1 $ and less than $ n $. 
			If $ n $ is prime, there is nothing more to prove. 
			Otherwise, there are integers $ a, b $ where $ n = ab $, and $ 1 < a \leq b < n $. By the induction hypothesis, $ a = p_1p_2...p_j \text{ and } b = q_1q_2...q_k $ are products of primes. But then $ n = ab = p_1p_2...p_jq_1q_2...q_k $ is a product of primes.
			\end{proof}
		
			\paragraph{Proof of uniqueness}
			\begin{proof}
			Suppose, to the contrary, that there is an integer that has two distinct prime factorizations. Let $ n $ be the least such integer and write $ n = p_1 p_2 ... p_j = q_1 q_2 ... q_k $, where each $ p_i \text{ and } q_i $ is prime. (Note that $ j $ and $ k $ are both at least $ 2 $.) We see that $ p_1 \text{ divides } q_1 q_2 ... q_k\text{ , so } p_1\text{  divides some } q_i $ by Euclid's lemma. Without loss of generality, say that $ p_1 \text{ divides } q_1 $. Since $ p_1 $ and $ q_1 $ are both prime, it follows that $ p_1 = q_1 $. Returning to our factorizations of $ n $, we may cancel these two terms to conclude that $ p_2 ... p_j = q_2 ... q_k $. We now have two distinct prime factorizations of some integer strictly smaller than $ n $, which contradicts the minimality of $ n $.
			\end{proof}
		}
	
		\searchableSubsection{Some Proofs on the Integers}{number theory}{\bigskip
			\labeledProposition{For any integer $m$, $\sqrt{m}$ is rational iff $m$ is a square, i.e. $m=a^2$ for some integer $a$.}{rationality_of_sqrt_integers}
			To begin with we show the easier direction of implication: $(m = a^2) \implies$ ($\sqrt{m}$ is rational).
			\begin{proof}
			Assume $m,a,b \in \Z{}$.
			\begin{align*}
			m &= a^2 \\
			\iff \sqrt{m} &= \abs{a} \\
			&= a/b \text{ for } b = 1 \text{ or } -1. \qedhere
			\end{align*}
			\end{proof}
			Now the other (harder) direction, ($\sqrt{m}$ is rational) $\implies (m = a^2)$.
			\begin{proof}
			Assume $m,a,b \in \Z{}$. ($\sqrt{m}$ is rational) can be formalized as:
			\[ \exists\,m,a,b \in \Z{} \cdot (\sqrt{m} = \frac{a}{b}) \; \wedge \; \text{($a$ and $b$ are coprime)} \]
			\begin{align*}
			\sqrt{m} &= \frac{a}{b} \\[8pt]
			\implies m &= \frac{a^2}{b^2} \\[8pt]
			\iff mb^2 &= a^2 \\ 
			\end{align*}
			But $a$ and $b$ are coprime so they don't share any prime factors. This means that $a^2$ and $b^2$ also don't share any prime factors. So, if $\abs{b} > 1$, the prime factorization of $mb^2$ is necessarily different from that of $a^2$ meaning that $mb^2 \neq a^2$ contradicting the hypothesis of coprimality.
			On the other hand, if $\abs{b} = 1$, then $b$ has no prime factors (its prime factorization is empty) and so $mb^2$ has the same prime factorization as $m$ which may be equal to that of $a^2$ in the case that $m = a^2$.
			\end{proof}
			\bigskip
			
			\bigskip
			\labeledProposition{For all nonnegative integers $a > b$ the difference of squares $a^2 - b^2$ does not give a remainder of 2 when divided by 4.}{a_squared_minus_b_squared}
			Beginner's attempt - try proof by contradiction:
			\begin{align*}
			a^2 - b^2 &= 4n + 2 \\
			2k &= 4n + 2 &\sidecomment{by $a^2 - b^2$ even}\\
			k &= 2n + 1 \implies \text{ $k$ is some odd number.}
			\end{align*}
			So, proof by contradiction is our first instinct but doesn't seem to get us anywhere.
			Instead, proceed by cases:
			\paragraph{Case $a, b$ are even:}
			\begin{align*}
			\exists\,k,l \in \Z\; \cdot \;a &= 2k, b = 2l \\
			\implies a^2 - b^2 &= 4k^2 - 4l^2 \\
			&= 4\left(k^2 - l^2\right) \\
			&= 4m\; \text{ where } \;m \in \Z\;  \\
			\end{align*}
			So $4$ divides $a^2 - b^2$ with $0$ remainder.
			\paragraph{Case $a, b$ are odd:}
			\begin{align*}
			\exists\,k,l \in \Z\; \cdot \;a &= 2k+1, b = 2l+1 \\
			\implies a^2 - b^2 &= \left(4k^2 + 4k + 1\right) - \left(4l^2 + 4l + 1\right) \\
			&= 4\left(k^2 + k - l^2 - l\right) \\
			&= 4m\; \text{ where } \;m \in \Z\;  \\
			\end{align*}
			So, again, $4$ divides $a^2 - b^2$ with $0$ remainder.
			\paragraph{Case $a$ even, $b$ odd:}
			\begin{align*}
			\exists\,k,l \in \Z\; \cdot \;a &= 2k, b = 2l+1 \\
			\implies a^2 - b^2 &= 4k^2 - \left(4l^2 + 4l + 1\right) \\
			&= 4\left(k^2 - l^2 - l\right) - 1 \\
			&= 4m + 3\; \text{ where } \;m=k^2 - l^2 - l - 1 \in \Z\;  \\
			\end{align*}
			So, here, $4$ divides $a^2 - b^2$ with $3$ remainder. So the proposition is proven as we have proven all the possible cases.\\
			There is also another approach given in the Cambridge University Discrete Mathematics lecture notes,
			TODO
		}
		
		\searchableSubsection{Absolute Value}{absolute value}{\bigskip
			\[ \vert{x}\vert \ge x, \vert{y}\vert \ge y	\implies \vert{x}\vert + \vert{y}\vert \ge x + y \]
			\begin{align*}			
				\vert{x + y}\vert =
					\begin{cases} 
				      \vert{x}\vert + \vert{y}\vert & x,y \geq 0 \\
				      \vert{ -\vert{x}\vert + \vert{y}\vert }\vert & x < 0,y \geq 0 \\
				      \vert{ \vert{x}\vert - \vert{y}\vert }\vert & x \geq 0,y < 0 \\
				      \vert{ -(\vert{x}\vert + \vert{y}\vert) }\vert & x,y < 0
				   \end{cases}
				\iff
					\begin{cases} 
						\vert{ \vert{x}\vert + \vert{y}\vert }\vert & x,y \geq 0 \text{ or } x,y < 0) \\
						\vert{ \vert{x}\vert - \vert{y}\vert }\vert & x < 0,y \geq 0 \text{ or } x \geq 0,y < 0 \\
					\end{cases}
			\end{align*}
			\bigskip
			Clearly, $\vert{ \vert{x}\vert + \vert{y}\vert }\vert \ge \vert{ \vert{x}\vert - \vert{y}\vert }\vert$ so that,
			\[ \vert{x + y}\vert \leq \vert{ \vert{x}\vert + \vert{y}\vert }\vert = \vert{x}\vert + \vert{y}\vert \]
			and this is known as the "triangle inequality".\bigskip
			
			\labeledProposition{$\vert{x - y}\vert \leq \vert{x - z}\vert + \vert{y - z}\vert$}{asdfa}
			\begin{proof}
			\begin{align*}
				\vert{x - y}\vert = \vert{(x - z) + (z - y)}\vert \leq \vert{x - z}\vert + \vert{z - y}\vert = \vert{x - z}\vert + \vert{y - z}\vert
			\end{align*}
			\end{proof}
			
			\labeledProposition{$\vert{x - y}\vert \geq \vert{ \vert{x}\vert - \vert{y}\vert }\vert$}{asdfa}
			\begin{proof}
			Need to show $-\vert{x - y}\vert \leq \vert{x}\vert - \vert{y}\vert \leq \vert{x - y}\vert$. So, prove as two separate inequalities:
			\begin{align*}				
				&&\vert{y}\vert = \vert{x + (y - x)}\vert &\leq \vert{x}\vert + \vert{y - x}\vert \\
				&\iff &-\vert{y - x}\vert = -\vert{x - y}\vert &\leq \vert{x}\vert - \vert{y}\vert \\
			\end{align*}
			\begin{align*}				
				&&\vert{x}\vert = \vert{(x - y) + y}\vert &\leq \vert{x - y}\vert + \vert{y}\vert \\
				&\iff &\vert{x}\vert - \vert{y}\vert &\leq \vert{x - y}\vert
			\end{align*}
			\end{proof}
		}
		
		
\end{document}
