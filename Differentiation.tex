\documentclass[MathsNotesBase.tex]{subfiles}




\date{\vspace{-6ex}}


\begin{document}	
	\searchableSection{Differentiation}{analysis, differentiation}
	\bigskip\bigskip
	
	\searchableSubsection{\texorpdfstring{Differentiation as a Linear\\Transformation}{Differentiation as a Linear Transformation}}{analysis, differentiation}{\bigskip\bigskip
		
		\newcommand{\vtheta}{\V{\theta}}
		\newcommand{\valpha}{\V{\alpha}}
		\notation{The derivative of $y(x)$ with respect to $x$ will be denoted $\Dif_x y$ and of $f(x)$ with $\Dif_x f$.}
		\boxeddefinition{The set ${ P_n \subset \R{\R{}} }$ of all univariate real-valued polynomials ${ p: \R{} \longmapsto \R{} }$ of degree $n$ is defined as,
			\[ P_n = \setc{p \in \R{\R{}}}{p(x) = \sum_{i=0}^n \theta_i x^i, \hspace{10pt} \theta_i \in \R{}}. \]
			Or alternatively, in vector notation,
			\[ P_n = \setc{p \in \R{\R{}}}{p(x) = \vtheta^T \x, \hspace{10pt} \vtheta \in \R{n+1}} \]
			where $\x$ is the standard basis of the space of degree-$n$ polynomials,
			\[ (1,x,\dots,x^n)^T. \]
		}
		
		\bigskip
		\subsubsection{Linear Algebra of First-order Differential Equations}
		\bigskip
		The derivative $\Dif_x p(x)$, of the univariate degree-$n$ polynomial ${ p \in P_n }$ can be described as a linear transformation as,
			\[ \Dif_x p(x) = (A\vtheta)^T \x = \x^T A\vtheta \]
			where $A$ is the ${ n \times (n+1) }$ matrix,
			\[ A = \begin{bmatrix}
					0 & 1 & 0 & \cdots & 0\\
					0 & 0 & 2 & \cdots & 0\\
					\vdots &  &  &  & \vdots\\
					0 & 0 & 0 & \cdots & n
					\end{bmatrix}.
			\]
		The matrix $A$ clearly has rank $n$ and a 1-dimensional kernel so $\Dif_x$ transforms from $n+1$-space to $n$-space. The nullspace of $A$ being,
		\[ nullspace(A) = t\begin{bmatrix}1\\0\\\vdots\\0\end{bmatrix} \hspace{20pt} t \in \R{} \]
		the interpretation of which is that the derivative of constant polynomials is zero.\\\\
		
%		When we have a differential equation such as,
%		\begin{equation}\label{eq:first-order-lin-diffeq}
%			\Dif_x y(x) = ax + b \hspace{20pt} a,b \in R{}
%		\end{equation}	
%		we are trying to determine the function $y(x)$ from its derivative ${ ax + b }$ but, as we can see from the matrix $A$, the derivative transformation is singular and not invertible. This is why, when we find the antiderivative of a function through the process of integration, the result is a family of functions parameterized with the constant of integration. Expressing equation \ref{eq:first-order-lin-diffeq} in matrix form,
%		\begin{equation}
%			\x^T A\vtheta = \x^T (b,a,0,\dots,0)^T
%		\end{equation}
%		Because the members of $\x$ are linearly independent, this equation implies that the coefficients are equal,
%		\[ A\vtheta = (b,a,0,\dots,0)^T. \]
		
		\medskip
		In general the differential equation,
		\begin{equation}
			\Dif_x y(x) = \alpha_0 + \alpha_1 x + \cdots + \alpha_{n-1} x^{n-1}
		\end{equation}
		has the vector form,
		\begin{equation}\label{eq:first-order-lin-diffeq}
			\x^T A\vtheta = \x^T \valpha \iff A\vtheta = \valpha
		\end{equation}
		where the solution is
		\[ y(x) = \theta_0 + \theta_1 x + \cdots + \theta_n x^n = \x^T \vtheta. \]
		\note{Note that we can say
			\[ \x^T A\vtheta = \x^T \valpha \iff A\vtheta = \valpha \]
			because $\x^T$ is a basis and therefore a linearly independent set. By linear independence, the equation on the left implies that the coefficients are equal.\\\\
			This is equivalent to analysing $P_n$, the ${ (n+1) }$-dimensional vector space of polynomials, by working in the coordinate space formed by the coefficients. This space is $\R{n+1}$ which is isomorphic to any ${ (n+1) }$-dimensional vector space by \autoref{prop:vector_space_isomorphic_to_coordinate_space_of_same_dimension}. 
		}
		
		\smallskip
		The matrix form of \autoref{eq:first-order-lin-diffeq} is,
		\begin{equation}\label{eq:matrix-general-first-order-lin-diff-eq}
		\begin{bmatrix}
		0 & 1 & 0 & \cdots & \cdots & 0\\
		0 & 0 & 2 & \cdots & \cdots & 0\\
		\vdots &  &  &  &  & \vdots\\
		0 & 0 & 0 & \cdots & n-1 & 0\\
		0 & 0 & 0 & \cdots & \cdots & n
		\end{bmatrix}\begin{bmatrix}\theta_0\\\theta_1\\\vdots\\\vdots\\\vdots\\\theta_n\end{bmatrix} = \begin{bmatrix}\alpha_0\\\alpha_1\\\alpha_2\\\vdots\\\alpha_{n-1}\end{bmatrix}.
		\end{equation}
		Constructing the augmented matrix and using Gaussian Elimination we obtain,
		\begin{align*}
		&\begin{amatrix}{6}
		0 & 1 & 0 & \cdots & \cdots & 0 & \alpha_0\\
		0 & 0 & 2 & \cdots & \cdots & 0 & \alpha_1\\
		\vdots &  &  &  &  &  & \vdots\\
		0 & 0 & 0 & \cdots & n-1 & 0 & \alpha_{n-2}\\
		0 & 0 & 0 & \cdots & \cdots & n & \alpha_{n-1}
		\end{amatrix} \\[20pt]
		\leadsto &\begin{amatrix}{6}
		0 & 1 & 0 & \cdots & \cdots & 0 & \alpha_0\\
		0 & 0 & 1 & \cdots & \cdots & 0 & \alpha_1/2\\
		\vdots &  &  &  &  &  & \vdots\\
		0 & 0 & 0 & \cdots & 1 & 0 & \alpha_{n-2}/n-1\\
		0 & 0 & 0 & \cdots & \cdots & 1 & \alpha_{n-1}/n
		\end{amatrix}
		\end{align*}
		so that $\theta_0$ is a free variable and we can read off the other values of the coefficients $\theta_i$ corresponding to the columns of the matrix as follows:
		\begin{equation}\label{eq:matrix-general-first-order-lin-diff-eq-soln}
			\begin{bmatrix}\theta_0\\\theta_1\\\theta_2\\\vdots\\\theta_{n-1}\\\theta_n\end{bmatrix} =
			\begin{bmatrix}t\\\alpha_0\\\alpha_1/2\\\vdots\\\alpha_{n-2}/n-1\\\alpha_{n-1}/n\end{bmatrix} \hspace{20pt} \text{for } t \in \R{}
		\end{equation}
		and this implies that the solution to the differential equation is:
		\begin{equation}\label{eq:general-first-order-lin-diff-eq-soln}
			y(x) = t + \alpha_0 x + \frac{\alpha_1}{2} x^2 + \cdots + \frac{\alpha_{n-1}}{n} x^n, \hspace{20pt} t \in \R{}.
		\end{equation}
		We can rewrite the result in \autoref{eq:matrix-general-first-order-lin-diff-eq-soln} as,
		\begin{equation}
			\begin{bmatrix}0\\\alpha_0\\\alpha_1/2\\\vdots\\\alpha_{n-2}/n-1\\\alpha_{n-1}/n\end{bmatrix} + t\begin{bmatrix}1\\0\\0\\\vdots\\0\\0\end{bmatrix}
			\hspace{20pt} \text{for } t \in \R{}
		\end{equation}
		which makes it clear that we have a particular solution (with ${ t = 0 }$) plus a 1-dimensional nullspace. However, in practical applications modeled by differential equations, there will typically be an initial condition --- an initial value of $y$ such as $y(0)$ for example --- and this will be used to determine a particular solution. In this situation (known as IVP or Initial Value Problems) the particular solution involves finding a value of the parameter $t$ that fits the initial condition. For this reason, the solution in \autoref{eq:general-first-order-lin-diff-eq-soln} is referred to as the general solution of the differential equation while the particular solution has a particular value of the parameter $t$.\\\\
		
		\bigskip
		\subsubsection{Linear Algebra of Second-order Differential Equations}
		\bigskip
		The most simple type of second-order linear differential equation looks like,
		\begin{equation}\label{eq:eq:second-order-lin-diffeq}
			 \Dif_x^2 y(x) = \alpha_0 + \alpha_1 x + \cdots + \alpha_{n-2} x^{n-2}.
		\end{equation}
		We can add an extra row of zeros to the matrix of $\Dif_x$ to make it square so that we can take powers of it,
		\[ A = \begin{bmatrix}
				0 & 1 & 0 & \cdots & 0\\
				0 & 0 & 2 & \cdots & 0\\
				\vdots &  &  &  & \vdots\\
				0 & 0 & 0 & \cdots & n \\
				0 & 0 & 0 & \cdots & 0
				\end{bmatrix}
		\]
		and output an extra coefficient with value zero.\\
		
		Then the vector form of \autoref{eq:eq:second-order-lin-diffeq} is
		\begin{equation}
			A^2 \vtheta = \valpha
		\end{equation}
		and the matrix form is
		\begin{equation}
			\begin{bmatrix}
			0 & 0 & 2 & 0 & \cdots & 0\\
			0 & 0 & 0 & 6 & \cdots & 0\\
			\vdots &  &  &  &  & \vdots\\
			0 & 0 & 0 & 0 & \cdots & n(n-1)\\
			0 & 0 & 0 & 0 & \cdots & 0\\
			0 & 0 & 0 & 0 & \cdots & 0
			\end{bmatrix}\begin{bmatrix}\theta_0\\\theta_1\\\vdots\\\vdots\\\vdots\\\theta_n\end{bmatrix} = \begin{bmatrix}\alpha_0\\\alpha_1\\\vdots\\\alpha_{n-2}\\0\\0\end{bmatrix}
		\end{equation}
		which gives the solution to the differential equation as:
		\begin{equation}\label{eq:general-first-order-lin-diff-eq-soln}
		y(x) = s + tx + \frac{\alpha_0}{2} x^2 + \frac{\alpha_1}{6} x^3 + \cdots + \frac{\alpha_{n-2}}{n(n-1)} x^n, \hspace{20pt} s,t \in \R{}.
		\end{equation}
		So, here the kernel is 2-dimensional and we need two initial values to determine a particular solution.
		
		\bigskip
		\subsubsection{Eigenvectors of the Differentiation Operator}\label{sss:eigenvectors-of-differentiation}
		\bigskip
		If we had the differential equation,
		\begin{equation}
		\Dif_x y(x) = \alpha_0 + \alpha_1 x + \alpha_2 x^2 + \alpha_3 x^3 + \cdots
		\end{equation}
		so that the derivative of $y(x)$ is an infinite power series (not an infinite polynomial as polynomials are by definition finite), then the general solution would take the form, for ${ t \in \R{} }$,
		\begin{equation}
			y(x) = t + \alpha_0 x + \frac{\alpha_1}{2} x^2 + \frac{\alpha_2}{3} x^3 + \cdots
		\end{equation}
		So, if we had
		\[ \alpha_0 = \alpha_1,\, \frac{\alpha_1}{2} = \alpha_2,\, \frac{\alpha_2}{3} = \alpha_3 \]
		then the general solution would be, for ${ t \in \R{} }$,
		\begin{equation}
			y(x) = t + \alpha_1 x + \alpha_2 x^2 + \alpha_3 x^3 + \cdots
		\end{equation}
		which is a family of functions which includes the derivative ${ \Dif_x y(x) }$ --- the derivative being the member of this set of functions with ${ t = \alpha_0 }$.\\
		
		\note{Note that, for polynomials, the derivative is a transformation between finite-dimensional vector spaces and so the Dimension Formula (\autoref{theo:linear_map_dimension_formula}) of linear transformations applies. We can see this in that the derivative has a one-dimensional kernel (the set of constant polynomials) and the derivative maps from $P_n$ to $P_{n-1}$, the image having one less dimension then the domain space because there is a one-dimensional kernel.\\
			However, the coordinate space of the power series above is isomorphic to an infinite-dimensional vector space where the Dimension Formula no longer applies (see \autoref{theo:no-dimension-formula-for-infinite-vector-spaces}). So when we differentiate it we get a result of the same dimensionality despite there being a non-trivial one-dimensional kernel.
		}
		
		In order to achieve this we need the coefficients to form the series,
		\begin{align*}
		&\alpha_0,\; \frac{\alpha_0}{1},\; \frac{\alpha_0}{1 \times 2},\; \frac{\alpha_0}{1 \times 2 \times 3},\; \dots
		\end{align*}
		so that
		\begin{align*}
		y(x) &= \alpha_0 + \frac{\alpha_0}{1}x + \frac{\alpha_0}{1 \times 2}x^2 + \frac{\alpha_0}{1 \times 2 \times 3}x^3 + \dots \\[8pt]
		&= \alpha_0 + \frac{\alpha_0}{1!}x + \frac{\alpha_0}{2!}x^2 + \frac{\alpha_0}{3!}x^3 + \dots &\sidecomment{} \\[8pt]
		&= \alpha_0\left(1 + \frac{x}{1!} + \frac{x^2}{2!} + \frac{x^3}{3!} + \dots \right) = \alpha_0 e^x.
		\end{align*}
		So, the general solution is,
		\[ y(x) = t e^x \hspace{20pt} t \in \R{} \]
		and the particular solution, for ${ t = \alpha_0 }$, is
		\[ y(x) = \alpha_0 e^x. \]
		
		\bigskip\bigskip
		\labeledTheorem{The derivative of $e^{f(x)}$ is ${ f'(x)e^{f(x)} }$ where $f'$ is the derivative of the function $f$.}{derivative-of-exp-function}
		\begin{proof}
			\begin{align*}
			&& e^{f(x)} &= 1 + \frac{f(x)}{1!} + \frac{(f(x))^2}{2!} + \frac{(f(x))^3}{3!} + \cdots \\[8pt]
			&\implies & \frac{\dif}{\dif x} e^{f(x)} &= f'(x) + f'(x)\frac{f(x)}{1!} + f'(x)\frac{(f(x))^2}{2!} + f'(x)\frac{(f(x))^3}{3!} + \cdots &\sidecomment{} \\[8pt]
			&&&= f'(x) e^{f(x)}.
			\end{align*}
		\end{proof}
		\begin{corollary}
			Any function of the form $Ae^{f(x)}$ is an eigenfunction of the differentiation operator with eigenvalue equal to $f'(x)$.
		\end{corollary}
		\begin{proof}
			\[ \frac{\dif}{\dif x} Ae^{f(x)} = A \frac{\dif}{\dif x} e^{f(x)} = A f'(x)e^{f(x)} = f'(x)(Ae^{f(x)}). \qedhere \]
		\end{proof}
		\begin{corollary}
			The set of functions $e^{\int f'(x) \dif x}$ form an eigenspace of the differentiation operator with eigenvalue $f'(x)$.
		\end{corollary}
	
		\bigskip
		\labeledProposition{For any ${ a,x \in \R{} }$, \[ \frac{\dif}{\dif x} a^x = (\ln{a}) a^x. \]}{derivative-of-a-to-power-x-is-ln-a-times-a-to-power-x}
		\begin{proof}
			\begin{align*}
			\frac{\dif}{\dif x}a^x &= \frac{\dif}{\dif x} e^{(\ln{a})x} &\sidecomment{by \autoref{prop:exp-x-is-inverse-of-ln}}\\[8pt]
			&= (\ln{a})e^{(\ln{a})x} &\sidecomment{by \autoref{theo:derivative-of-exp-function}} \\[8pt]
			&= (\ln{a})a^x &\sidecomment{by \autoref{prop:exp-x-is-inverse-of-ln}.} \qedhere\\[8pt]
			\end{align*}
		\end{proof}
		
	}



% ------------


	\pagebreak	
	\searchableSection{Homogeneous Functions}{analysis, differentiation, homogeneous functions}{\label{section:homogeneous-functions}
	\bigskip\bigskip

		\boxeddefinition{A \textbf{homogeneous} function is a multivariate function $f(x_1, \dots, x_n)$ such that,
			\[ f(\lambda x_1, \dots, \lambda x_n) = \lambda^d f(x_1,\dots, x_n) \hspace{20pt} d \in \Z{},\, \lambda \in \R{}. \]
			The integer power $d$ is known as the \textbf{degree} so that $f$ is described as \textbf{homogeneous of degree $d$}.
		}
	
		\bigskip
		\begin{lemma}\label{lem:homogeneous-func-has-powers-that-sum-to-degree}
			If a function $f$ is homogeneous of degree $d$ then it can be expressed as a polynomial whose terms contain powers of variables such that the powers sum to $d$.
		\end{lemma}
		\begin{proof}
			Assume $f$ can be expressed as a multivariate polynomial. Then $f(x_1, \dots, x_n)$ can be expressed as the sum of terms of the form,
			\[ \alpha x_1^{i_1} \cdots x_n^{i_n} \]
			for some constant ${ \alpha \in \R{} }$. Therefore, for each term of $f(\lambda x_1, \dots, \lambda x_n)$ we have,
			\[ \alpha (\lambda x_1)^{i_1} \cdots (\lambda x_n)^{i_n} = \alpha \lambda^{i_1} x_1^{i_1} \cdots \lambda^{i_n} x_n^{i_n} = \lambda^{(i_1 + \cdots + i_n)} (\alpha x_1^{i_1} \cdots x_n^{i_n}). \]
			Since $f$ is homogeneous of degree $d$ we also have,
			\[ f(\lambda x_1, \dots, \lambda x_n) = \lambda^d f(x_1,\dots, x_n) \]
			where each term of $\lambda^d f(x_1,\dots, x_n)$ has the form,
			\[ \lambda^d (\alpha x_1^{i_1} \cdots x_n^{i_n}) \]
			which means that, for every term of the expression for $f(\lambda x_1, \dots, \lambda x_n)$, we must have,
			\[ \lambda^{(i_1 + \cdots + i_n)} = \lambda^d \iff i_1 + \cdots + i_n = d. \qedhere \]
		\end{proof}
		
		\bigskip
		\labeledProposition{\textbf{Euler's Theorem of Homogeneous Functions}: If $f(x_1, \dots, x_n)$ is a homogeneous function of degree $d$ then,
			\[ d \cdot f(x_1, \dots, x_n) = x_1 \frac{\partial f}{\partial x_1} + \cdots + x_n \frac{\partial f}{\partial x_n}. \]
		}{eulers-theorem-of-homogeneous-functions}
		\begin{proof}
			Assume $f$ can be expressed as a multivariate polynomial. Then $f(x_1, \dots, x_n)$ can be expressed as the sum of terms of the form,
			\[ \alpha x_1^{i_1} \cdots x_n^{i_n} \]
			for some constant ${ \alpha \in \R{} }$. If we take the partial derivative of $f$ with respect to $x_1$, each term of the result will take the form,
			\[ i_1 \alpha x_1^{(i_1 - 1)}x_2^{i_2}\cdots x_n^{i_n} \]
			and each term of the partial derivative with respect to $x_2$ will have the form,
			\[ i_2 \alpha x_1^{i_1}x_2^{(i_2 - 1)}\cdots x_n^{i_n} \]
			and the $n$-th partial derivative will have terms,
			\[ i_n \alpha x_1^{i_1}x_2^{i_2}\cdots x_n^{(i_n - 1)}. \]
			Now if we look at the terms of the expression $x_1f_{x_1}$,
			\[ x_1 \cdot i_1 \alpha x_1^{(i_1 - 1)}x_2^{i_2}\cdots x_n^{i_n} = i_1 (\alpha x_1^{i_1}x_2^{i_2}\cdots x_n^{i_n}) \]
			we see that the terms are the same as the terms of the original function $f(x_1, \dots, x_n)$ except multiplied by the power of $x_1$ in that term. Therefore, each term of the expression ${ x_1f_{x_1} + \cdots + x_nf_{x_n} }$ has the form,
			\[ (i_1 + i_2 + \cdots + i_n)(\alpha x_1^{i_1}x_2^{i_2}\cdots x_n^{i_n}) = d(\alpha x_1^{i_1}x_2^{i_2}\cdots x_n^{i_n}) \]
			where we have used \autoref{lem:homogeneous-func-has-powers-that-sum-to-degree} to determine that,
			\[ i_1 + \cdots + i_n = d. \]
			Therefore,
			\[ x_1f_{x_1} + \cdots + x_nf_{x_n} = d \cdot f(x_1, \dots, x_n). \qedhere \]
		\end{proof}
	}
\end{document}