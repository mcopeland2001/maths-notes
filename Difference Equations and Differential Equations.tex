\documentclass[MathsNotesBase.tex]{subfiles}

\date{\vspace{-6ex}}

\begin{document}
	\searchableSection{\texorpdfstring{Difference and Differential\\ Equations}{Difference and Differential Equations}}{differential equations, difference equations}
	\bigskip\bigskip
	
	\searchableSubsection{Difference Equations}{difference equations}{
		\bigskip\bigskip
		\note{A \textbf{difference equation} is also known as a \textbf{recurrence equation}.}
		
		\bigskip
		\subsubsection{First-order Difference Equations}
		\boxeddefinition{Let ${ y_t }$ be the $t$-th value in a sequence (typically $t$ represents time). Then,
			\[ y_t = ay_{t-1} + b, \hspace{20pt} t \geq 1 \]
			is called a \textbf{first-order linear difference equation with constant coefficients}. The value $y_0$ is called an initial condition.\\
			A solution to such an equation is an explicit expression for $y$ in terms of $t$ and $y_0$.\\
			If ${ b = 0 }$ we have,
			\[ y_t = ay_{t-1} \iff y_t - ay_{t-1} = 0 \]
			which is known as a \textbf{homogeneous} first-order linear difference equation with constant coefficients.
		}
	
		\bigskip
		\labeledProposition{A first-order linear difference equation with constant coefficients of the form ${ y_t = ay_{t-1} + b }$ where ${ a = 1 }$ is a arithmetic progression.
		}{first-order-lin-diff-eqn-const-coeff-1-multiplier-is-arith-prog}
		\begin{proof}
			Let ${ y_t = y_{t-1} + b }$ then,
			\[ y_1 = y_0 + b,\, y_2 = y_1 + b = (y_0 + b) + b = y_0 + 2b,\, y_3 = y_0 + 3b, \dots \]
			so we have ${ y_t = y_0 + tb }$. If we describe this as an arithmetic progression we have,
			\[ x_n = a + nd \]
			where the zeroth term $a$ corresponds to $y_0$, the common difference $d$ corresponds to $b$ and, clearly, $t$ and $n$ are both term indices.
		\end{proof}
		
		\medskip
		\labeledProposition{A first-order linear difference equation with constant coefficients of the form ${ y_t = ay_{t-1} + b }$ where ${ b = 0 }$ is a geometric progression.}{first-order-lin-diff-eqn-const-coeff-0-const-term-is-geom-prog}
		\begin{proof}
			Let ${ y_t = ay_{t-1} + 0 }$ then,
			\[ y_1 = ay_0,\, y_2 = ay_1 = a(ay_0) = a^2y_0,\, y_3 = a^3y_0, \dots \]
			so we have ${ y_t = a^ty_0 }$. If we describe this as a geometric progression we have,
			\[ x_n = ar^n \]
			where the zeroth term $a$ corresponds to $y_0$, the common ratio $r$ corresponds to $a$, and $n$ is the term index corresponding to $t$.
		\end{proof}
	
		\bigskip
		\subsubsection{Examples of first-order linear difference equations w/ const. coefficients}
		\begin{exe}
			\item{Let $y_t$ be an account balance after $t$ years and $r$ be the annual interest rate paid on the account. Suppose also, that each year $I$ is withdrawn from this account. Then the formula for $y_t$ is,
				\[ y_t = (1 + r)y_{t-1} - I. \]
			}
		\end{exe}
	}



	\bigskip\bigskip
	\searchableSubsection{Differential Equations}{differential equations}{
		\bigskip\bigskip
		
		\subsubsection{Linear Ordinary Differential Equations}
		\bigskip
		
		\subsubsubsection{First Order}
		\begin{align*}
		&& \frac{dy}{dx} &= ax + b \\
		&\iff & \int \frac{dy}{dx} \dx &= \int ax + b \dx &\sidecomment{} \\
		&\iff & y &= a'x^2 + bx + c. &\sidecomment{${ a' = a/2}$, c is any constant}
		\end{align*}
		Integrating we see that a first order differential equation only determines the function upto a constant value. In order to determine a specific function we need a relation between a value of $x$ and a value of $y$ (i.e. a point, in graphical terms). Typically this is described as an initial condition.
		
		\bigskip\bigskip
		\subsubsubsection{Second Order}
		\begin{align*}
		&& \frac{d^2y}{dx^2} &= ax + b \\
		&\iff & \int \frac{d^2y}{dx^2} \dx &= \int ax + b \dx  &\sidecomment{} \\
		&\iff & \frac{dy}{dx} &= a'x^2 + bx + c  &\sidecomment{${ a' = a/2 }$, c is any constant} \\
		&\iff & \int \frac{dy}{dx} \dx &= \int a'x^2 + bx + c \dx  &\sidecomment{} \\
		&\iff & y &= a''x^3 + b'x^2 + cx + d. &\sidecomment{${ a'' = a/6,\, b' = b/2}$, d is any constant}
		\end{align*}
		Integrating a second order equation twice we see that we introduced two constants of integration, ${ c,d }$, and the last two terms ${ cx + d }$ are undetermined. So a second order equation of this type has only determined a function upto a first-degree polynomial (a line). To determine a specific function, in this case, we require two relations between $x$ and $y$ (two points determine a line).
		
		\bigskip\bigskip
		\subsubsection{Recurrence Differential Equations}
		\bigskip
		\boxeddefinition{This is \textbf{not} standard terminology but what is meant here by \textbf{recurrence} differential equations is equations that describe the derivative of a function --- say $y(x)$ --- in terms of the value of the function itself. So, for example, a linear first-order example would have the form,
			\[ \frac{\dif y}{\dif x} = f(x)y + g(x). \]
		}
	
		Since the eigenfunction of differentiation is the exponent function ${ f(x) = e^x }$ (see \autoref{sss:eigenvectors-of-differentiation}), the solution to these differential equations will always involve the exponent function.
		
		\bigskip\bigskip
		\subsubsubsection{The form ${ \frac{\dif y}{\dif x} = f(x)y }$}
		The most simple form has a single $y$-term whose coefficient may be a function of $x$. This form is separable as,
		\begin{align*}
		&& \frac{\dif y}{\dif x} &= f(x)y \\[8pt]
		&\iff & \frac{1}{y} \frac{\dif y}{\dif x} &= f(x) &\sidecomment{} \\[8pt]
		&\iff & \int \frac{1}{y} \frac{\dif y}{\dif x} \dif x &= \int f(x) \dif x &\sidecomment{} \\[8pt]
		&\iff & \ln \abs{y} &= F(x) + c &\sidecomment{${ y \neq 0 }$, $F$ is an antiderivative of $f$} \\[8pt]
		&\iff & \abs{y} &= e^{F(x)}\cdot e^c &\sidecomment{} \\[8pt]
		&\iff & y &= Ae^{F(x)} &\sidecomment{${ A \in \R{}. }$}
		\end{align*}
		Check solution:
		\[ \frac{\dif y}{\dif x} = f(x)Ae^{F(x)} = f(x)y. \]
		\note{Note that the solution has the form,
			\[ y = Ae^{F(x)} \]
			where $F(x)$ is an antiderivative of $f(x)$, the coefficient of $y$ in the original differential equation. Since the antiderivative is unique upto a constant factor, the other possible antiderivatives are achieved by the value of the coefficient $A$ because,
			\[ e^{F(x) + c} = e^{F(x)}\cdot e^c = Ae^{F(x)}. \]
		}
		
		\bigskip\bigskip
		\subsubsubsection{The form ${ \frac{\dif y}{\dif x} = f(x)y + g(x) }$}
		This form is not separable as it is. But if we multiply both sides by $e^{-F(x)}$, where $F(x)$ is an antiderivative of $f(x)$, then
		\begin{align*}
		&& \frac{\dif y}{\dif x} &= f(x)y + g(x) \\[8pt]
		&\iff & e^{-F(x)}\frac{\dif y}{\dif x} &= e^{-F(x)}f(x)y + e^{-F(x)}g(x) &\sidecomment{} \\[8pt]
		&\iff & e^{-F(x)}\frac{\dif y}{\dif x} - e^{-F(x)}f(x)y &= e^{-F(x)}g(x) &\sidecomment{} \\[8pt]
		&\iff & \frac{\dif}{\dif x} \left(e^{-F(x)}y\right) &= e^{-F(x)}g(x) &\sidecomment{} \\[8pt]
		&\iff & \int \frac{\dif}{\dif x} \left(e^{-F(x)}y\right) \dif x &= \int e^{-F(x)}g(x) \dif x &\sidecomment{} \\[8pt]
		&\iff & e^{-F(x)}y + c &= \int e^{-F(x)}g(x) \dif x &\sidecomment{constant ${ c \in \R{} }$} \\[8pt]
		&\iff & e^{-F(x)}y &= \int e^{-F(x)}g(x) \dif x - c &\sidecomment{} \\[8pt]
		&\iff & y &= e^{F(x)}\int \frac{g(x)}{e^{F(x)}} \dif x + Ae^{F(x)}. &\sidecomment{${ A = -c }$}
		\end{align*}
		Check solution:
		\begin{align*}
		\frac{\dif y}{\dif x} &= f(x)\left(e^{F(x)}\int \frac{g(x)}{e^{F(x)}} \dif x\right) + e^{F(x)}\frac{g(x)}{e^{F(x)}} + f(x)Ae^{F(x)} \\[8pt]
		&= f(x)\left(e^{F(x)}\int \frac{g(x)}{e^{F(x)}} \dif x + Ae^{F(x)}\right) + g(x) &\sidecomment{} \\[8pt]
		&= f(x)y + g(x).
		\end{align*}
		\note{Note that
			\[ Ae^{F(x)}\int \frac{g(x)}{Ae^{F(x)}} \dif x = Ae^{F(x)}\frac{1}{A}\int \frac{g(x)}{e^{F(x)}} \dif x = e^{F(x)}\int \frac{g(x)}{e^{F(x)}} \dif x. \]
		}
	
		\paragraph{Example:}
		\begin{exe}
			\item{${\bm{ x\frac{\dif y}{\dif x} - 2y = 6 }}$:\\\\
				The first step is to rearrange it to obtain an expression for the derivative. 
				\begin{align*}
				&& x\frac{\dif y}{\dif x} - 2y &= 6 \\
				&\iff & \frac{\dif y}{\dif x} &= \frac{2}{x}y + \frac{6}{x}. &\sidecomment{} \\
				\end{align*}
				Then we can apply the derived formula using the antiderivative ${ F(x) = 2\ln{x} }$.
				\begin{align*}
				y &= e^{2\ln{x}}\int \frac{6/x}{e^{2\ln{x}}} \dif x \\
				&= 6x^2 \int \frac{1}{x^3} \dif x &\sidecomment{} \\
				&= 6x^2\left(-\frac{1}{2x^2} + c\right) &\sidecomment{} \\
				&= -3 + c'x^2. &\sidecomment{${ c' = 6c }$} \\
				\end{align*}
				We can confirm this result by performing the calculation.
				\begin{align*}
				&& e^{-2\ln{x}}\frac{\dif y}{\dif x} &= e^{-2\ln{x}}\frac{2}{x}y + e^{-2\ln{x}}\frac{6}{x} \\
				&\iff & x^{-2}\frac{\dif y}{\dif x} - x^{-2}\frac{2}{x}y &= x^{-2}\frac{6}{x} &\sidecomment{} \\
				&\iff & x^{-2}\frac{\dif y}{\dif x} - \frac{2}{x^3}y &= \frac{6}{x^3} &\sidecomment{} \\
				&\iff & \frac{\dif }{\dif x}(x^{-2}y) &= \frac{6}{x^3} &\sidecomment{} \\
				&\iff & \int \frac{\dif }{\dif x}(x^{-2}y) \dif x &= 6 \int \frac{1}{x^3} \dif x &\sidecomment{} \\
				&\iff & x^{-2}y &= 6(-\frac{1}{2x^2} + c) &\sidecomment{} \\
				&\iff & x^{-2}y &= -3\frac{1}{x^2} + c' &\sidecomment{${ c' = 6c }$} \\
				&\iff & y &= -3 + c'x^2. &\sidecomment{} \\
				\end{align*}
			}
		\end{exe}
	
		\bigskip\bigskip
		\subsubsubsection{Homogeneous Differential Equations}
		\boxeddefinition{A \textbf{homogeneous differential equation} is an equation of the form,
			\[ f(x,y)\frac{\dif y}{\dif x} = g(x,y) \]
			where the functions $f,g$ are both homogeneous of degree $d$.
		}
	
		\bigskip\note{The functions $f,g$ need not be linear and so these differential equations are not necessarily linear.}
		
		The key insight here is that, due to the homogeneous function property (see: \ref{section:homogeneous-functions}), if we define the function $y$ to be ${ y(x) = x \cdot v(x) }$ --- which we may do because, due to the non-trivial kernel of differentiation, we are only able to determine a solution to a differential equation upto a constant term so we can set the constant term to 0 for convenience during the calculation --- then,
		\[ f(\lambda x, \lambda y) = \lambda^d f(x,y) \implies f(x,xv) = x^d g(v). \]
		When this is applied in a differential equation of the above form we obtain,
		\begin{align*}
		&& f(x,y)\frac{\dif y}{\dif x} &= g(x,y) \\[8pt]
		&\iff & x^d f_1(v) \left(v + x\frac{\dif v}{\dif x}\right) &= x^d f_2(v) &\sidecomment{${y = xv,\,\, y' = v + xv'}$} \\[8pt]
		&\iff & f_1(v) \left(v + x\frac{\dif v}{\dif x}\right) &= f_2(v) &\sidecomment{} \\[8pt]
		&\iff & v + x\frac{\dif v}{\dif x} &= \frac{f_2(v)}{f_1(v)} = f_3(v) &\sidecomment{} \\[8pt]
		&\iff & x\frac{\dif v}{\dif x} &= f_3(v) - v = f_4(v) &\sidecomment{} \\[8pt]
		&\iff & \frac{1}{f_4(v)}\frac{\dif v}{\dif x} &= \frac{1}{x} &\sidecomment{} \\[8pt]
		&\iff & \int \frac{1}{f_4(v)}\frac{\dif v}{\dif x} \dif x &= \int \frac{1}{x} \dif x = \ln{\abs{x}} + c &\sidecomment{} \\[8pt]
		&\iff & \int \frac{1}{f_4(v)} \dif v &= \ln{\abs{x}} + c &\sidecomment{} \\[8pt]
		&\iff & \frac{1}{f_4'(v)} \ln{\abs{f_4(v)}} &= \ln{\abs{x}} + c. &\sidecomment{} \\[8pt]
		\end{align*}

	}

\end{document}