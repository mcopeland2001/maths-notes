\documentclass[MathsNotesBase.tex]{subfiles}




\date{\vspace{-6ex}}


\begin{document}
\searchableSubsection{\chapterTitle{Algebra}}{abstract algebra, complex numbers}{\bigskip\bigskip}

	\searchableSubsection{\sectionTitle{Abstract Algebra}}{abstract algebra}{\bigskip}
	
	\bigskip\bigskip
	\searchableSubsection{Groups}{abstract algebra}{\bigskip
		
		\begin{definition}
		A binary operation is a function,\\
		\[ f:G \times G \mapsto G \]
		which - by the definition of a function - maps a unique tuple from $G \times G$ to a unique value in the codomain $G$.
		\end{definition}
		\bigskip
		
		\begin{definition}
		Let $G$ be a set and $*$ a binary operation on $G$ and denote this ($G,*$). Then ($G,*$) is a \textbf{group} if:
		\paragraph*{closure} $\forall x,y \in G, x * y \in G$;
		\paragraph*{associativity} $\forall x,y,z \in G, (x * y) * z = x * (y * z)$;
		\paragraph*{identity} $\exists e \in G \suchthat \forall x \in G, e*x = x*e = x$;
		\paragraph*{inverse} $\forall x \in G, \exists x^{-1} \in G \suchthat x * x^{-1} = x^{-1} * x = e$.
		\end{definition}
		These are known as the \textit{group axioms}.
		
		\bigskip\smallskip
		\begin{definition}
		The group is an \textbf{Abelian group} if it has the additional property:	
		\paragraph*{commutativity} $\forall x,y \in G, x * y = y * x \in G$.
		\end{definition}
		
		\bigskip\smallskip
		\paragraph{Notation} from here on we will use juxtaposition notation for the group operation (so $xy = x * y$) and (usually) $1$ for the identity element instead of $e$. This is known as \textit{multiplicative notation}.
		
		\bigskip\bigskip
		\subsubsection{Corollaries of the group axioms}
		The group operation is defined to map a unique tuple in $G \times G$ to a unique value in $G$ so that if we have $x,y \in G$ then $f((x, y)) = f(x,y) = xy \in G$ and for $a,b,c \in G$,
		\[ a = b \iff (c, a) = (c, b) \implies f((c, a)) = f((c, b)) \iff ca = cb \]
		\[ \therefore a = b \implies ca = cb \]
		
		Then, using all the group axioms - associativity, inverse and identity,
		\[ ca = cb \implies \inv{c}(ca) = \inv{c}(cb) \iff (\inv{c}c)a = (\inv{c}c)b \iff 1a = 1b \iff a = b \]
		
		Therefore we have the principle of cancellation,
		\[ ca = cb \implies a = b \]
		\textit{Note} that, since we have used the axioms of inverse and identity and the definitions of these require these elements to exhibit these properties from both the left and the right, the principle of cancellation can also be shown from both the left and the right. So, also,
		\[ ac = bc \implies a = b \]
		
		\bigskip\bigskip
		There are (at least) two approaches to finding the other consequences of the group axioms.
		\paragraph*{First approach.} We begin by noticing that the law of cancellation implies that,
		\subparagraph*{unique identity and inverses}$\forall a,x,b \in G, ax = b$ has a unique solution because,
		\[ ax = ax' \iff x = x' \]
		That unique solution is $a^{-1}b$. If $b=a$ we have $ax=a$ and $x$, by identity axiom, is an identity element. Since, the solution to this equation - $x$ - is unique, it follows that there is a unique value that is the identity element. Then, if we let $b$ be this unique identity element we have $ax=1$ and the unique solution, $x$, is the inverse of a, i.e. $\inv{a}$. Therefore, the inverses of group elements are also unique.
		
		\paragraph*{Second approach.} This approach begins by showing the uniqueness of the identity element solely using the defintion of the identity. Here, for clarity, we revert to using $e$ to denote the identity element.
		\subparagraph*{unique identity} Assume there are two identity elements, $e, e'$. Then, by the definition of the identity $ee' = e'e = e = e'$ so that there is a single value that has the property of the identity element.\\\\
		Then, using the definition of the inverse we have,
		\subparagraph{unique inverses} Assume there are two distinct inverses of an element $a$: $\inv{a}$ and $a'$. Then, 
		\begin{align*}
		&& a\inv{a} = 1 &= aa' &\sidecomment{defn. of inverse, uniqueness of identity}\\
		& \iff & \inv{a} &= a' &\sidecomment{law of cancellation}
		\end{align*}
		\bigskip\bigskip

		\paragraph*{Some Examples of Groups}
		\begin{itemize}
		\item{ ($\R{} \setminus \{0\}, \times)$ } is a group whereas ($\R{}, \times)$ is not a group because $0$ has no multiplicative inverse.
		\item{ ($\R{}, +)$ } is a group.
		\item{ The set of $n \times n$ invertible matrices is called the General Linear group and denoted $GL_n$  }
		\end{itemize}
			\subsubsection{Permutations and Symmetric Groups}\bigskip
		\begin{definition}
			A \textbf{permutation} is a bijection from a set to itself. Since permutations are bijective, they are invertible and since they are functions, function composition defines an associative law of composition over them. As a result, they form a group. 
		\end{definition}
		\begin{definition}
			The \textbf{symmetric group} defined over a set is the group whose elements are the permutations of the objects of the set and whose law of composition is the composition of functions.
		\end{definition} 
		\notation{The symmetric group over the integers from 1 to n is denoted $S_n$.}\\
		\subparagraph{$\bm{S_2}$}
		The symmetric group $S_2$ consists of the two elements $i$ and $\tau$ which are, respectively, the identity map and the transposition which interchanges 1 and 2. The group composition law is described by the fact that the identity map is the identity of the composition and by the relation $\tau\tau = \tau^2 = i$. Which results in the multiplication table:	
		\[
		\begin{aligned}
			i &\cdot i &&= i \\
			i &\cdot \tau &&= \tau \\		
			\tau &\cdot i &&= \tau \\
			\tau &\cdot \tau &&= i \\
		\end{aligned}
		\]
		Note that the law of composition is commutative.
		\subparagraph{$\bm{S_3}$}
		The symmetric group $S_3$ contains $3!$ elements. It is the smallest group whose law of composition is not commutative. It can be described using any two permutations of $\{1, 2, 3\}$. For example, if we take,
		\begin{align*}
			x =
			\begin{bmatrix}
			0 & 1 & 0 \\
			0 & 0 & 1 \\
			1 & 0 & 0 \\
			\end{bmatrix},\;
			y =
			\begin{bmatrix}
			0 & 1 & 0 \\
			1 & 0 & 0 \\
			0 & 0 & 1 \\
			\end{bmatrix}
		\end{align*}
		Then the permutations are,
		\[ \{1, x, x^2, y, xy, x^2y\} = \setc{x^iy^j}{0 \leq i \leq 2,\; 0 \leq j \leq 1} \]
	}
	
	
	
	
	
	\pagebreak
	\searchableSubsection{\sectionTitle{Fields}}{abstract algebra}{\bigskip}
		
		\bigskip
		\searchableSubsection{Complex Numbers}{abstract algebra, complex numbers}{\bigskip
			\labeledProposition{For every $\alpha \in \C{}$, there exists a unique $\beta \in \C{}$ such that $\alpha + \beta = 0$.}{unique_complex_additive_inverse}
			\begin{proof}
			By contradiction: Say there are two such elements, $\beta, \gamma$ such that,
			\begin{align*}
			\alpha + \beta &= 0 = \alpha + \gamma \\
			(\alpha + \beta) + \beta &= (\alpha + \beta) + \gamma \\
			0 + \beta &= \beta = 0 + \gamma = \gamma \qedhere\\
			\end{align*}
			\end{proof}
			
			\labeledProposition{For every $\alpha \in \C{}$ with $\alpha \neq 0$, there exists a unique $\beta \in \C{}$ such that $\alpha\beta = 1$.}{unique_complex_multiplicative_inverse}
			\begin{proof}
			By contradiction: Say there are two such elements, $\beta, \gamma$ then,
			\begin{align*}
			\alpha\beta &= 1 = \alpha\gamma \\
			\beta &= \frac{1}{\alpha} = \gamma \qedhere\\
			\end{align*}
			\end{proof}
		}
		
		\bigskip\bigskip
		\searchableSubsection{Complex Numbers Problems}{abstract algebra, complex numbers}{\bigskip\bigskip}
		
		\searchable{subsubsection}{Find all the roots of $x^3 = 1$ for $x \in \C{}$}{complex numbers}{
		Since $x^3 - 1 = (x - 1)(x^2 + x + 1)$, we have (via zero-factor theorem) possible roots from,
		\[ x - 1 = 0 \iff x = 1 \]
		\begin{align*}
		x^2 + x + 1 &= 0
		\implies x = \frac{-1 \pm \sqrt{-3}}{2} = \frac{-1 \pm \sqrt{3}\,i}{2}
		\end{align*}
		More generally,
		\[ (a+bi)+(a-bi)=2a \] 
		and since also,
		\[ \left[\frac{-1 + \sqrt{3}\,i}{2}\right]^2 = \frac{-1 - \sqrt{3}\,i}{2} \]
		as well as the reverse,
		\[ \left[\frac{-1 - \sqrt{3}\,i}{2}\right]^2 = \frac{-1 + \sqrt{3}\,i}{2} \]
		this means that if $x = \frac{-1 \pm \sqrt{3}\,i}{2}$ then $x^2 + x$ is of the form $(a+bi)+(a-bi) = 2a$ and so we have that $x^2 + x = -1 \iff x^2 + x + 1 = 0$.\\\\
		In addition,
		\[ (a+bi)(a-bi)=a^2+b^2 \]
		which means that if $x = \frac{-1 \pm \sqrt{3}\,i}{2}$ then $x^3 = x^2x$ is of the form $(a+bi)(a-bi)=a^2+b^2$ so we have
		that $x^3 = {\frac{-1}{2}}^2 + {\frac{\sqrt{3}}{2}}^2 = \frac{1}{4} + \frac{3}{4} = 1$.\\\\
		So we see that - allowing for complex $x$ - the cubic polynomial $x^3 - 1$ has 3 roots as we should expect from the \href{https://en.wikipedia.org/wiki/Fundamental_theorem_of_algebra}{Fundamental Theorem of Algebra} \textit{\color{red}{(is this the correct interpretation of this?)}}.
		}

\end{document}