\documentclass[MathsNotesBase.tex]{subfiles}




\date{\vspace{-6ex}}


\begin{document}
\searchableSubsection{\chapterTitle{Algebra}}{abstract algebra, complex numbers}{\bigskip\bigskip}

	\searchableSubsection{\sectionTitle{Abstract Algebra}}{abstract algebra}{\bigskip}
	
	\bigskip\bigskip
	\searchableSubsection{Groups}{abstract algebra}{\bigskip
		
		\boxeddefinition{
		A binary operation is a function,
		\[ f:G \times G \longmapsto G \]
		which - by the definition of a function - maps a unique tuple from $G \times G$ to a unique value in the codomain $G$.
		}
		\bigskip
		
		\boxeddefinition{
		Let $G$ be a set and $*$ a binary operation on $G$ and denote this ($G,*$). Then ($G,*$) is a \textbf{group} if:
		\paragraph*{closure} $\forall x,y \in G, x * y \in G$;
		\paragraph*{associativity} $\forall x,y,z \in G, (x * y) * z = x * (y * z)$;
		\paragraph*{identity} $\exists e \in G \suchthat \forall x \in G, e*x = x*e = x$;
		\paragraph*{inverse} $\forall x \in G, \exists x^{-1} \in G \suchthat x * x^{-1} = x^{-1} * x = e$.\\\\
		These are known as the \textbf{group axioms}.
		}
		
		
		
		\bigskip\smallskip
		\boxeddefinition{
		The group is an \textbf{Abelian group} if it has the additional property:	
		\paragraph*{commutativity} $\forall x,y \in G, x * y = y * x \in G$.
		}
		
		\bigskip\smallskip 
		\notation{from here on we will use juxtaposition notation for the group operation (so $xy = x * y$) and (usually) $1$ for the identity element instead of $e$. This is known as \textit{multiplicative notation}.}
		
		\bigskip\bigskip
		\subsubsection{Corollaries of the group axioms}
		The group operation is defined to map a unique tuple in $G \times G$ to a unique value in $G$ so that if we have $x,y \in G$ then $f((x, y)) = f(x,y) = xy \in G$ and for $a,b,c \in G$,
		\[ a = b \iff (c, a) = (c, b) \implies f((c, a)) = f((c, b)) \iff ca = cb \]
		\[ \therefore a = b \implies ca = cb \]
		
		Then, using all the group axioms - associativity, inverse and identity,
		\[ ca = cb \implies \inv{c}(ca) = \inv{c}(cb) \iff (\inv{c}c)a = (\inv{c}c)b \iff 1a = 1b \iff a = b \]
		
		Therefore we have the principle of cancellation,
		\[ ca = cb \implies a = b \]
		\textit{Note} that, since we have used the axioms of inverse and identity and the definitions of these require these elements to exhibit these properties from both the left and the right, the principle of cancellation can also be shown from both the left and the right. So, also,
		\[ ac = bc \implies a = b \]
		
		\bigskip\bigskip
		There are (at least) two approaches to finding the other consequences of the group axioms.
		\paragraph*{First approach.} We begin by noticing that the law of cancellation implies that,
		\subparagraph*{unique identity and inverses}$\forall a,x,b \in G, ax = b$ has a unique solution because,
		\[ ax = ax' \iff x = x' \]
		That unique solution is $a^{-1}b$. If $b=a$ we have $ax=a$ and $x$, by identity axiom, is an identity element. Since, the solution to this equation - $x$ - is unique, it follows that there is a unique value that is the identity element. Then, if we let $b$ be this unique identity element we have $ax=1$ and the unique solution, $x$, is the inverse of a, i.e. $\inv{a}$. Therefore, the inverses of group elements are also unique.
		
		\paragraph*{Second approach.} This approach begins by showing the uniqueness of the identity element solely using the defintion of the identity. Here, for clarity, we revert to using $e$ to denote the identity element.
		\subparagraph*{unique identity} Assume there are two identity elements, $e, e'$. Then, by the definition of the identity $ee' = e'e = e = e'$ so that there is a single value that has the property of the identity element.\\\\
		Then, using the definition of the inverse we have,
		\subparagraph{unique inverses} Assume there are two distinct inverses of an element $a$: $\inv{a}$ and $a'$. Then, 
		\begin{align*}
		&& a\inv{a} = 1 &= aa' &\sidecomment{defn. of inverse, uniqueness of identity}\\
		& \iff & \inv{a} &= a' &\sidecomment{law of cancellation}
		\end{align*}
		\bigskip\bigskip

		\paragraph*{Some Examples of Groups}
		\begin{itemize}
		\item{ ($\R{} \setminus \{0\}, \times)$ } is a group whereas ($\R{}, \times)$ is not a group because $0$ has no multiplicative inverse.
		\item{ ($\R{}, +)$ } is a group.
		\item{ The set of $n \times n$ invertible matrices is called the General Linear group and denoted $GL_n$  }
		\end{itemize}
			\subsubsection{Permutations and Symmetric Groups}\bigskip
		\boxeddefinition{
			A \textbf{permutation} is a bijection from a set to itself. Since permutations are bijective, they are invertible and since they are functions, function composition defines an associative law of composition over them. As a result, they form a group. 
		}
		\boxeddefinition{
			The \textbf{symmetric group} defined over a set is the group whose elements are the permutations of the objects of the set and whose law of composition is the composition of functions.
		}
		\notation{The symmetric group over the integers from 1 to n is denoted $S_n$.}\\
		\subparagraph{$\bm{S_2}$}
		The symmetric group $S_2$ consists of the two elements $i$ and $\tau$ which are, respectively, the identity map and the transposition which interchanges 1 and 2. The group composition law is described by the fact that the identity map is the identity of the composition and by the relation $\tau\tau = \tau^2 = i$. Which results in the multiplication table:	
		\[
		\begin{aligned}
			i &\cdot i &&= i \\
			i &\cdot \tau &&= \tau \\		
			\tau &\cdot i &&= \tau \\
			\tau &\cdot \tau &&= i \\
		\end{aligned}
		\]
		Note that the law of composition is commutative.
		\subparagraph{$\bm{S_3}$}
		The symmetric group $S_3$ contains $3!$ elements. It is the smallest group whose law of composition is not commutative. It can be described using any two permutations of $\{1, 2, 3\}$. For example, if we take,
		\begin{align*}
			x =
			\begin{bmatrix}
			0 & 1 & 0 \\
			0 & 0 & 1 \\
			1 & 0 & 0 \\
			\end{bmatrix},\;
			y =
			\begin{bmatrix}
			0 & 1 & 0 \\
			1 & 0 & 0 \\
			0 & 0 & 1 \\
			\end{bmatrix}
		\end{align*}
		Then the permutations are,
		\[ \{1, x, x^2, y, xy, x^2y\} = \setc{x^iy^j}{0 \leq i \leq 2,\; 0 \leq j \leq 1} \]
		These are the elements of the group. The composition law over these elements is the function composition of these permutation functions and its multiplication table is characterized by the rules:
		\begin{align*}
			x^3 = 1,\; y^2 = 1,\; yx = x^2y
		\end{align*}
		These are derived directly from the permutations themselves. Note that this composition law is not associative as $yx \neq xy$. \\
		Any product of the elements $x,y$ and of their inverses can be brought into the form $x^iy^j$ with $i,j$ taking the ranges given above by repeated application of the above rules. To do so, we move all occurrences of $y$  to the right side using the last relation and bring the exponents into the indicated ranges using the first two relations:
		\begin{align*} 
			\inv{x}y^3x^2y &= x^2yx^2y = x^2(yx)xy = x^2(x^2y)xy = x^4(yx)y\\
						&= x^4(x^2y)y = x^6y^2 = (x^3)^2y^2 = 1 \cdot 1 = 1
		\end{align*}
		Rules like these that determine a complete multiplication table are called \textit{defining relations} for the group.
	}
	
	
	\pagebreak
	\searchableSubsection{Subgroups}{abstract algebra}{\bigskip
		\boxeddefinition{
			A subset $H$ of a group $G$ is called a \textbf{subgroup} if it has the following properties:
			\begin{itemize}
				\item{\textbf{Closure:}} If $a \in H$ and $b \in H$ then $ab \in H$.
				\item{\textbf{Identity:}} $1 \in H$.
				\item{\textbf{Inverses:}} If $a \in H$ then $\inv{a} \in H$.
			\end{itemize}
		}
		These conditions show that the subset $H$ is a group with respect to the \textit{induced law of composition} created by applying the law of composition of $G$ on the members of $H$. Note that the associative property is not mentioned because the associativity of the composition of members of $G$ automatically carries over to $H$.\\\\
		
		\noindent\fbox{%
				\parbox{\textwidth}{%
				Every group, at a minimum, has two trivial subgroups: the maximal subgroup - the group itself; and the minimal subgroup - the set containing just the identity. A subgroup that is neither of these is known as a \mbox{\textit{proper subgroup}}.
			}%
		}\bigskip
		
		Important examples are the subgroups of the additive group of integers $\Z{+}$. Denote the subset of $\Z{+}$ consisting of all multiples of a given integer $b$ by $b\Z{}$ such that,
		\[ b\Z{} = \setc{n \in \Z{}}{n = bk,\; k \in \Z{}} \]
		\labeledProposition{For any integer $b$, the subset $b\Z{}$ is a subgroup of $\Z{+}$ and every subgroup of $\Z{+}$ is of the form $b\Z{}$ for some integer $b$.}{bZ_subgroup}
		\begin{proof}
			$b\Z{}$ is a subgroup of $\Z{+}$ because,
			\begin{itemize}
				\item{$b(0) = 0 \in b\Z{}$;}
				\item{If $a_1,a_2 \in b\Z{}$ then $a_1 = bk_1, a_2 = bk_2$ for $k_1, k_2 \in \Z{}$ and so $a_1 + a_2 = bk_1 + bk_2 = b(k_1 + k_2) \in b\Z{}$}
				\item{For any $a = bk \in b\Z{}$, $-a = b(-k) \in b\Z{}$}
			\end{itemize}
			Now we need to prove that any subgroup of $\Z{+}$ is $b\Z{}$ for some $b$. Let $H$ be an arbitrary subgroup of $\Z{+}$. Then by subgroup properties,
			\begin{itemize}
				\item{$0 \in H$;}
				\item{If $a_1,a_2 \in H$ then $a_1 + a_2 \in H$}
				\item{For any $a \in H$, $-a \in H$}
			\end{itemize}
			We proceed to show that there is always some integer $b$ such that $H = b\Z{}$.\\
			Firstly, if $H$ is the minimal subgroup \{0\} then $H$ trivially conforms to $b\Z{}$ with $b = 0$.\\
			Otherwise, $\exists a \in H \suchthat a \neq 0$ then also $\exists -a \in H \suchthat -a \neq 0$. One of these must be a positive non-zero integer so there is at least one such member of $H$. We take $b$ to be the smallest positive non-zero integer in $H$. Then,
			\paragraph{$\bm{b\Z{} \in H}$}
			\begin{itemize}
				\item{$b \in H$ (by selection) so by subgroup properties $b + b \in H$ and $(b + b) + b \in H$ and $b + \cdots + b \in H$}
				\item{By subgroup properties $b \in H \implies -b \in H$}
			\end{itemize}
			So, $\setc{bk \in \Z{}}{k \in \Z{}}$ is in $H$.
			\paragraph{$\bm{H \in b\Z{}}$}
			Take any $n \in H$. Using division with remainder and dividing by $b$ we get,
			\[ n = bq + r \;\; q \in \Z{},\, 0 \leq r < b \]
			But, since $b\Z{} \in H$ this means that $bq \in H$ and so $-bq \in H$. Therefore $n - bq = r \in H$. But $0 \leq r < b$ and, by assumption, $b$ is the smallest positive \textit{non-zero} integer in $H$ and so, $r = 0$. So, every $n \in H$ divides by $b$.
		\end{proof}
	
		\subsubsection{Greatest Common Divisor}
		If we extend this to groups which are generated by two integers $a,b$, then we have a subgroup of $\Z{+}$,
		\[ a\Z{} + b\Z{} = \setc{n \in \Z{}}{n = ar + bs \;\; r, s \in \Z{}} \]
		This is known as the subgroup \textit{generated} by $a,b$ because it is the smallest subgroup which contains $a$ and $b$. \autoref{prop:bZ_subgroup} tells us that it has the form $d\Z{}$ for some integer $d$.
		\begin{corollary}
			If $d$ is the positive integer which generates the subgroup $a\Z{} + b\Z{}$ then $d$ is the greatest common divisor of $a$ and $b$ and so,
			\begin{itemize}
				\item{$d$ can be written in the form $d = ar + bs$ for some integers $r$ and $s$.}
				\item{$d$ divides $a$ and $b$.}
				\item{If an integer $e$ divides $a$ and $b$, it also divides $d$.}
			\end{itemize}
		\end{corollary} 
		\begin{proof}
			The first property follows directly from the definition of the subgroup. The second property is a result of the fact that $a,b$ are in the subgroup $a\Z{} + b\Z{}$ so that $d\Z{} = a$ and $d\Z{} = b$. The third property is evident because $d = ar + bs = ek_1r + ek_2s = e(k_1r + k_2s)$.
		\end{proof}
	
		\subsubsection{Cyclic Subgroups}
		If we take a single member of a group (along with its inverse and the identity), the subgroup generated by that element takes the form (using multiplicative notation),
		\[ H = \{ x^{-(n-1)},\, \cdots , x^{-2}, x^{-1}, 1, x, x^2,\, \cdots , x^{n-1} \} \]
		where, either, $x^n = 1$ so that there are $n$ distinct values in the group, or else $n$ is infinite and the values never repeat.
		
		\labeledProposition{The set $S$ of integers $n$ such that $x^n = 1$ is a subgroup of $\Z{+}$}{cyclic_powers_add_subgroup_integers}
		\begin{proof}
			If $x^m = 1$ and $x^n = 1$, then $x^{m+n} = x^mx^n = 1$ also so we have closure of addition. Since $x^0 = 1$, 0 is in the subgroup so we have an identity. Finally, for some $n$ in the subgroup, $x^n = 1 \iff x^{-n} = x^nx^{-n} = x^0 = 1$ so $n$ being in the subgroup implies that $-n$ is also in the subgroup and we have inverses.
		\end{proof}
		\begin{corollary}
			It follows from $S$ being a subgroup of $\Z{+}$ and from \autoref{prop:bZ_subgroup} that $S$ has the form $m\Z{}$ where $m$ is the smallest positive integer such that $x^m = 1$. Therefore, in $H$, the $m$ elements $1, x, x^2, \cdots , x^{m-1}$ are all different and any element in $H$ will simplify to one of them: for $n \in S$, $n = mq + r$ such that $x^n = (x^m)^qx^r = 1^qx^r = x^r$.
		\end{corollary}
		
		\boxeddefinition{The \textbf{order} of a group $G$ is the number of distinct elements it contains where an object and its inverse are not considered - for this purpose - distinct. It is typically denoted $\vert G \vert$.\\
			An element of a group is said to have \textbf{order} $m$ (possibly infinity) if the cyclic subgroup it generates has order $m$. This means that $m$ is the smallest positive integer with the property $x^m = 1$ or, if the order is infinite, that, $x^m \neq 1$ for all $m \neq 0$.
		}
		\paragraph{Examples}
		\paragraph{Cyclic group with order 3}
		\[ G = {1, x, x^2} \]
		where $x^3 = 1$ is a cyclic group of order 3 generated by the element $x$. Note that, in fact, since this is a group, it must also contain the inverses, $x^{-1}, x^{-2}$. 
		\paragraph{Cyclic group with infinite order}
		\[
			\begin{bmatrix}
			1 & 1	\\
			0 & 1 	\\
			\end{bmatrix} 
		\]
		under matrix multiplication (which is commutative in this case), generates a cyclic group of infinite order because
		\[
			\begin{bmatrix}
			1 & 1	\\
			0 & 1 	\\
			\end{bmatrix}^n =
			\begin{bmatrix}
			1 & n	\\
			0 & 1 	\\
			\end{bmatrix}
		\].
		\paragraph{The Klein Four Group, $V$} is the simplest group that is not cyclic (it cannot be generated by a single element). It appears in many forms but, as an example, it can be realized as the group consisting of the four matrices,
		\[
			\begin{bmatrix}
			\pm 1 & 0 		\\
				0 & \pm 1 	\\
			\end{bmatrix} 
		\]
		Any two non-identity elements generate $V$.
	}

	
	\bigskip\bigskip
	\searchableSubsection{Isomorphisms}{abstract algebra}{\bigskip
		\boxeddefinition{An \textbf{isomorphism} is a bijection between two groups that preserves the structure of the groups by being compatible with the law of composition of both groups. More formally, two groups are \textbf{isomorphic} if there exists a bijection $\phi : G \longmapsto G'$ such that,
			\[ \phi(ab) = \phi(a)\phi(b) \text{ for all } a,b \in G \]
			where $ab$ represents composition according to the law of composition of $G$ and $\phi(a)\phi(b)$ represents composition according to the law of composition of $G'$.
		}
	
		\labeledProposition{As a consequence of this sole property that, across the bijection, the respective laws of composition are preserved, all other properties of the groups are also preserved.}{isomorphism_preserves_structure}
		\begin{proof}
			Let $e' = \phi(e) \in G'$, then,
			\begin{itemize}
				\item{Since $G'$ is a group, it is closed under its law of composition and so $ \phi(ab) = \phi(a)\phi(b) \in G' $ which shows that if $ab \in G$ then $\phi(ab) \in G'$.}
				\item{$ e' = \phi(e) = \phi(ee) = \phi(e)\phi(e) = e'e' $ which implies that $e'$ is the identity in $G'$ so that $\phi$ maps the identity in $G$ to the identity in $G'$.}
				\item{Since $G'$ is a group, every element in $G'$ has an inverse so, 
					\[ e' = \phi(e) = \phi(a\inv{a}) = \phi(a)\phi(\inv{a}) \iff \inv{\phi(a)} = \phi(\inv{a}) \]
					which shows that $\phi$ maps $\inv{a} \in G$ to $\inv{\phi(a)} \in G'$.}
			\end{itemize}
		\end{proof}
		
		 For example, if $e \in G$ is the identity of $G$ mapped to an element $ e' = \phi(e) \in G' $, then for any $a \in G$ mapped to $a' = \phi(a) \in G'$,
		\[ a' = \phi(a) = \phi(ea) = \phi(e)\phi(a) = e'a' \]
		And $a' = e'a' = a'e'$ means that $e'$ is the identity in $G'$. Furthermore, the order of elements in $G$ and $G'$ will also be the same as,
		\[ a^n = e \iff e' = \phi(e) = \phi(a^n) = \phi(a)^n = (a')^n \]
		
		Since two isomorphic groups have the same properties, it is often convenient to identify them with each other when speaking informally. For example, the symmetric group $S_n$ of permutations of $\{1,\cdots,n\}$ is isomorphic to the group of permutation matrices, a subgroup of $GL_n(\R{})$ and we often blur the distinction between these two groups.\\
		
		\notation{Sometimes when two groups are isomorphic this is indicated using the notation,
			\[G \approx G'\]
		}
	
		\subsubsection{Examples}
		\begin{itemize}
			\item{
				Let $C = \{ \cdots, a^{-2}, a^{-1}, 1, a, a^2, \cdots \}$ be an infinite cyclic group. Then the map,
				\[ \phi : \Z{+} \longmapsto C \; \suchthat \; \phi(n) = a^n \]
				is an isomorphism where the preservation of the respective laws of composition can be seen as,
				\[ \phi(m + n) = a^{m + n} = a^ma^n = \phi(m)\phi(n) \]
			}
			\item{Let $G$ be the set of real matrices of the form,
				\[
				\begin{bmatrix}
				1 & x 	\\
				0 & 1 	\\
				\end{bmatrix} 
				\]
				This is a subgroup of $GL_2(\R{})$ and so, its law of composition is the same as that of $GL_2(\R{})$, i.e. matrix multiplication.
				\[
				\begin{bmatrix}
				1 & x 	\\
				0 & 1 	\\
				\end{bmatrix}
				\begin{bmatrix}
				1 & y 	\\
				0 & 1 	\\
				\end{bmatrix} = 
				\begin{bmatrix}
				1 & x + y\\
				0 & 1 	\\
				\end{bmatrix}
				\]
				So, $G$ is isomorphic to $\R{+}$, the additive group of reals.
			}
			\item{Any two cyclic groups of the same order are isomorphic because, if
				\[ G = \{1, x, x^2, \cdots, x^{n-1}\}, G' = \{1, y, y^2, \cdots, y^{n-1}\} \]
				are two cyclic groups of order $n$ then the map $\phi(x^i) = y^i$ is an isomorphism.
			}
		\end{itemize}
		
		\boxeddefinition{The groups isomorphic to a given group $G$ form what is called the \textbf{isomorphism class} of $G$. Groups are often classified into isomorphism classes, for example, there is one isomorphism class of groups of order 3 and there are two classes of groups of order 4 and five classes of 12.}
		
		\subsubsection{Automorphisms}
		\boxeddefinition{The domain and codomain of an isomorphism can be the same set of objects so that $\phi : G \longmapsto G$. This is known as an \textbf{automorphism}.}
		\paragraph{Example}
		Let $G = \{1, x, x^2\}$ be a cyclic group of order 3 so that $x^3 = 1$. The transposition which interchanges $x$ and $x^2$ is an automorphism of $G$,
		\[\begin{array}{*6c}
			&1 &&\longmapsto &&1 \\
			&x &&\longmapsto &&x^2 \\
			&x^2 &&\longmapsto &&x \\
		\end{array}\]
		This is because $x$ and $x^2$ have the same order because $x^3 = 1$ and also $(x^2)^3 = x^6 = (x^3)^2 = 1^2 = 1$ and so the law of composition is preserved.
		\paragraph{Conjugation}
		The most important example of automorphism is conjugation.\\\\
		\boxeddefinition{\textbf{Conjugation} by $b \in G$ is the map from $G$ to itself defined by,
			\[ \phi(a) = ba\inv{b} \]
			with the result that,
			\[ ba = \phi(a)b \]
			so that we can think of conjugation of $a$ by $b$ as the way that we need to change $a$ if we want to move the multiplication by $b$ to the other side.
		}
		This is an automorphism because it
		\begin{itemize}
			\item{is compatible with law of composition, 
				\[ \phi(xy) = bxy\inv{b} = bx\inv{b}by\inv{b} = \phi(x)\phi(y) \]
			}
			\item{has an inverse so it is bijective, 
				\[ (\inv{\phi} \circ \phi)(a) = \inv{\phi}(\phi(a)) = \inv{b}(ba\inv{b})b = (\inv{b}b)a(\inv{b}b) = a \]
				Note that this is different from the inverse element of $a$ corresponding under the mapping $\phi$,
				\[ \phi(a)\phi(\inv{a}) = ba\inv{b}b\inv{a}\inv{b} = ba(1)\inv{a}\inv{b} = b(1)\inv{b} = 1 \]
			}
		\end{itemize}
		Another thing to note is that - in an abelian group where the composition law is commutative - conjugation becomes the identity map,
		\[ ba = ab \iff ba\inv{b} = a \iff \phi(a) = a \]
	}

	
	\bigskip\bigskip
	\searchableSubsection{Homomorphisms}{abstract algebra}{\bigskip
		\boxeddefinition{A \textbf{homomorphism} is a mapping (not necessarily bijective) between two groups, $\phi : G \longmapsto G'$, such that,
			\[ \phi(ab) = \phi(a)\phi(b) \text{ for all } a,b \in G \]
			where $ab$ represents composition according to the law of composition of $G$ and $\phi(a)\phi(b)$ represents composition according to the law of composition of $G'$. 
		}
		So, the difference between a \textit{homomorphism} and a \textit{isomorphism} is that the latter is bijective whereas the former is not. As a result, a \textit{homomorphism} may be one-way only.
		
		\paragraph{Examples of homomorphisms}
		\begin{itemize}
			\item{
				Let $C = \{ a^{n-1}, \cdots, a^{-2}, a^{-1}, 1, a, a^2, \cdots, a^{n-1} \}$ be a finite cyclic group. Then the map,
				\[ \phi : \Z{+} \longmapsto C \; \suchthat \; \phi(n) = a^n \]
				is a homomorphism. Note that if $C$ were an infinite cyclic group then this would be an isomorphism. 
			}
			\item{the sign of a permutation $sign : S_n \longmapsto {\pm1}$}
			\item{the determinant function $det : GL_n(\R{}) \longmapsto \R{\times}$}
			\item{an arguably trivial example is called the \textit{inclusion} map $i : H \longmapsto G$ of a subgroup $H$ into a group $G$, defined by $i(x) = x$. It functions as the identity for elements in the subgroup $H$ but, since it is not surjective, there is no inverse mapping.}
		\end{itemize}
		
		\subsubsection{Image of a homomorphism}
		Since a homomorphism is not bijective it has an image different to the codomain group,
		\[ im \; \phi = \setc{x \in G'}{\exists a \in G \suchthat \phi(a) = x} \]
		and it is a subgroup of the codomain group $G'$.\\\\
		\notation{The image of the mapping $\phi$ with domain $G$ is sometimes denoted $\phi(G)$.}
		
		\subsubsection{Kernel of a homomorphism}
		\boxeddefinition{The \textbf{kernel} of a homomorphism is the set of elements in the domain that are mapped to the identity,
			\[ ker \; \phi = \setc{a \in G}{\phi(a) = e'} \]
		}
		The kernel of a homomorphism is a subgroup of the domain group $G$. This can be seen as if $a,b \in ker \; \phi$ then,
		
		\begin{itemize}
			\item{closure: $ \phi(ab) = \phi(a)\phi(b) = e' \cdot e' = e' \iff ab \in ker \; \phi $ }
			\item{identity: $ \phi(e) = \phi(ee) = \phi(e)\phi(e) \iff  $ }
			\item{inverses: $ \phi(e) = \phi(a\inv{a}) = \phi(a)\phi(\inv{a})  $}
		\end{itemize}
	}
	
	
	
	\pagebreak
	\searchableSubsection{\sectionTitle{Fields}}{abstract algebra}{\bigskip}
		
		\bigskip
		\searchableSubsection{Complex Numbers}{abstract algebra, complex numbers}{\bigskip
			\labeledProposition{For every $\alpha \in \C{}$, there exists a unique $\beta \in \C{}$ such that $\alpha + \beta = 0$.}{unique_complex_additive_inverse}
			\begin{proof}
			By contradiction: Say there are two such elements, $\beta, \gamma$ such that,
			\begin{align*}
			\alpha + \beta &= 0 = \alpha + \gamma \\
			(\alpha + \beta) + \beta &= (\alpha + \beta) + \gamma \\
			0 + \beta &= \beta = 0 + \gamma = \gamma \qedhere\\
			\end{align*}
			\end{proof}
			
			\labeledProposition{For every $\alpha \in \C{}$ with $\alpha \neq 0$, there exists a unique $\beta \in \C{}$ such that $\alpha\beta = 1$.}{unique_complex_multiplicative_inverse}
			\begin{proof}
			By contradiction: Say there are two such elements, $\beta, \gamma$ then,
			\begin{align*}
			\alpha\beta &= 1 = \alpha\gamma \\
			\beta &= \frac{1}{\alpha} = \gamma \qedhere\\
			\end{align*}
			\end{proof}
		}
		
		\bigskip\bigskip
		\searchableSubsection{Complex Numbers Problems}{abstract algebra, complex numbers}{\bigskip\bigskip}
		
		\searchable{subsubsection}{Find all the roots of $x^3 = 1$ for $x \in \C{}$}{complex numbers}{
		Since $x^3 - 1 = (x - 1)(x^2 + x + 1)$, we have (via zero-factor theorem) possible roots from,
		\[ x - 1 = 0 \iff x = 1 \]
		\begin{align*}
		x^2 + x + 1 &= 0
		\implies x = \frac{-1 \pm \sqrt{-3}}{2} = \frac{-1 \pm \sqrt{3}\,i}{2}
		\end{align*}
		More generally,
		\[ (a+bi)+(a-bi)=2a \] 
		and since also,
		\[ \left[\frac{-1 + \sqrt{3}\,i}{2}\right]^2 = \frac{-1 - \sqrt{3}\,i}{2} \]
		as well as the reverse,
		\[ \left[\frac{-1 - \sqrt{3}\,i}{2}\right]^2 = \frac{-1 + \sqrt{3}\,i}{2} \]
		this means that if $x = \frac{-1 \pm \sqrt{3}\,i}{2}$ then $x^2 + x$ is of the form $(a+bi)+(a-bi) = 2a$ and so we have that $x^2 + x = -1 \iff x^2 + x + 1 = 0$.\\\\
		In addition,
		\[ (a+bi)(a-bi)=a^2+b^2 \]
		which means that if $x = \frac{-1 \pm \sqrt{3}\,i}{2}$ then $x^3 = x^2x$ is of the form $(a+bi)(a-bi)=a^2+b^2$ so we have
		that $x^3 = {\frac{-1}{2}}^2 + {\frac{\sqrt{3}}{2}}^2 = \frac{1}{4} + \frac{3}{4} = 1$.\\\\
		So we see that - allowing for complex $x$ - the cubic polynomial $x^3 - 1$ has 3 roots as we should expect from the \href{https://en.wikipedia.org/wiki/Fundamental_theorem_of_algebra}{Fundamental Theorem of Algebra} \textit{\color{red}{(is this the correct interpretation of this?)}}.
		}

\end{document}